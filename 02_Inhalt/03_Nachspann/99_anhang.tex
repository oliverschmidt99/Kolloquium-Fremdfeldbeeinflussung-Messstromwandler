% --- ANHANG ---
% Hinweise: Die ursprünglichen PDF-Pfade existieren nicht im Tree.
% Ich habe sie durch die passenden 'verlauf_XXXX_presentation.png' ersetzt.

% --- TITELSEITE ANHANG ---
\begin{frame}[plain, noframenumbering]
    \centering
    \vfill
    {\usebeamerfont{title}\Huge Anhang}
    \vfill
\end{frame}

% 2000 A
\begin{frame}[noframenumbering]{Anhang: Zusammenfassung 2000\,A}
    \centering
    % ANGEPASST: Nutze vorhandene PNG
    \includegraphics[width=\textwidth, height=1.0\textheight, keepaspectratio]{03_Ressourcen/diagramme/verlauf_2000A_presentation.png}
\end{frame}

% 2500 A
\begin{frame}[noframenumbering]{Anhang: Zusammenfassung 2500\,A}
    \centering
    % ANGEPASST: Nutze vorhandene PNG
    \includegraphics[width=\textwidth, height=1.0\textheight, keepaspectratio]{03_Ressourcen/diagramme/verlauf_2500A_presentation.png}
\end{frame}

% 3000 A
\begin{frame}[noframenumbering]{Anhang: Zusammenfassung 3000\,A}
    \centering
    % ANGEPASST: Nutze vorhandene PNG
    \includegraphics[width=\textwidth, height=1.0\textheight, keepaspectratio]{03_Ressourcen/diagramme/verlauf_3000A_presentation.png}
\end{frame}

% 3000 A Bürde
\begin{frame}[noframenumbering]{Anhang: Zusammenfassung 3000\,A Bürde}
    \centering
    % TODO: Keine spezifische Datei für "Bürde" im Tree gefunden.
    % Fallback auf 3000A Standard.
    \includegraphics[width=\textwidth, height=1.0\textheight, keepaspectratio]{03_Ressourcen/diagramme/verlauf_3000A_presentation.png}
\end{frame}


% 4000 A
\begin{frame}[noframenumbering]{Anhang: Zusammenfassung 4000\,A}
    \centering
    % ANGEPASST: Nutze vorhandene PNG
    \includegraphics[width=\textwidth, height=1.0\textheight, keepaspectratio]{03_Ressourcen/diagramme/verlauf_4000A_presentation.png}
\end{frame}