\section{Fazit und Ausblick}

\begin{frame}{Zusammenfassung der Ergebnisse}
\begin{columns}[c]
  \begin{column}{0.62\textwidth}
    \textbf{Technisch}
    \begin{itemize}
      \item \textbf{Parallel:} ab $>$\SI{2500}{A} Sättigung, bis \SI{-6}{\%}.
      \item \textbf{$\Delta$-Geometrie:} reduziert Fehler um $>$\SI{90}{\%}.
      \item \textbf{FFP:} nahezu immun, aber teurer.
    \end{itemize}

    \vspace{0.25cm}
    \textbf{Wirtschaftlich}
    \begin{itemize}
      \item \textbf{Bestes P/L:} Standardwandler + $\Delta$.
      \item \textbf{FFP lohnt sich:} wenn Umbau nicht möglich (Nachrüstung/Platz).
    \end{itemize}
  \end{column}

  \begin{column}{0.34\textwidth}
    \centering
    \begin{beamercolorbox}[sep=0.35em,center,rounded=true]{cHSblue!10}
      \textbf{Empfehlung}\par\vspace{0.1cm}
      \Huge $\Delta$\par\vspace{0.05cm}
      \normalsize Dreieck
    \end{beamercolorbox}
  \end{column}
\end{columns}

\slidetime{15:00}{16:30}
\keynote{
  \begin{itemize}
    \item Parallel kritisch
    \item $\Delta$ reduziert Fehler $>$90\%
    \item Standard + $\Delta$ meist ausreichend
  \end{itemize}
}
\end{frame}


% --- Folie 2: Ausblick und Handlungsempfehlung ---
\begin{frame}{Ausblick und Handlungsempfehlung}

\textbf{Neuanlagen}
\begin{itemize}
    \item $\Delta$-Anordnung als Standard ab $I > \SI{2500}{A}$
    \item CAD-Vorlagen anpassen (L2-Versatz)
\end{itemize}

\vspace{0.3cm}
\textbf{Bestandsanlagen}
\begin{itemize}
    \item Bei Abrechnungsrelevanz: FFP-Wandler
    \item Kompensationswandler nur bei extremen Störfeldern
\end{itemize}

\vspace{0.3cm}
\begin{beamercolorbox}[sep=0.35em,center,rounded=true]{green!10}
    \centering
    \textit{Klasse 1 ist durch geometrische Optimierung nahezu kostenneutral erreichbar.}
\end{beamercolorbox}

\slidetime{16:30}{18:00}

\keynote{
    \begin{itemize}
        \item Neu: $\Delta$ als Standard
        \item Bestand: FFP
        \item Klasse 1 ohne Mehrkosten möglich
    \end{itemize}
}

\note{
    \textbf{Handlungsempfehlung:}

    \textbf{Neuanlagen:}  
    Ab 2500 A sollte die mittlere Phase (L2) konstruktiv versetzt werden.  
    Die Anpassung der CAD-Standards verursacht kaum Mehrkosten, verhindert jedoch systematisch Sättigungseffekte.

    \textbf{Bestand:}  
    Wenn kein Schienenumbau möglich ist, sind FFP-Wandler die technisch sichere Lösung.

    \textbf{Fazit:}  
    Die Genauigkeitsklasse 1 kann auch bei hohen Strömen eingehalten werden – primär durch Geometrie, nicht durch teurere Wandler.
}
\end{frame}
