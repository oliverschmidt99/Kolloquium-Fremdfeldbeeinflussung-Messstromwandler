\section{Fazit und Ausblick}

% --- Folie 1: Bewertung und Zusammenfassung ---
\begin{frame}{Bewertung der Lösungsansätze}
    % Schriftgröße für die ganze Folie etwas verkleinern, damit es passt
    \small 
    
    \begin{columns}[c, onlytextwidth]
      \begin{column}{0.65\textwidth}
        
        % 1. Kompensierte Wandler
        \textbf{1. Kompensierte Wandler}
        \begin{itemize} \setlength\itemsep{0pt} % Kein Abstand zwischen Punkten
          \item Technisch führend (höchste Genauigkeit)
          \item Investitionskosten Faktor 2 bis 6 höher
        \end{itemize}
    
        \vspace{0.3cm} % Manueller Abstand, aber kontrolliert
        
        % 2. FFP
        \textbf{2. Fremdfeld-Protektion (FFP)}
        \begin{itemize} \setlength\itemsep{0pt}
          \item Hohe Genauigkeit bei Ausrichtung
          \item Ideal zur Nachrüstung im Bestand
        \end{itemize}
    
        \vspace{0.3cm}
        
        % 3. Dreieck
        \textbf{3. Dreiecksanordnung (Standard)}
        \begin{itemize} \setlength\itemsep{0pt}
          \item Normerfüllung durch Geometrie
          \item Preis-Leistungs-Sieger
        \end{itemize}
        
        \vspace{0.2cm}
        \textit{\footnotesize Fazit: Alle Technologien reduzieren L2-Verzerrung.}
      \end{column}
    
      \begin{column}{0.32\textwidth}
        \centering
        
        % Empfehlungs-Box
        \begin{beamercolorbox}[sep=0.5em,center,rounded=true]{cHSblue!10}
          \textbf{Empfehlung}\par\vspace{0.1cm}
          \Huge $\Delta$\par\vspace{0.05cm}
          \normalsize Dreieck
        \end{beamercolorbox}
        
        \vspace{0.5cm}
        
        % Strategie-Box
        \begin{beamercolorbox}[sep=0.5em,center,rounded=true]{green!10}
            \centering
            \footnotesize 
            \textbf{Neu:} Dreieck\\ 
            \vspace{0.1cm}
            \textbf{Bestand:} FFP\\
            \vspace{0.1cm}
            \textbf{Präzise:} Kompensiert
        \end{beamercolorbox}
      \end{column}
    \end{columns}
    
    \slidetime{15:00}{16:30}
    \keynote{
      \begin{itemize}
        \item Kompensiert: Präzise aber teuer
        \item FFP: Lösung für Bestand
        \item Dreieck: Wirtschaftlichste Lösung
      \end{itemize}
    }
    \note{
        \textbf{Detaillierte Bewertung}
        Alle untersuchten Technologien konnten die starke Verzerrung auf der Phase L2 verbessern. Die Unterschiede liegen in der Anwendung:
    
        \textbf{Kompensierte Wandler}
        Diese Technologie ist messmethodisch führend und bietet maximale Genauigkeit. Ökonomisch betrachtet sind sie jedoch um den Faktor 2 bis 6 teurer als Standardwandler.
    
        \textbf{FFP-Technologie}
        Die Protektoren arbeiten sehr genau, solange sie korrekt zum Magnetfeld ausgerichtet sind. Ihre Stärke liegt in der Nachrüstbarkeit alter Anlagen.
    
        \textbf{Dreiecksanordnung}
        Sie ist der klare Preis-Leistungs-Sieger. Durch die einfache geometrische Verschiebung halten günstige Standardwandler die Normvorgaben sicher ein.
    }
\end{frame}


% --- Folie: Bewertung Dreiecksanordnung (Fokus) ---
\begin{frame}{Bewertung: Dreiecksanordnung (Empfehlung)}

    \begin{columns}[c]
        % Linke Spalte: Text (55%)
        \begin{column}{0.55\textwidth}
            \begin{itemize}
                \item Preis-Leistungs-Sieger im Vergleich
                \item Normerfüllung durch geometrische Verschiebung im Nennstrombereich realisiert
                \item Einsatz kostengünstiger Standardwandler möglich
            \end{itemize}

            \begin{beamercolorbox}[sep=0.5em,center,rounded=true]{cHSblue!10}
                \textbf{Fazit}
                \par\vspace{0.1cm}
                \small
                Die Verzerrung auf Leiter L2 wird korrigiert und die Genauigkeitsklasse 1 sicher eingehalten.
            \end{beamercolorbox}
        \end{column}

        % Rechte Spalte: Bild (45%)
        \begin{column}{0.45\textwidth}
            \centering
            % Bild kann jetzt größer sein, da mehr Platz vorhanden ist
            \includegraphics[width=\linewidth, height=0.75\textheight, keepaspectratio]{03_Ressourcen/Bilder/kupferschinen_gesamtaufbau_3d.png}
            \par\vspace{0.1cm} \tiny Realisierung der Dreiecksanordnung
        \end{column}
    \end{columns}

    \slidetime{16:30}{18:00}

    \keynote{
        \begin{itemize}
            \item Einfache Verschiebung
            \item Große Wirkung
            \item Kosteneffizienteste Lösung
        \end{itemize}
    }

    \note{
        \textbf{Bewertung der Dreiecksanordnung}
        Die Dreiecksanordnung in Kombination mit günstigen Standardwandlern zeigt sich in unseren Messungen als wahrer Preis-Leistungs-Gewinner.

        \textbf{Funktionsweise}
        Durch die einfache geometrische Verschiebung der mittleren Phase konnten wir die Messwerte im Nennstrombereich direkt verbessern. Die starke Verzerrung auf dem Leiter L2 wird physikalisch kompensiert.

        \textbf{Ergebnis}
        Damit ist es möglich, die geforderte Norm einzuhalten, ohne auf teure Spezialwandler zurückgreifen zu müssen.
    }
\end{frame}