\section{Fazit und Ausblick}

% --- Folie 1: Bewertung und Zusammenfassung ---
\begin{frame}{Bewertung der Lösungsansätze}
  % Schriftgröße für die ganze Folie etwas verkleinern, damit es passt
  \small

  \begin{columns}[c, onlytextwidth]
    \begin{column}{0.65\textwidth}

      % 1. Kompensierte Wandler
      \textbf{1. Kompensierte Wandler}
      \begin{itemize} \setlength\itemsep{0pt} % Kein Abstand zwischen Punkten
        \item Technisch führend (höchste Genauigkeit)
        \item Investitionskosten Faktor 2 bis 6 höher
      \end{itemize}

      \vspace{0.3cm} % Manueller Abstand, aber kontrolliert

      % 2. FFP
      \textbf{2. Fremdfeld-Protektion (FFP)}
      \begin{itemize} \setlength\itemsep{0pt}
        \item Hohe Genauigkeit bei Ausrichtung
        \item Ideal zur Nachrüstung im Bestand
      \end{itemize}

      \vspace{0.3cm}

      % 3. Dreieck
      \textbf{3. Dreiecksanordnung (Standard)}
      \begin{itemize} \setlength\itemsep{0pt}
        \item Normerfüllung durch Geometrie
        \item Preis-Leistungs-Sieger
      \end{itemize}

      \vspace{0.2cm}
      \textit{\footnotesize Fazit: Alle Technologien reduzieren L2-Verzerrung.}
    \end{column}

    \begin{column}{0.32\textwidth}
      \centering

      % Empfehlungs-Box
      \begin{beamercolorbox}[sep=0.5em,center,rounded=true]{cHSblue!10}
        \textbf{Empfehlung}\par\vspace{0.1cm}
        \Huge $\Delta$\par\vspace{0.05cm}
        \normalsize Dreieck
      \end{beamercolorbox}

      \vspace{0.5cm}

      % Strategie-Box
      \begin{beamercolorbox}[sep=0.5em,center,rounded=true]{green!10}
        \centering
        \footnotesize
        \textbf{Neu:} Dreieck\\
        \vspace{0.1cm}
        \textbf{Bestand:} FFP\\
        \vspace{0.1cm}
        \textbf{Präzise:} Kompensiert
      \end{beamercolorbox}
    \end{column}
  \end{columns}

  \slidetime{16:50}{17:30}
  \keynote{
    \begin{itemize}
      \item Kompensiert: Präzise aber teuer
      \item FFP: Lösung für Bestand
      \item Dreieck: Wirtschaftlichste Lösung
    \end{itemize}
  }
  \note{
    Alle Ansätze reduzieren die L2-Verzerrung, aber mit unterschiedlichem Einsatzfokus:
    \textbf{Kompensiert} = beste Genauigkeit, teuer (Faktor 2--6).
    \textbf{FFP} = gut für Nachrüstung/Bestand (Ausrichtung/Bauraum beachten).
    \textbf{Dreieck} = Systemlösung für Neuentwicklung: Normrobust im Nennbereich bei geringen Mehrkosten.
  }
\end{frame}


% --- Folie: Customer Win (nur Ergebnisse) ---
\begin{frame}{Customer Win (Rolf Janssen): Standardwandler + Dreiecksanordnung}

  \begin{columns}[c]
    % Linke Spalte: Ergebnisse
    \begin{column}{0.60\textwidth}

      \Large
      \textbf{\(\approx\) \SI{1000}{\euro} Einsparung}\\
      \normalsize
      pro dreiphasigem Feld

      \vspace{0.6cm}

      \Large
      \textbf{Normkonforme Messungen}\\
      \normalsize
      im Nennbereich (80--100\,\% Last) trotz Fremdfeld

    \end{column}

    % Rechte Spalte: Bild
    \begin{column}{0.40\textwidth}
      \centering
      \includegraphics[width=\linewidth, height=0.75\textheight, keepaspectratio]{03_Ressourcen/Bilder/kupferschinen_gesamtaufbau_3d.png}
      \par\vspace{0.1cm}
      \tiny Realisierung der Dreiecksanordnung (Leiterführung)
    \end{column}
  \end{columns}

  \slidetime{17:30}{18:00}

  \keynote{
    \begin{itemize}
      \item \(\approx\) \SI{1000}{\euro} weniger pro Feld
      \item Normkonform im Nennbereich (80--100\,\%) trotz Fremdfeld
    \end{itemize}
  }

  \note{
    Hintergrundzahl: Standard \SI{210}{\euro} vs. kompensiert \SI{1205.73}{\euro} \(\Rightarrow\) \(\approx\) \SI{1000}{\euro} pro dreiphasigem Feld.
    Mehrwert: im Nennbereich (80--100\,\%) normkonform trotz Fremdfeld (v.\,a. L2 stabil).
    Voraussetzung: Leiterführung im Bauraum umsetzbar.
  }
\end{frame}
