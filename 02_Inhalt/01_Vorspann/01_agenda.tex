% --- Folie 3b: Mini-Agenda ---
\begin{frame}{Agenda}
    % Zeigt nur Sections (Hauptkapitel), keine Subsections

    \tableofcontents[hideallsubsections]

    \slidetime{00:25}{00:45}
    \keynote{
        \begin{itemize}
            \item Übersicht des Vortrags
            \item Roter Faden
        \end{itemize}
    }
    \note{
        \textbf{Struktur:}
        Ich habe meine Präsentation in folgende Bereiche gegliedert, um Sie durch das Thema zu führen.

        \textbf{Einleitung:}
        Ich beginne mit der Motivation und der Problemstellung. Daraus leite ich die konkrete Zielsetzung meiner Arbeit ab.

        \textbf{Hauptteil:}
        Anschließend gehe ich kurz auf die theoretischen Grundlagen und den Versuchsaufbau ein. Im Kernteil präsentiere ich Ihnen dann die Messergebnisse und deren Analyse.

        \textbf{Abschluss:}
        Zum Schluss fasse ich die Erkenntnisse zusammen und gebe einen Ausblick auf mögliche Handlungsfelder für die Zukunft.
    }
\end{frame}