\section{Einleitung}

\subsection{Motivation und Problemstellung}

% --- Folie 1: Motivation ---
\begin{frame}{Motivation und Problemstellung}
    \begin{columns}[c, onlytextwidth]
        % Text-Spalte verkleinert auf 0.40, um dem Bild mehr Raum zu geben
        \begin{column}{0.40\textwidth}
            \begin{itemize}
                \item Trend zu hoher Leistungsdichte auf minimalem Bauraum
                \item Hohe Primärströme bei geringen Schienenabständen
                \item Starke magnetische Fremdfeldkopplung
            \end{itemize}
        \end{column}

        % Bild-Spalte vergrößert auf 0.58
        \begin{column}{0.58\textwidth}
            \centering
            \includegraphics[width=\linewidth, keepaspectratio]{03_Ressourcen/Bilder/schaltschrank_front.jpg}
            \par\vspace{0.1cm} {\tiny \color{gray} Kompakte Feldverteilung}
        \end{column}
    \end{columns}

    % Zeit: 45s (01:30 - 02:15)
    \slidetime{01:30}{02:15}
    \keynote{
        \begin{itemize}
            \item Mehr Leistung auf weniger Raum
            \item Physikalisches Problem: Fremdfelder
            \item L2 misst falsch
        \end{itemize}
    }
    \note{
        \begin{itemize}
            \item \textbf{Trend:} Mehr \textbf{Leistungsdichte} \(\rightarrow\) höhere Ströme bei weniger Bauraum.
            \item \textbf{Ursache:} Kleine Leiterabstände \(\Rightarrow\) starke \textbf{magnetische Fremdfelder}.
            \item \textbf{Worst Case:} \textbf{L2} wird von beiden Nachbarphasen überlagert \(\Rightarrow\) stärkste Einkopplung.
            \item \textbf{Konsequenz:} \textbf{Sättigung} des Kerns \(\rightarrow\) Sekundärstrom sinkt \(\rightarrow\) \textbf{Untererfassung}.
            \item \textbf{Ziel:} Robustheit gegen Fremdfeld bei vertretbaren \textbf{Kosten} und \textbf{Verfügbarkeit}.
        \end{itemize}
}
\end{frame}


% --- Kombi-Folie: PAC + Wirtschaftlichkeit (nur Ergebnisse) ---
\begin{frame}{Messabweichung \& wirtschaftliche Relevanz}

    \noindent
    \begin{minipage}[t]{0.32\textwidth}
        \vspace{0pt}
        \centering
        \includegraphics[width=0.90\linewidth, keepaspectratio]{03_Ressourcen/pac/PAC_Anzeige-4000A-Parallel-100p-Referenz.pdf}
        \vspace{0.15cm}

        {\footnotesize \textbf{\color{green!50!black} Referenz (Kl. 0,2S)}}
        \vspace{0.15cm}

        {\scriptsize Ergebnis:\par\vspace{0.05cm}}
        {\Large \textbf{\color{green!50!black} L2: 0,047\%}}
    \end{minipage}
    \hfill
    \begin{minipage}[t]{0.32\textwidth}
        \vspace{0pt}
        \centering
        \includegraphics[width=0.90\linewidth, keepaspectratio]{03_Ressourcen/pac/PAC_Anzeige-4000A-Parallel-100p-Pruefling.pdf}
        \vspace{0.15cm}

        {\footnotesize \textbf{\color{red} Prüfling (Kl. 1,0)}}
        \vspace{0.15cm}

        {\scriptsize Ergebnis:\par\vspace{0.05cm}}
        {\Large \textbf{\color{red} L2: -3,24\%}}
    \end{minipage}
    \hfill
    \begin{minipage}[t]{0.34\textwidth}
        \vspace{0pt}
        \centering % Hier auf Zentrierung umgestellt
        
        \setbeamercolor{lossbox}{bg=cHSblue!10,fg=black}
        \begin{beamercolorbox}[wd=\linewidth, sep=0.8em, rounded=true, shadow=true]{lossbox}
            \centering % Zentriert den Text innerhalb der Box
            \small \textbf{Wirtschaftlicher Verlust}\par\vspace{0.35em}
            {\Large \textbf{\textcolor{red}{$\approx$ 50.000 € / Jahr}}}\par\vspace{0.2em}
            {\scriptsize (Beispielrechnung, Dauerlast)}
        \end{beamercolorbox}
        
        \note{
            \begin{itemize}
                \item \textbf{Vergleich:} Referenz (Kl. 0,2S) vs. Prüfling (Kl. 1,0) bei 4000 A.
                \item \textbf{Worst Case:} \textbf{L2} zeigt $\varepsilon \approx -3,24\%$ $\Rightarrow$ deutliche \textbf{Untererfassung}.
                \item \textbf{Relevanz:} Bei Verrechnung summiert sich der Fehler $\rightarrow$ \textbf{realer Kostenhebel}.
                \item \textbf{Point:} Weniger erfasst $\Rightarrow$ scheinbar gespart, aber für die Abrechnung kritisch.
            \end{itemize}
        }
    \end{minipage}

\end{frame}

\subsection{Zielsetzung der Arbeit}

% --- Folie 3: Zielsetzung ---
\begin{frame}{Zielsetzung der Arbeit}

    % Farben lokal wie bei deiner Loss-Box-Logik
    \setbeamercolor{goalbox}{bg=cHSblue!8, fg=black}
    \setbeamercolor{stepbox}{bg=cHSblue!10, fg=black}

    % Leitfrage (oben, 1 Satz)
    \begin{beamercolorbox}[wd=\linewidth, sep=0.7em, rounded=true, shadow=true]{goalbox}
        \small \textbf{Leitfrage:}
        Wie wird die Strommessung im kompakten Drehstromsystem
        (Worst Case: \textbf{L2}) \textbf{fremdfeldrobust} – bei vertretbaren Kosten/Bauraum?
    \end{beamercolorbox}

    \vspace{0.6em}

    % 3-Schritt-Logik als Kacheln
    \begin{columns}[T,onlytextwidth]
        \begin{column}{0.32\textwidth}
            \begin{beamercolorbox}[wd=\linewidth, sep=0.8em, center, rounded=true, shadow=true]{stepbox}
                \textbf{1 \,Analyse}\par
                {\scriptsize Fehlermechanismen \& Einflussgrößen}
            \end{beamercolorbox}
        \end{column}

        \begin{column}{0.32\textwidth}
            \begin{beamercolorbox}[wd=\linewidth, sep=0.8em, center, rounded=true, shadow=true]{stepbox}
                \textbf{2 \,Vergleich}\par
                {\scriptsize Standard / kompensiert / Spezial \& Geometrien}
            \end{beamercolorbox}
        \end{column}

        \begin{column}{0.32\textwidth}
            \begin{beamercolorbox}[wd=\linewidth, sep=0.8em, center, rounded=true, shadow=true]{stepbox}
                \textbf{3 \,Empfehlung}\par
                {\scriptsize Entscheidungsvorlage für die Neukonstruktion}
            \end{beamercolorbox}
        \end{column}
    \end{columns}

    \vspace{0.5em}

    % Kriterien (unten als „Footline“ der Folie)
    \begin{beamercolorbox}[wd=\linewidth, sep=0.5em, rounded=true]{alerted text}
        \scriptsize \textbf{Bewertung:} Normkonformität \(\cdot\) Fremdfeldrobustheit \(\cdot\) Bauraum \(\cdot\) Kosten \(\cdot\) Verfügbarkeit
    \end{beamercolorbox}

    % Zeit: 30s (04:00 - 04:30)
    \slidetime{04:00}{04:30}
    \keynote{ \begin{itemize} \item Leitfrage \item 3 Schritte \item Kriterien \end{itemize} }
    \note{
        \begin{itemize}
            \item Ziel: Ursache der Abweichung im 3-Phasen-Aufbau (Worst Case L2) verstehen.
            \item Dann Konzepte/Geometrien gegeneinander bewerten.
            \item Ergebnis: klare Empfehlung + Randbedingungen (Norm/Kosten/Bauraum/Verfügbarkeit).
        \end{itemize}
    }
\end{frame}
