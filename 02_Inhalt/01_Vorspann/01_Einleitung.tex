\section{Einleitung}

\subsection{Motivation und Problemstellung}

% --- Folie 1: Motivation ---
\begin{frame}{Motivation und Problemstellung}
    \begin{columns}[c, onlytextwidth]
        % Text-Spalte verkleinert auf 0.40, um dem Bild mehr Raum zu geben
        \begin{column}{0.40\textwidth}
            \begin{itemize}
                \item Trend zu hoher Leistungsdichte auf minimalem Bauraum
                \item Hohe Primärströme bei geringen Schienenabständen
                \item Starke magnetische Fremdfeldkopplung
            \end{itemize}
        \end{column}

        % Bild-Spalte vergrößert auf 0.58
        \begin{column}{0.58\textwidth}
            \centering
            \includegraphics[width=\linewidth, keepaspectratio]{03_Ressourcen/Bilder/schaltschrank_front.jpg}
            \par\vspace{0.1cm} {\tiny \color{gray} Kompakte Feldverteilung}
        \end{column}
    \end{columns}

    % Zeit: 45s (01:30 - 02:15)
    \slidetime{01:30}{02:15}
    \keynote{
        \begin{itemize}
            \item Mehr Leistung auf weniger Raum
            \item Physikalisches Problem: Fremdfelder
            \item L2 misst falsch
        \end{itemize}
    }
    \note{
        \begin{itemize}
            \item \textbf{Trend:} Mehr \textbf{Leistungsdichte} \(\rightarrow\) höhere Ströme bei weniger Bauraum.
            \item \textbf{Ursache:} Kleine Leiterabstände \(\Rightarrow\) starke \textbf{magnetische Fremdfelder}.
            \item \textbf{Worst Case:} \textbf{L2} wird von beiden Nachbarphasen überlagert \(\Rightarrow\) stärkste Einkopplung.
            \item \textbf{Konsequenz:} \textbf{Sättigung} des Kerns \(\rightarrow\) Sekundärstrom sinkt \(\rightarrow\) \textbf{Untererfassung}.
            \item \textbf{Ziel:} Robustheit gegen Fremdfeld bei vertretbaren \textbf{Kosten} und \textbf{Verfügbarkeit}.
        \end{itemize}
}
\end{frame}


% --- Kombi-Folie: PAC + Wirtschaftlichkeit (nur Ergebnisse) ---
\begin{frame}{Messabweichung \& wirtschaftliche Relevanz}


    \noindent
    \begin{minipage}[t]{0.32\textwidth}
        \vspace{0pt}
        \centering
        \includegraphics[width=0.90\linewidth, keepaspectratio, page=1]{03_Ressourcen/pac/PAC_Anzeige-4000A-Parallel-100p-Referenz.pdf}
        \vspace{0.15cm}

        {\footnotesize \textbf{\color{green!50!black} Referenz (Kl.~0,2S)}}
        \vspace{0.15cm}

        {\scriptsize Ergebnis (Fokus):\par\vspace{0.05cm}}
        {\Large \textbf{\color{green!50!black} L2: \SI{0,047}{\percent}}}
    \end{minipage}
    \hfill
    \begin{minipage}[t]{0.32\textwidth}
        \vspace{0pt}
        \centering
        \includegraphics[width=0.90\linewidth, keepaspectratio, page=1]{03_Ressourcen/pac/PAC_Anzeige-4000A-Parallel-100p-Pruefling.pdf}
        \vspace{0.15cm}

        {\footnotesize \textbf{\color{red} Prüfling (Kl.~1,0)}}
        \vspace{0.15cm}

        {\scriptsize Ergebnis (Fokus):\par\vspace{0.05cm}}
        {\Large \textbf{\color{red} L2: \SI{-3,24}{\percent}}}
    \end{minipage}
    \hfill
    \begin{minipage}[t]{0.34\textwidth}
        \vspace{0pt}
        \raggedright

        \setbeamercolor{lossbox}{bg=cHSblue!10,fg=black}
        \begin{beamercolorbox}[wd=\linewidth, sep=0.8em, rounded=true, shadow=true]{lossbox}
            \centering
            \small \textbf{Wirtschaftlicher Verlust}
            \par\vspace{0.35em}
            {\Large \textbf{\color{red} \(\approx\) 50\,000\,€ / Jahr}}
            \par\vspace{0.2em}
            {\scriptsize (Beispielrechnung, Dauerlast)}
            % optional, wenn du's noch knapper willst: diese Zeile löschen
        \end{beamercolorbox}

        \note{
            \begin{itemize}
                \item \textbf{Vergleich:} Referenz (Kl.~0,2S) vs. Prüfling (Kl.~1,0) bei \SI{4000}{A}.
                \item \textbf{Worst Case:} \textbf{L2} zeigt \(\varepsilon \approx \SI{-3,24}{\percent}\) \(\Rightarrow\) deutliche \textbf{Untererfassung}.
                \item \textbf{Relevanz:} Bei Verrechnung/Monitoring summiert sich der Fehler \(\rightarrow\) \textbf{realer Kostenhebel}.
                \item \textbf{Point (mit Augenzwinkern):} \glqq weniger erfasst \(\Rightarrow\) scheinbar gespart\grqq{} \(\rightarrow\) für \textbf{Abrechnung} und \textbf{Schutz} aber kritisch.
            \end{itemize}
}


    \end{minipage}



    \slidetime{02:15}{04:00}
    \keynote{
        \begin{itemize}
            \item PAC-Vergleich: Referenz vs. Prüfling
            \item Worst Case L2: \(\varepsilon \approx \SI{-3,24}{\percent}\)
            \item \(\approx\) 50\,000\,€ / Jahr (Beispiel)
        \end{itemize}
    }

\end{frame}


\subsection{Zielsetzung der Arbeit}

% --- Folie 3: Zielsetzung ---
\begin{frame}{Zielsetzung der Arbeit}
    \hfill
    \begin{itemize}
        \item Systematische Analyse der Fehler im Drehstromsystem
        \item Vergleich von Standard-, Kompensierten- und Spezialwandlern
        \item Ableitung von Handlungsempfehlungen für die Neukonstruktion
    \end{itemize}
    \hfill
    % Zeit: 30s (04:00 - 04:30)
    \slidetime{04:00}{04:30}
    \keynote{ \begin{itemize} \item Leitfrage \item Vergleich \& Empfehlung \end{itemize} }
    \note{
        \begin{itemize}
            \item \textbf{Analyse:} Fehlermechanismen im \textbf{Drehstromsystem} (insb. L2) systematisch verstehen.
            \item \textbf{Vergleich:} \textbf{Standard}, \textbf{kompensierte} und \textbf{Spezial}-Wandler sowie Geometrievarianten.
            \item \textbf{Bewertung:} Normkonformität, Robustheit gegen Fremdfeld und wirtschaftliche Randbedingungen.
            \item \textbf{Output:} Konkrete \textbf{Handlungsempfehlung} für die Neukonstruktion.
        \end{itemize}
}
\end{frame}