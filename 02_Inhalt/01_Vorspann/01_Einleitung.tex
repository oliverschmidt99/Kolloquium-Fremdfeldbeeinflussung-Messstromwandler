\section{Einleitung}

\subsection{Motivation und Problemstellung}

% --- Folie 1: Motivation ---
\begin{frame}{Motivation und Problemstellung}
\begin{columns}[c, onlytextwidth]
    % Text-Spalte verkleinert auf 0.40, um dem Bild mehr Raum zu geben
    \begin{column}{0.40\textwidth}
        \begin{itemize}
            \item Trend zu hoher Leistungsdichte auf minimalem Bauraum
            \item Hohe Primärströme bei geringen Schienenabständen
            \item Starke magnetische Fremdfeldkopplung
        \end{itemize}
    \end{column}
    
    % Bild-Spalte vergrößert auf 0.58
    \begin{column}{0.58\textwidth}
        \centering
        \includegraphics[width=\linewidth, keepaspectratio]{03_Ressourcen/Bilder/schaltschrank_front.jpg}
        \par\vspace{0.1cm} {\tiny \color{gray} Kompakte Feldverteilung}
    \end{column}
\end{columns}

    % Zeit: 45s (01:30 - 02:15)
    \slidetime{00:45}{01:35}
    \keynote{
        \begin{itemize}
            \item Mehr Leistung auf weniger Raum
            \item Physikalisches Problem: Fremdfelder
            \item L2 misst falsch
        \end{itemize} 
    }
    \note{
        \textbf{Trend}
        Schaltanlagen müssen heute immer leistungsfähiger werden, während der verfügbare Bauraum gleich bleibt oder sogar schrumpft. Wir bringen also mehr Strom auf weniger Fläche unter.

        \textbf{Physik}
        Diese Kombination aus hohen Primärströmen und minimalen Abständen führt zwangsläufig zu starken magnetischen Fremdfeldern.

        \textbf{Problem}
        Besonders die mittlere Phase L2 ist diesen Feldern von beiden Seiten ausgesetzt. Das führt zu einer verzerrten Messung und im schlimmsten Fall zur Sättigung des Wandlers.

        \textbf{Ziel}
        Meine Aufgabe war es daher, eine Lösung zu finden, die technisch robust gegen diese Störungen ist, aber wirtschaftlich im Rahmen bleibt.
    }
\end{frame}


% --- Folie 2: Messabweichung ---
\begin{frame}{Problemstellung – Messabweichung}
    {\centering \footnotesize 
    \textbf{Parameter:} Übersetzungsfehler $\varepsilon$ bei $100\,\%~I_{\text{pn}}$ \hspace{0.5cm} 
    \textbf{Grenzwert:} Klasse 1 erlaubt $\pm 1\,\%$ \par}
    
    \vspace{0.2cm}
    \textbf{Vergleich bei \SI{4000}{A} Primärstrom – Symmetrische Last}
    \vspace{0.1cm}
    
    \begin{columns}[T]
        % LINKE SPALTE
        \begin{column}{0.48\textwidth}
            \centering
            \includegraphics[width=0.55\linewidth, keepaspectratio, page=1]{03_Ressourcen/pac/PAC_Anzeige-4000A-Parallel-100p-Referenz.pdf}
            
            \vspace{0.1cm}
            \textbf{\footnotesize \color{green!50!black} Referenz (Sollwert) Kl. 0,2S}
            
            \vspace{0.1cm} \scriptsize
            \begin{tabular}{l r}
                \color{gray} L1 & \color{gray} \SI{-0.005}{\percent} \\ 
                \color{green!50!black}\textbf{L2} & \color{green!50!black}\textbf{\SI{0.047}{\percent}} \\
                \color{gray} L3 & \color{gray} \SI{-0.012}{\percent}
            \end{tabular}
        \end{column}
        
        % RECHTE SPALTE
        \begin{column}{0.48\textwidth}
            \centering
            % ÄNDERUNG: Ebenfalls 0.75\linewidth
            \includegraphics[width=0.55\linewidth, keepaspectratio, page=1]{03_Ressourcen/pac/PAC_Anzeige-4000A-Parallel-100p-Pruefling.pdf}
            
            \vspace{0.1cm}
            \textbf{\footnotesize \color{red} Prüfling (Beeinflusst) Kl. 1,0}
            
            \vspace{0.1cm} \scriptsize
            \begin{tabular}{l r}
                \color{gray} L1 & \color{gray} $\approx$ \SI{-0.704}{\percent} \\ 
                \color{red}\textbf{L2} & \color{red}\textbf{$\approx$ \SI{-3.238}{\percent}} \\
                \color{gray} L3 & \color{gray} $\approx$ \SI{-1.230}{\percent}
            \end{tabular}
        \end{column}
    \end{columns}

    % Deine Notizen bleiben hier gleich...
    \slidetime{01:35}{02:35}
    % ...
\end{frame}

\subsection{Wirtschaftliche Relevanz}

% --- Folie 2b: Problemstellung - Wirtschaftlichkeit ---

\begin{frame}{Problemstellung – Wirtschaftliche Relevanz}
    \vspace{0.2cm}
    \textbf{Analyse der Abweichung}
    \begin{itemize}
        \item Messdifferenz / Untererfassung Phase L2 \(\Delta I \approx \SI{130}{A}\)
        \item Relative Abweichung \(\approx \SI{-3,25}{\%}\)
        \item Kritisch für Netzschutz und Verrechnung
    \end{itemize}
    \vspace{0.2cm}
    \setbeamercolor{bluebox}{bg=cHSblue!10,fg=black}
    \begin{center}
        \begin{beamercolorbox}[wd=0.95\textwidth, sep=0.5em, center, rounded=true, shadow=true]{bluebox}
            \small \textbf{Potenzieller Abrechnungsfehler (Beispielrechnung)}
            \par\vspace{0.3em}
            % Formelzeile für Klarheit
            \footnotesize \( \Delta E = U_{\text{L-N}} \cdot \Delta I_{\text{L2}} \cdot \cos \varphi \cdot t \)
            \par\vspace{0.3em}            
            \renewcommand{\arraystretch}{1.05}
            \small
            \begin{tabular}{rl}
                                              & \(\Delta I_{\text{L2}} = \SI{130}{A}\) bei \SI{230}{V} (L-N) \\
                \(\times\)                    & Leistungsfaktor (\(\cos \varphi = 0,9\))               \\
                \(\times\)                    & Dauerlast (\SI{8760}{h/a})                             \\
                \(\times\)                    & Strompreis (\SI{0,20}{}~\text{€}/\text{kWh})           \\
                \hline
                \rule{0pt}{1.2em} \(\approx\) & \textbf{\Large \color{red} 47\,000\,\text{€} / Jahr}
            \end{tabular}
        \end{beamercolorbox}
    \end{center}

    % Zeit: 45s (03:15 - 04:00)
    \slidetime{02:35}{03:20}
    \keynote{ \begin{itemize} \item 130 A Untererfassung \item 47k€ Differenz \end{itemize} }
    \note{
        \textbf{Messdifferenz:}
        Durch die Abweichung fehlen uns in der Phase L2 dauerhaft etwa 130 Ampere im gemessenen Stromwert, was wir als Untererfassung bezeichnen.

        \textbf{Kosten:}
        Rechnet man diesen Fehler über die Formel für die Wirkleistung auf ein Jahr Dauerlast hoch, ergibt sich eine Abrechnungsdifferenz von knapp 47.000 Euro.

        \textbf{Relevanz:}
        Dieser Wert gilt pro Abgangsfeld, was das wirtschaftliche Risiko für den Betreiber verdeutlicht.
    }
\end{frame}

\subsection{Zielsetzung der Arbeit}

% --- Folie 3: Zielsetzung ---
\begin{frame}{Zielsetzung der Arbeit}
    \textbf{Leitfrage der Untersuchung}
    \vspace{0.2em}
    \begin{beamercolorbox}[sep=0.5em,center,rounded=true,shadow=false]{cHSblue!10}
        \textit{Welche Kombination aus Wandler, Geometrie und FFP hält Klasse 1 unter Fremdfeldbedingungen ein – bei minimalen Mehrkosten?}
    \end{beamercolorbox}
    
    \vspace{0.5cm}
    \textbf{Untersuchungsschwerpunkte}
    \begin{itemize}
        \item Systematische Analyse der Fehler im Drehstromsystem (L1, L2, L3)
        \item Vergleich von Standard-, Kompensierten- und Spezialwandlern
        \item Ableitung von Handlungsempfehlungen für die Neukonstruktion
    \end{itemize}

    % Zeit: 30s (04:00 - 04:30)
    \slidetime{03:20}{03:45}
    \keynote{ \begin{itemize} \item Leitfrage \item Vergleich \& Empfehlung \end{itemize} }
    \note{

        \textbf{Vorgehen:}
        Dazu analysiere ich die Fehler im System, vergleiche verschiedene Wandlertechnologien und prüfe geometrische Anpassungen.

        \textbf{Output:}
        Am Ende steht eine klare Handlungsempfehlung für die Konstruktion neuer Anlagen.
    }
\end{frame}