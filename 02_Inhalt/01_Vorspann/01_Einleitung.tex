\section{Einleitung}

\subsection{Motivation und Problemstellung}

% --- Folie 1: Motivation ---
\begin{frame}{Motivation und Problemstellung}
    \begin{columns}[c, onlytextwidth]
        % Text-Spalte verkleinert auf 0.40, um dem Bild mehr Raum zu geben
        \begin{column}{0.40\textwidth}
            \begin{itemize}
                \item Trend zu hoher Leistungsdichte auf minimalem Bauraum
                \item Hohe Primärströme bei geringen Schienenabständen
                \item Starke magnetische Fremdfeldkopplung
            \end{itemize}
        \end{column}

        % Bild-Spalte vergrößert auf 0.58
        \begin{column}{0.58\textwidth}
            \centering
            \includegraphics[width=\linewidth, keepaspectratio]{03_Ressourcen/Bilder/schaltschrank_front.jpg}
            \par\vspace{0.1cm} {\tiny \color{gray} Kompakte Feldverteilung}
        \end{column}
    \end{columns}

    % Zeit: 45s (01:30 - 02:15)
    \slidetime{00:45}{01:35}
    \keynote{
        \begin{itemize}
            \item Mehr Leistung auf weniger Raum
            \item Physikalisches Problem: Fremdfelder
            \item L2 misst falsch
        \end{itemize}
    }
    \note{
        \textbf{Trend}
        Schaltanlagen müssen heute immer leistungsfähiger werden, während der verfügbare Bauraum gleich bleibt oder sogar schrumpft. Wir bringen also mehr Strom auf weniger Fläche unter.

        \textbf{Physik}
        Diese Kombination aus hohen Primärströmen und minimalen Abständen führt zwangsläufig zu starken magnetischen Fremdfeldern.

        \textbf{Problem}
        Besonders die mittlere Phase L2 ist diesen Feldern von beiden Seiten ausgesetzt. Das führt zu einer verzerrten Messung und im schlimmsten Fall zur Sättigung des Wandlers.

        \textbf{Ziel}
        Meine Aufgabe war es daher, eine Lösung zu finden, die technisch robust gegen diese Störungen ist, aber wirtschaftlich im Rahmen bleibt.
    }
\end{frame}


% --- Kombi-Folie: PAC + Wirtschaftlichkeit (nur Ergebnisse) ---
\begin{frame}{Messabweichung \& wirtschaftliche Relevanz}


    \noindent
    \begin{minipage}[t]{0.32\textwidth}
        \vspace{0pt}
        \centering
        \includegraphics[width=0.90\linewidth, keepaspectratio, page=1]{03_Ressourcen/pac/PAC_Anzeige-4000A-Parallel-100p-Referenz.pdf}
        \vspace{0.15cm}

        {\footnotesize \textbf{\color{green!50!black} Referenz (Kl.~0,2S)}}
        \vspace{0.15cm}

        {\scriptsize Ergebnis (Fokus):\par\vspace{0.05cm}}
        {\Large \textbf{\color{green!50!black} L2: \SI{0,047}{\percent}}}
    \end{minipage}
    \hfill
    \begin{minipage}[t]{0.32\textwidth}
        \vspace{0pt}
        \centering
        \includegraphics[width=0.90\linewidth, keepaspectratio, page=1]{03_Ressourcen/pac/PAC_Anzeige-4000A-Parallel-100p-Pruefling.pdf}
        \vspace{0.15cm}

        {\footnotesize \textbf{\color{red} Prüfling (Kl.~1,0)}}
        \vspace{0.15cm}

        {\scriptsize Ergebnis (Fokus):\par\vspace{0.05cm}}
        {\Large \textbf{\color{red} L2: \SI{-3,24}{\percent}}}
    \end{minipage}
    \hfill
    \begin{minipage}[t]{0.34\textwidth}
        \vspace{0pt}
        \raggedright

        \setbeamercolor{lossbox}{bg=cHSblue!10,fg=black}
        \begin{beamercolorbox}[wd=\linewidth, sep=0.8em, rounded=true, shadow=true]{lossbox}
            \centering
            \small \textbf{Wirtschaftlicher Verlust}
            \par\vspace{0.35em}
            {\Large \textbf{\color{red} \(\approx\) 47\,000\,€ / Jahr}}
            \par\vspace{0.2em}
            {\scriptsize (Beispielrechnung, Dauerlast)}
            % optional, wenn du's noch knapper willst: diese Zeile löschen
        \end{beamercolorbox}

        \note{
            \textbf{Witz (Augenzwinkern):}
            Man könnte das Fremdfeld fast als \glqq Feature\grqq{} verkaufen:
            \glqq Sie sparen sich ca. 47\,000\,€ pro Jahr \textendash{} wird verbraucht, aber nicht erfasst.\grqq{}
            \par\smallskip
            \textbf{Spaß beiseite:} Genau deshalb ist das für Verrechnung und Schutz kritisch.
        }


    \end{minipage}


\end{frame}


\subsection{Zielsetzung der Arbeit}

% --- Folie 3: Zielsetzung ---
\begin{frame}{Zielsetzung der Arbeit}
    \textbf{Leitfrage der Untersuchung}
    \vspace{0.2em}
    \begin{beamercolorbox}[sep=0.5em,center,rounded=true,shadow=false]{cHSblue!10}
        \textit{Welche Kombination aus Wandler, Geometrie und FFP hält Klasse 1 unter Fremdfeldbedingungen ein – bei minimalen Mehrkosten?}
    \end{beamercolorbox}

    \vspace{0.5cm}
    \textbf{Untersuchungsschwerpunkte}
    \begin{itemize}
        \item Systematische Analyse der Fehler im Drehstromsystem (L1, L2, L3)
        \item Vergleich von Standard-, Kompensierten- und Spezialwandlern
        \item Ableitung von Handlungsempfehlungen für die Neukonstruktion
    \end{itemize}

    % Zeit: 30s (04:00 - 04:30)
    \slidetime{03:20}{03:45}
    \keynote{ \begin{itemize} \item Leitfrage \item Vergleich \& Empfehlung \end{itemize} }
    \note{

        \textbf{Vorgehen:}
        Dazu analysiere ich die Fehler im System, vergleiche verschiedene Wandlertechnologien und prüfe geometrische Anpassungen.

        \textbf{Output:}
        Am Ende steht eine klare Handlungsempfehlung für die Konstruktion neuer Anlagen.
    }
\end{frame}