\section{Einleitung}

% --- Folie 1: Motivation ---
\begin{frame}{Motivation}
    \textbf{Kontext der Energiewende}
    \begin{itemize}
        \item Dezentralisierung erhöht Anforderungen an die Energieverteilung
        \item Niederspannungsschaltanlagen als zentrale Netzknoten
        \item Steigende Relevanz präziser Abrechnung und Netzstabilität
    \end{itemize}

    \vspace{0.5cm}

    \textbf{Konstruktiver Zielkonflikt}
    \begin{itemize}
        \item Wirtschaftliche Forderung nach kompakten Anlagen
        \item Hohe Packungsdichte der verbauten Komponenten
        \item Führung hoher Ströme auf engem Raum
        \item Räumliche Nähe von Sammelschienen und Messstromwandlern
    \end{itemize}

    \slidetime{01:30}{03:00}
    \keynote{
    \begin{itemize}
        \item Wandel der Anforderungen
        \item Platzmangel vs. Leistung
        \item Konflikt: Kompaktheit und Physik
    \end{itemize} }
    \note{
    \textbf{Stichpunkte zur Motivation:}
    \begin{itemize}
        \item Die Energiewende bringt neue Herausforderungen für die Verteilung
        \item Präzise Messwerte sind Geld wert (Abrechnung)
        \item Gleichzeitig müssen Anlagen immer kompakter und günstiger werden
        \item Das führt dazu, dass wir viel Strom auf wenig Raum haben
        \item Genau hier entsteht der Konflikt zwischen Baugröße und Messgenauigkeit
    \end{itemize} }
\end{frame}


% --- Folie 2: Problemstellung (Deine L2-Folie) ---
\begin{frame}{Problemstellung}


    \vspace{0.6em}
    \begin{center}
        \Large \alert{L2: \SI{130}{A} Messabweichung bei \SI{4000}{A}}\\
        \normalsize (\(\approx\) \SI{-3,28}{\%} – kritisch für Schutz und Abrechnung)
    \end{center}

    \vspace{0.3em}
    \begin{itemize}
        \item \textbf{Beobachtung:} \,Die Messung der Phase \textbf{L2} wird durch Fremdfelder benachbarter Leiter deutlich verfälscht
        \item \textbf{Ursache:} \,Kompakte Bauweise \(\rightarrow\) Messstromwandler und Sammelschienen liegen \textbf{sehr nahe beieinander}
    
        \item \textbf{Beispiel (L2, \SI{230}{V}):} \,\textbf{ca. 47\,000~€/a} mögliche Abrechnungsabweichung (bei \SI{8760}{h/a}, 0{,}20~€/kWh)
    \end{itemize}

% --- NOTIZEN GEHÖREN IN DEN FRAME ---
\slidetime{01:00}{02:00}
\keynote{
    \begin{itemize}
        \item Fokus auf \textbf{L2}: In kompakter Schienenanordnung koppeln Fremdfelder stark ein \(\rightarrow\) Messabweichung.
        \item Die Abweichung wirkt direkt auf \textbf{Energieverrechnung} und kann zudem Schutzorgane beeinflussen.
    \end{itemize}
}
\note{
    \begin{itemize}
        \item \textbf{Überschlag (nur Phase L2):} \(\Delta I=\SI{130}{A}\), \SI{230}{V}, \(\cos\varphi\approx 0{,}9\)
        \item \textbf{Leistungsdifferenz:} ca. \SI{27}{kW}
        \item \textbf{Energie/Jahr:} bei \SI{8760}{h/a} ca. \SI{235}{MWh/a}
        \item \textbf{Kosten/Jahr:} bei 0{,}20~€/kWh ca. 47\,000~€/a
        \item \textbf{Auswirkung:} \,Fehlerhafte Messwerte gefährden \textbf{Verrechnung}
        \item 47\,000~€/a ist pro Feld. 
    \end{itemize}
}
\end{frame}


% --- Folie 3: Zielsetzung ---
\begin{frame}{Zielsetzung der Arbeit}
    \textbf{Evaluation einer Lösung}
    \begin{itemize}
        \item Findung einer technisch zuverlässigen und wirtschaftlichen Konfiguration
        \item Sicherstellung der Messgenauigkeit unter Fremdfeldeinfluss
    \end{itemize}

    \vspace{0.5cm}

    \textbf{Untersuchungsschwerpunkte}
    \begin{itemize}
        \item Systematische Analyse der Fehler im Drehstromsystem (L1, L2, L3)
        \item Vergleich von Standardwandlern und kompensierten Spezialwandlern
        \item Prüfung konstruktiver Maßnahmen durch Anpassung der Leitergeometrie
        \item Ableitung von Handlungsempfehlungen für die Neukonstruktion
    \end{itemize}

    \slidetime{06:00}{07:00}
    \keynote{
    \begin{itemize}
        \item Standard vs. Spezial
        \item Geometrieoptimierung
        \item Wirtschaftlichkeit prüfen
    \end{itemize} }
    \note{
    \textbf{Ziele:}
    \begin{itemize}
        \item Es geht nicht nur um "Messung korrigieren", sondern um die beste Lösung
        \item Brauchen wir teure Spezialwandler? Oder reicht eine bessere Schienenführung?
        \item Ich vergleiche verschiedene Wandlertypen bei Strömen bis 5000 A
        \item Am Ende soll eine klare Empfehlung für die neue Anlagengeneration stehen
    \end{itemize} }
\end{frame}