\section{Einleitung}

% --- Folie 1: Motivation ---
\begin{frame}{Motivation}

    \begin{columns}[c, onlytextwidth]
        % --- Text ---
        \begin{column}{0.55\textwidth}
            \textbf{Herausforderung Kompaktbauweise}
            \vspace{0.5em}
            \begin{itemize}
                \item Minimale Abstände zwischen Hochstromschienen und Wandlern
                \item \textbf{Folge:} Signifikante Messabweichungen durch magnetische Fremdfelder
                \item \textbf{Ziel:} Findung einer technisch robusten und kosteneffizienten Lösung
            \end{itemize}
        \end{column}

        % --- Bild ---
        \begin{column}{0.35\textwidth}
            \centering
            \includegraphics[height=0.7\textheight, keepaspectratio]{03_Ressourcen/Bilder/schaltschrank_front_zugeschnitten.jpg}
            \par\vspace{0.1cm}
            {\tiny \color{gray} Frontansicht der Anlage}
        \end{column}
        
        \begin{column}{0.08\textwidth} \hfill \end{column}
    \end{columns}

    \slidetime{01:30}{03:00}
    \keynote{
        \begin{itemize}
            \item Bild: Enge zeigen
            \item Folge: Abweichung
            \item Ziel: Günstige Lösung
        \end{itemize} }
    \note{
        \textbf{Zum Bild (Herausforderung Kompaktbauweise):}
        Wie Sie an der Frontansicht der Anlage sehen, ist die Packungsdichte extrem hoch. Wir haben hier kaum Spielraum.

        \textbf{Minimale Abstände:}
        Das führt physikalisch dazu, dass die Hochstromschienen und die Messwandler auf engstem Raum nebeneinander liegen.

        \textbf{Folge (Messabweichungen):}
        Die direkte Konsequenz daraus ist, dass die starken Magnetfelder der Nachbarleiter in die Wandler einkoppeln und die Messung verfälschen.

        \textbf{Ziel (Lösung):}
        Mein Ziel ist es daher, eine Lösung zu finden, die technisch robust misst, aber wirtschaftlich bleibt – also ohne die Kosten der Anlage unnötig in die Höhe zu treiben.
    }
\end{frame}

% --- Folie 2a: Problemstellung - Messabweichung ---
\begin{frame}{Problemstellung – Messabweichung}

    \small \textbf{Vergleich bei \SI{4000}{A} Primärstrom – Symmetrische Last}
    \vspace{0.1cm}

    \begin{columns}[T]
        % --- Linke Spalte: Prüfling ---
        \begin{column}{0.48\textwidth}
            \centering
            \includegraphics[height=0.40\textheight, keepaspectratio, page=1]{03_Ressourcen/pac/PAC_Anzeige-4000A-Parallel-100p-Pruefling.pdf}
            \par\vspace{0.1cm}
            \textbf{\footnotesize \color{red} Prüfling (Beeinflusst) Kl. 1,0}
            \vspace{0.1cm}
            \scriptsize
            \begin{tabular}{l r}
                L1:                     & \color{gray} $\approx$ \SI{-0.704}{\percent}        \\
                \color{red}\textbf{L2:} & \color{red}\textbf{$\approx$ \SI{-3.238}{\percent}} \\
                L3:                     & \color{gray} $\approx$ \SI{-1.230}{\percent}
            \end{tabular}
        \end{column}

        % --- Rechte Spalte: Referenz ---
        \begin{column}{0.48\textwidth}
            \centering
            \includegraphics[height=0.40\textheight, keepaspectratio, page=1]{03_Ressourcen/pac/PAC_Anzeige-4000A-Parallel-100p-Referenz.pdf}
            \par\vspace{0.1cm}
            \textbf{\footnotesize \color{green!50!black} Referenz (Sollwert) Kl. 0,2S}
            \vspace{0.1cm}
            \scriptsize
            \begin{tabular}{l r}
                L1:                                & \color{gray} \SI{-0.005}{\percent}                  \\
                \color{green!50!black}\textbf{L2:} & \color{green!50!black}\textbf{\SI{0.047}{\percent}} \\
                L3:                                & \color{gray} \SI{-0.012}{\percent}
            \end{tabular}
        \end{column}
    \end{columns}

    \slidetime{03:00}{04:00}
    \keynote{
        \begin{itemize}
            \item Vergleich Soll/Ist
            \item L2: -3,2 % Fehler
            \item System verzerrt
        \end{itemize} }
    \note{
        \textbf{Vergleich bei 4000 A:}
        Schauen wir uns die konkreten Zahlen an. Rechts sehen Sie die Referenzmessung: Das System ist perfekt symmetrisch belastet.

        \textbf{Prüfling (Links):}
        Links sehen wir den beeinflussten Wandler im Schaltschrank.

        \textbf{Abweichung L2:}
        Sofort fällt die Phase L2 auf. Hier fehlen uns über 3,2 Prozent des Stroms – das ist der Haupteffekt des Fremdfeldes.

        \textbf{Abweichung L1/L3:}
        Aber auch die Außenleiter sind nicht immun. L1 und L3 zeigen ebenfalls Abweichungen zwischen 0,7 und 1,2 Prozent. Das gesamte System ist also verzerrt.
    }
\end{frame}

% --- Folie 2b: Problemstellung - Wirtschaftlichkeit ---
\begin{frame}{Problemstellung – Wirtschaftliche Relevanz}

    \vspace{0.2cm}
    \textbf{Analyse der Abweichung}
    \begin{itemize}
        \item Fehlbetrag Phase L2: \(\Delta I \approx \SI{130}{A}\)
        \item Relative Abweichung: \(\approx \SI{-3,25}{\%}\)
        \item Kritisch für Netzschutz und Verrechnung
    \end{itemize}

    \vspace{0.2cm}

\setbeamercolor{bluebox}{bg=cHSblue!10,fg=black}
    \begin{center}
        \begin{beamercolorbox}[wd=0.85\textwidth, sep=0.5em, center, rounded=true, shadow=true]{bluebox}
            \small
            \textbf{Beispielrechnung (pro Abgangsfeld)}
            \par\vspace{0.2em}
            
            % arraystretch leicht reduziert, damit die 4. Zeile gut passt
            \renewcommand{\arraystretch}{1.05} 
            \begin{tabular}{rl}
                                              & \(\Delta I_{\text{L2}} = \SI{130}{A}\) bei \SI{230}{V} \\
                \(\times\)                    & Leistungsfaktor (\(\cos \varphi = 0,9\))               \\
                \(\times\)                    & Dauerlast (\SI{8760}{h/a})                             \\
                \(\times\)                    & Strompreis (\SI{0,20}{}~\text{€}/\text{kWh})           \\
                \hline
                \rule{0pt}{1.2em} \(\approx\) & \textbf{\Large \color{red} 47\,000\,\text{€} / Jahr}
            \end{tabular}
        \end{beamercolorbox}
    \end{center}

    \slidetime{04:00}{05:00}
    \keynote{
        \begin{itemize}
            \item 130 A fehlen
            \item Rechnung: Dauerlast
            \item Ergebnis: 47k€ Verlust
        \end{itemize} }
    \note{
        \textbf{Fehlbetrag Phase L2:}
        Diese 3,2 Prozent klingen abstrakt, bedeuten aber konkret, dass uns rund 130 Ampere Messwert fehlen.

        \textbf{Kritisch für Verrechnung:}
        Das ist besonders kritisch, wenn diese Messung für die Abrechnung genutzt wird.

        \textbf{Beispielrechnung:}
        Rechnen wir das einmal konservativ hoch: Bei einer Dauerlast über das ganze Jahr und einem Strompreis von 20 Cent...

        \textbf{Ergebnis (47.000 €):}
        ...kommen wir auf einen Fehlbetrag von 47.000 Euro. Und das gilt für ein einziges Abgangsfeld. Das wirtschaftliche Risiko ist also enorm.
    }
\end{frame}

% --- Folie 3: Zielsetzung ---
\begin{frame}{Zielsetzung der Arbeit}
    \textbf{Evaluation einer Lösung}
    \begin{itemize}
        \item Findung einer technisch zuverlässigen und wirtschaftlichen Konfiguration
        \item Sicherstellung der Messgenauigkeit unter Fremdfeldeinfluss
    \end{itemize}

    \vspace{0.2cm}

    \textbf{Untersuchungsschwerpunkte}
    \begin{itemize}
        \item Systematische Analyse der Fehler im Drehstromsystem (L1, L2, L3)
        \item Vergleich von Standardwandlern und kompensierten Spezialwandlern
        \item Prüfung konstruktiver Maßnahmen durch Anpassung der Leitergeometrie
        \item Ableitung von Handlungsempfehlungen für die Neukonstruktion
    \end{itemize}

    \slidetime{04:30}{05:30}
    \keynote{
        \begin{itemize}
            \item Lösung finden
            \item Analyse Fehlerbild
            \item Vergleich der Wandler
            \item Empfehlung geben
        \end{itemize} }
    \note{
        \textbf{Evaluation einer Lösung:}
        Der Kern meiner Arbeit ist es nun, eine Konfiguration zu finden, die technisch zuverlässig misst, aber wirtschaftlich bleibt.

        \textbf{Systematische Analyse:}
        Dazu analysiere ich zunächst systematisch, wie sich der Fehler im Drehstromsystem verhält.

        \textbf{Vergleich von Wandlern:}
        Ich untersuche: Brauchen wir zwingend teure, kompensierte Spezialwandler? Oder performen Standardwandler unter besseren Bedingungen ähnlich gut?

        \textbf{Prüfung konstruktiver Maßnahmen:}
        Parallel dazu prüfe ich, ob wir das Problem konstruktiv lösen können, indem wir einfach die Leitergeometrie anpassen.

        \textbf{Ableitung von Handlungsempfehlungen:}
        Am Ende soll eine klare Empfehlung für die Konstruktion der neuen Schaltanlagengeneration stehen.
    }
\end{frame}