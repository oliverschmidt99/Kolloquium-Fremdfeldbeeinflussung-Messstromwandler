\begin{frame}{Methodik: Versuchsplan und Auswertung}
    \textbf{Untersuchungsparameter}
    \begin{itemize}
        \item Abstand $d$ zwischen Messstromwandler und Störleiter
        \item Störstrom bzw. Fremdfeldstärke (Stromniveau im Störleiter)
        \item Orientierung/Geometrie (Lage von Hin- und Rückleiter)
        \item Phasenlage zwischen Mess- und Störstrom (falls variiert)
    \end{itemize}

    \vspace{0.4cm}

    \textbf{Auswertung}
    \begin{itemize}
        \item Vergleich gegen Referenz (ohne Fremdfeld) $\rightarrow$ $\Delta$ Verhältnis- und Winkelfehler
        \item Sensitivitätsanalyse: Einfluss pro Parameter / Konfiguration
    \end{itemize}
\end{frame}

\begin{frame}{Versuchsaufbau: Messkette}
    \textbf{Aufbau}
    \begin{itemize}
        \item Hochstromquelle bis Nennstrom
        \item Prüfling: Messstromwandler
        \item Referenzwandler (hohe Genauigkeitsklasse) als Vergleich
        \item Variabler Störleiteraufbau mit definierten Abständen und Winkeln
    \end{itemize}

    \vspace{0.4cm}

    \begin{block}{TODO: Schema/Fotografie}
        Aufbau-Skizze oder Foto des Versuchsstandes einfügen (inkl. Abstandsdefinition $d$).
    \end{block}
\end{frame}
