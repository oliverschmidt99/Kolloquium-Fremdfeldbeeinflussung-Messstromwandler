\section{Problemstellung und Lösungsansätze}

\subsection{Wandlertechnologien}

% --- Folie 1: Funktionsprinzip und Aufbau (Standard) ---
\begin{frame}{Funktionsprinzip und Aufbau}
    \begin{columns}[c, onlytextwidth]
        \begin{column}{0.45\textwidth}
            \textbf{Aufgaben des Messstromwandlers}
            \begin{itemize}
                \item Transformation hoher Primärströme (\SI{1}{A} / \SI{5}{A})
                \item Galvanische Trennung
                \item Bündelung des magnetischen Flusses
            \end{itemize}
        \end{column}
        \begin{column}{0.50\textwidth}
            \centering
            % Pfad ist korrekt laut Tree
            \includegraphics[width=1.2\linewidth]{03_Ressourcen/zeichnungen/aufbau_wandler.drawio.pdf}
            \par\vspace{0.2cm} {\tiny Prinzipieller Aufbau eines Aufsteckstromwandlers}
        \end{column}
    \end{columns}

    % Zeit angepasst an neuen Fluss: (05:00 - 05:45)
    \slidetime{03:45}{04:25}
    \keynote{ \begin{itemize} \item Transformator-Prinzip \item Schutzfunktion \end{itemize} }
    \note{
        \textbf{Einordnung}
        Ein Messstromwandler ist im Kern ein Transformator für Ströme: Aus mehreren \SI{1000}{A} Primärstrom wird ein normierter Sekundärstrom von \SI{1}{A} oder \SI{5}{A}, den Schutzrelais und Zähler sicher verarbeiten können.

        \textbf{Aufbau (Aufsteckwandler)}
        Die Kupferschiene bildet dabei praktisch eine \emph{einzige Primärwindung} (\(N_\mathrm{p}=1\)). Um sie herum liegt der geschlossene Eisenkern mit der Sekundärwicklung \(N_\mathrm{s}\). Idealisierend gilt
        \[
        I_\mathrm{s} \approx \frac{N_\mathrm{p}}{N_\mathrm{s}}\, I_\mathrm{p},
        \qquad
        I_\mathrm{p}=N_\mathrm{s} I_\mathrm{s} + I_\mathrm{m}.
        \]
        Der Magnetisierungsstrom \(I_\mathrm{m}\) ist der entscheidende Störanteil: Er wächst stark an, sobald der Kern in die Sättigung gerät.

        \textbf{Warum das später wichtig ist}
        Für Genauigkeitsklasse 1 darf der Übersetzungsfehler nur innerhalb \(\pm 1\,\%\) liegen. Jeder zusätzliche Fluss im Kern (z.\,B. durch Fremdfelder) erhöht \(I_\mathrm{m}\) \(\rightarrow\) der Sekundärstrom wird \emph{zu klein} \(\rightarrow\) wir \emph{untererfassen} den Primärstrom. Genau diesen Mechanismus sehen wir später in den Messergebnissen, besonders auf Phase L2.

        \textbf{Randbedingungen}
        Die Genauigkeit hängt außerdem von der \emph{Bürde} (Sekundärlast) und der Frequenz ab: Je höher die Bürde, desto höher die erforderliche Sekundärspannung und desto früher erreicht der Kern seinen Sättigungsbereich.
    }
\end{frame}

% --- Folie 2: Physik ---
\begin{frame}{Physikalisches Problem: Fremdfeldeinfluss}
    \textbf{Ursache: Räumliche Nähe}
    \begin{itemize}
        \item Starke Magnetfelder der Nachbarleiter koppeln ein
    \end{itemize}
    \vspace{0.4cm}
    \textbf{Wirkung: Partielle Sättigung}
    \begin{itemize}
        \item Nutzfluss + Störfluss = Sättigung im Eisen
        \item Permeabilität $\mu_r$ sinkt
    \end{itemize}
    \vspace{0.3cm}
    \begin{beamercolorbox}[sep=0.5em,center,rounded=true,shadow=false]{alerted text}
        \textbf{Resultat} Der Sekundärstrom sinkt, die Messung zeigt zu wenig an.
    \end{beamercolorbox}

    % Zeit: (07:15 - 08:00)
    \slidetime{04:25}{05:05}
    \keynote{ \begin{itemize} \item Sättigung des Kerns \item Messwert sinkt \end{itemize} }
    \note{
        \textbf{Superposition der Magnetfelder}
        In der Schaltanlage liegen die Leiter sehr dicht beieinander. Das Magnetfeld eines Leiters skaliert näherungsweise mit
        \[
        H(r)\propto \frac{I}{r},
        \]
        d.\,h. hoher Strom und kleiner Abstand sind die ungünstige Kombination. Diese Fremdfelder überlagern sich mit dem Nutzfeld des Wandlers (\emph{Superposition}).

        \textbf{Warum Phase L2 am stärksten betroffen ist}
        Die mittlere Phase L2 wird von \emph{zwei} Nachbarleitern gleichzeitig beeinflusst. Dadurch entsteht am Wandlerkern eine unsymmetrische Flussverteilung: In einzelnen Kernbereichen addiert sich Stör- und Nutzfluss, lokal wird die \(B\)-\(H\)-Kennlinie in die Sättigung gedrückt, während andere Bereiche noch linear sind.

        \textbf{Konsequenz im Ersatzbild}
        Sobald Teile des Kerns sättigen, sinkt die relative Permeabilität \(\mu_r\) stark. Der Magnetisierungsstrom \(I_\mathrm{m}\) steigt an und ``zieht'' einen Teil des Primärstroms, der dann \emph{nicht} mehr als Sekundärstrom erscheint:
        \[
        I_\mathrm{p}=N_\mathrm{s} I_\mathrm{s} + I_\mathrm{m}
        \;\Rightarrow\;
        I_\mathrm{s}\downarrow \text{ bei } I_\mathrm{m}\uparrow.
        \]
        Das äußert sich als negativer Übersetzungsfehler (Untererfassung) und kann zusätzlich den Phasenfehler vergrößern.

        \textbf{Merksatz für die Ergebnisse}
        Je höher der Primärstrom und je enger die Geometrie, desto eher kippt der Kern partiell in die Sättigung \(\rightarrow\) genau dort entsteht der starke Einbruch der L2-Kurve in den 4000-A-Messungen.
    }
\end{frame}


% --- Folie 3: Kompensation ---
\begin{frame}{Lösungsansatz: Kompensierte Wandler}
    \begin{columns}[c]
        % --- Textspalte ---
        \begin{column}{0.58\textwidth}
            \raggedright

            \textbf{Prinzip}
            \vspace{0.15cm}
            \begin{itemize}\setlength{\itemsep}{0.25em}
                \item Zusatzwicklungen erzeugen Gegenfeld
                \item Aktive Fehlerkompensation
            \end{itemize}

            \vspace{0.35cm}

            \textbf{Pro / Contra}
            \vspace{0.15cm}
            \begin{itemize}\setlength{\itemsep}{0.25em}
                \item Weniger Sättigung
                \item \textbf{Contra} teuer, mehr Bauraum
            \end{itemize}
        \end{column}

        % --- Bildspalte ---
        \begin{column}{0.40\textwidth}
            \centering
            \includegraphics[height=0.58\textheight, keepaspectratio]{03_Ressourcen/zeichnungen/aufbau_wandler_kompensiert.drawio.pdf}

            \vspace{0.15cm}
            {\tiny Prinzip der Kompensationswicklung}
        \end{column}
    \end{columns}

    \slidetime{05:05}{05:40}

    \keynote{
        \begin{itemize}\setlength{\itemsep}{0.2em}
            \item Gegenfeld erzeugen
            \item Aber: Teuer \& Groß
        \end{itemize}
    }

    \note{
        \textbf{Kompensation}
        Kompensierte Wandler nutzen Hilfswicklungen auf dem Kern, um aktiv ein magnetisches Gegenfeld zu erzeugen.

        \textbf{Vorteil}
        Dadurch wird die Sättigung des Eisenkerns verhindert, was zu sehr präzisen Messergebnissen führt.

        \textbf{Nachteil}
        Diese Technik ist jedoch spürbar teurer und benötigt Bauraum, der in unseren kompakten Anlagen oft nicht vorhanden ist.
    }
\end{frame}


% --- Folie 4: FFP ---
\begin{frame}{Lösungsansatz: Fremdfeld-Protektion (FFP)}
    \begin{columns}[c]
        \begin{column}{0.55\textwidth}
            \textbf{Konstruktive Optimierung}
            \begin{itemize}
                \item Gezielte Schirmung des Messkerns
                \item Umleitung der magnetischen Störfeldlinien
            \end{itemize}
            \vspace{0.5cm}
            \textbf{Zielsetzung}
            \begin{itemize}
                \item Einhaltung Genauigkeitsklasse 1
                \item Schutz vor partieller Sättigung
            \end{itemize}
        \end{column}
        \begin{column}{0.40\textwidth}
            \centering
            \begin{beamercolorbox}[sep=0.5em,center,rounded=true,shadow=false]{white}
                \centering
                % Pfad ist korrekt laut Tree
                \includegraphics[width=1.2\textwidth]{03_Ressourcen/Bilder/wandler-ffp-redur-patent.png}
                \par\vspace{0.2cm} {\tiny \color{gray} Quelle: Patent DE102021106843A1 (Redur)}
            \end{beamercolorbox}
        \end{column}
    \end{columns}

    % Zeit: (06:30 - 07:15)
    \slidetime{05:40}{06:20}
    \keynote{ \begin{itemize} \item Schirmung = Umleitung \item Nachrüstbar \end{itemize} }
    \note{
        \textbf{Konzept}
        Die FFP-Technologie setzt auf einen ferromagnetischen Schirm, der hier gelb dargestellt ist, um den Messkern zu schützen.

        \textbf{Wirkung}
        Dieser Schirm fängt externe Magnetfelder ein und leitet den Störfluss gezielt am eigentlichen Messkern vorbei.

        \textbf{Nutzen}
        Ein großer Vorteil dieser Lösung ist, dass sich Standardwandler damit kostengünstig nachrüsten lassen.
    }
\end{frame}


% --- Folie 5: Dreiecksanordnung ---
\begin{frame}{Lösungsansatz: Dreiecksanordnung}

    \centering
    \textbf{Realisierung der geometrischen Optimierung ($\Delta$-Anordnung)}


    % Pfad ist korrekt laut Tree
    \includegraphics[height=0.9\textheight, width=0.95\textwidth, keepaspectratio]{03_Ressourcen/zeichnungen/aufbau_wandler_FFP.1.drawio.pdf}

    % Zeit: (08:45 - 09:30)
    \slidetime{06:20}{07:05}
    \keynote{
        \begin{itemize}
            \item L2 versetzt (Dreieck)
            \item Symmetrische Abstände
            \item Feldkompensation
        \end{itemize}
    }
    \note{
        \textbf{Realisierung}
        Dieses Foto zeigt die praktische Umsetzung der geometrisch optimierten Leiterführung direkt im Schaltschrank.

        \textbf{Konstruktion}
        Anstatt alle Schienen flach nebeneinanderzuführen, haben wir die mittlere Phase L2 räumlich um 140 Millimeter versetzt angeordnet.

        \textbf{Effekt}
        Durch das entstehende gleichschenklige Dreieck heben sich die magnetischen Vektoren im Zentrum gegenseitig auf, was den Störeinfluss physikalisch minimiert.
    }
\end{frame}