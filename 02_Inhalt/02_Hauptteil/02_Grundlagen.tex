\section{Grundlagen der Arbeit}

\subsection{Wandlertechnologien}

% --- Folie 1: Funktionsprinzip und Aufbau (Standard) ---
\begin{frame}{Funktionsprinzip und Aufbau}
    \begin{columns}[c, onlytextwidth]
        \begin{column}{0.45\textwidth}
            \textbf{Aufgaben des Messstromwandlers}
            \begin{itemize}
                \item Transformation hoher Primärströme (\SI{1}{A} / \SI{5}{A})
                \item Galvanische Trennung
                \item Bündelung des magnetischen Flusses
            \end{itemize}
        \end{column}
        \begin{column}{0.50\textwidth}
            \centering
            % Pfad ist korrekt laut Tree
            \includegraphics[width=1.2\linewidth]{03_Ressourcen/zeichnungen/aufbau_wandler.drawio.pdf}
            \par\vspace{0.2cm} {\tiny Prinzipieller Aufbau eines Aufsteckstromwandlers}
        \end{column}
    \end{columns}

    % Zeit angepasst an neuen Fluss: (05:00 - 05:45)
    \slidetime{05:00}{05:45}
    \keynote{ \begin{itemize} \item Transformator-Prinzip \item Schutzfunktion \end{itemize} }
    \note{
        \textbf{Aufbau}
        Bei diesen Aufsteckstromwandlern fungiert die durchgeführte Kupferschiene direkt als Primärwicklung.

        \textbf{Funktion}
        Der Eisenkern bündelt den magnetischen Fluss um den Leiter und induziert einen proportionalen Sekundärstrom für die Messgeräte.

        \textbf{Sicherheit}
        Eine wesentliche Aufgabe ist dabei die galvanische Trennung um die empfindliche Messtechnik vom Hochstromnetz zu isolieren.
    }
\end{frame}

% --- Folie 2: Physik ---
\begin{frame}{Physikalisches Problem: Fremdfeldeinfluss}
    \textbf{Ursache: Räumliche Nähe}
    \begin{itemize}
        \item Starke Magnetfelder der Nachbarleiter koppeln ein
    \end{itemize}
    \vspace{0.4cm}
    \textbf{Wirkung: Partielle Sättigung}
    \begin{itemize}
        \item Nutzfluss + Störfluss = Sättigung im Eisen
        \item Permeabilität $\mu_r$ sinkt
    \end{itemize}
    \vspace{0.3cm}
    \begin{beamercolorbox}[sep=0.5em,center,rounded=true,shadow=false]{alerted text}
        \textbf{Resultat} Der Sekundärstrom sinkt, die Messung zeigt zu wenig an.
    \end{beamercolorbox}

    % Zeit: (07:15 - 08:00)
    \slidetime{07:15}{08:00}
    \keynote{ \begin{itemize} \item Sättigung des Kerns \item Messwert sinkt \end{itemize} }
    \note{
        \textbf{Ursache}
        Durch die räumliche Nähe koppeln die starken Magnetfelder der benachbarten Leiter ungewollt in den Wandler ein.

        \textbf{Sättigung}
        Die Summe aus Nutzfluss und Störfluss treibt den Eisenkern in die partielle Sättigung, wodurch die Permeabilität sinkt.

        \textbf{Resultat}
        Das Eisen leitet den magnetischen Fluss schlechter, weshalb der Sekundärstrom einbricht und wir zu wenig messen.
    }
\end{frame}


% --- Folie 3: Kompensation ---
\begin{frame}{Lösungsansatz: Kompensierte Wandler}
    \begin{columns}[c]
        \begin{column}{0.55\textwidth}
            \textbf{Prinzip}
            \begin{itemize}
                \item Zusatzwicklungen erzeugen Gegenfeld
                \item Aktive Fehlerkompensation
            \end{itemize}
            \vspace{0.3cm}
            \textbf{Pro / Contra}
            \begin{itemize}
                \item Weniger Sättigung
                \item \textbf{Contra} teuer, mehr Bauraum
            \end{itemize}
        \end{column}
        \begin{column}{0.40\textwidth}
            \centering
            % Pfad ist korrekt laut Tree
            \includegraphics[width=1\textwidth]{03_Ressourcen/zeichnungen/aufbau_wandler_kompensiert.drawio.pdf}
            \par\vspace{0.2cm} {\tiny Prinzip der Kompensationswicklung}
        \end{column}
    \end{columns}

    % Zeit: (05:45 - 06:30)
    \slidetime{05:45}{06:30}
    \keynote{ \begin{itemize} \item Gegenfeld erzeugen \item Aber: Teuer \& Groß \end{itemize} }
    \note{
        \textbf{Kompensation}
        Kompensierte Wandler nutzen Hilfswicklungen auf dem Kern, um aktiv ein magnetisches Gegenfeld zu erzeugen.

        \textbf{Vorteil}
        Dadurch wird die Sättigung des Eisenkerns verhindert, was zu sehr präzisen Messergebnissen führt.

        \textbf{Nachteil}
        Diese Technik ist jedoch spürbar teurer und benötigt Bauraum, der in unseren kompakten Anlagen oft nicht vorhanden ist.
    }
\end{frame}

% --- Folie 4: FFP ---
\begin{frame}{Lösungsansatz: Fremdfeld-Protektion (FFP)}
    \begin{columns}[c]
        \begin{column}{0.55\textwidth}
            \textbf{Konstruktive Optimierung}
            \begin{itemize}
                \item Gezielte Schirmung des Messkerns
                \item Umleitung der magnetischen Störfeldlinien
            \end{itemize}
            \vspace{0.5cm}
            \textbf{Zielsetzung}
            \begin{itemize}
                \item Einhaltung Genauigkeitsklasse 1
                \item Schutz vor partieller Sättigung
            \end{itemize}
        \end{column}
        \begin{column}{0.40\textwidth}
            \centering
            \begin{beamercolorbox}[sep=0.5em,center,rounded=true,shadow=false]{white}
                \centering
                % Pfad ist korrekt laut Tree
                \includegraphics[width=1.2\textwidth]{03_Ressourcen/Bilder/wandler-ffp-redur-patent.png}
                \par\vspace{0.2cm} {\tiny \color{gray} Quelle: Patent DE102021106843A1 (Redur)}
            \end{beamercolorbox}
        \end{column}
    \end{columns}

    % Zeit: (06:30 - 07:15)
    \slidetime{06:30}{07:15}
    \keynote{ \begin{itemize} \item Schirmung = Umleitung \item Nachrüstbar \end{itemize} }
    \note{
        \textbf{Konzept}
        Die FFP-Technologie setzt auf einen ferromagnetischen Schirm, der hier gelb dargestellt ist, um den Messkern zu schützen.

        \textbf{Wirkung}
        Dieser Schirm fängt externe Magnetfelder ein und leitet den Störfluss gezielt am eigentlichen Messkern vorbei.

        \textbf{Nutzen}
        Ein großer Vorteil dieser Lösung ist, dass sich Standardwandler damit kostengünstig nachrüsten lassen.
    }
\end{frame}


% --- Folie 5: Dreiecksanordnung ---
\begin{frame}{Lösungsansatz: Dreiecksanordnung}

    \centering
    \textbf{Realisierung der geometrischen Optimierung ($\Delta$-Anordnung)}

    \vspace{0.2cm}

    % Pfad ist korrekt laut Tree
    \includegraphics[height=0.75\textheight, width=0.95\textwidth, keepaspectratio]{03_Ressourcen/Bilder/kupferschinen_gesamtaufbau.png}

    % Zeit: (08:45 - 09:30)
    \slidetime{08:45}{09:30}
    \keynote{
        \begin{itemize}
            \item L2 versetzt (Dreieck)
            \item Symmetrische Abstände
            \item Feldkompensation
        \end{itemize}
    }
    \note{
        \textbf{Realisierung}
        Dieses Foto zeigt die praktische Umsetzung der geometrisch optimierten Leiterführung direkt im Schaltschrank.

        \textbf{Konstruktion}
        Anstatt alle Schienen flach nebeneinanderzuführen, haben wir die mittlere Phase L2 räumlich um 140 Millimeter versetzt angeordnet.

        \textbf{Effekt}
        Durch das entstehende gleichschenklige Dreieck heben sich die magnetischen Vektoren im Zentrum gegenseitig auf, was den Störeinfluss physikalisch minimiert.
    }
\end{frame}