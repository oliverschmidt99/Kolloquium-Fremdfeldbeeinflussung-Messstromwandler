\section{Problemstellung und Lösungsansätze}

\subsection{Wandlertechnologien}

% --- Folie 1: Funktionsprinzip und Aufbau (Standard) ---
\begin{frame}{Funktionsprinzip und Aufbau}
    \begin{columns}[c, onlytextwidth]
        \begin{column}{0.45\textwidth}
            \textbf{Aufgaben des Messstromwandlers}
            \begin{itemize}
                \item Transformation hoher Primärströme (\SI{1}{A} / \SI{5}{A})
                \item Galvanische Trennung
                \item Bündelung des magnetischen Flusses
            \end{itemize}
        \end{column}
        \begin{column}{0.50\textwidth}
            \centering
            % Pfad ist korrekt laut Tree
            \includegraphics[width=\linewidth]{03_Ressourcen/zeichnungen/aufbau_wandler.drawio.pdf}
            \par\vspace{0.2cm} {\tiny Prinzipieller Aufbau eines Aufsteckstromwandlers}
        \end{column}
    \end{columns}

    % Zeit angepasst an neuen Fluss: (05:00 - 05:45)
    \slidetime{03:45}{04:25}
    \keynote{ \begin{itemize} \item Transformator-Prinzip \item Schutzfunktion \end{itemize} }
    \note{
        \textbf{Einordnung}
        Ein Messstromwandler ist im Kern ein Transformator für Ströme: Aus mehreren \SI{1000}{A} Primärstrom wird ein normierter Sekundärstrom von \SI{1}{A} oder \SI{5}{A}, den Schutzrelais und Zähler sicher verarbeiten können.

        \textbf{Aufbau (Aufsteckwandler)}
        Die Kupferschiene bildet dabei praktisch eine \emph{einzige Primärwindung} (\(N_\mathrm{p}=1\)). Um sie herum liegt der geschlossene Eisenkern mit der Sekundärwicklung \(N_\mathrm{s}\). Idealisierend gilt
        \[
        I_\mathrm{s} \approx \frac{N_\mathrm{p}}{N_\mathrm{s}}\, I_\mathrm{p},
        \qquad
        I_\mathrm{p}=N_\mathrm{s} I_\mathrm{s} + I_\mathrm{m}.
        \]
        Der Magnetisierungsstrom \(I_\mathrm{m}\) ist der entscheidende Störanteil: Er wächst stark an, sobald der Kern in die Sättigung gerät.

        \textbf{Warum das später wichtig ist}
        Für Genauigkeitsklasse 1 darf der Übersetzungsfehler nur innerhalb \(\pm 1\,\%\) liegen. Jeder zusätzliche Fluss im Kern (z.\,B. durch Fremdfelder) erhöht \(I_\mathrm{m}\) \(\rightarrow\) der Sekundärstrom wird \emph{zu klein} \(\rightarrow\) wir \emph{untererfassen} den Primärstrom. Genau diesen Mechanismus sehen wir später in den Messergebnissen, besonders auf Phase L2.

        \textbf{Randbedingungen}
        Die Genauigkeit hängt außerdem von der \emph{Bürde} (Sekundärlast) und der Frequenz ab: Je höher die Bürde, desto höher die erforderliche Sekundärspannung und desto früher erreicht der Kern seinen Sättigungsbereich.
    }
\end{frame}

% --- Folie 2: Physik (mit Bild) ---
\begin{frame}{Physikalisches Problem: Fremdfeldeinfluss}

\begin{columns}[T,onlytextwidth]
    % --- Links: Kernaussagen ---
    \begin{column}{0.42\textwidth}
        \textbf{Ursache: Räumliche Nähe}
        \begin{itemize}\scriptsize
            \item Starke Magnetfelder der Nachbarleiter koppeln ein
        \end{itemize}

        \vspace{0.2cm}
        \textbf{Wirkung: Partielle Sättigung}
        \begin{itemize}\scriptsize
            \item Nutzfluss + Störfluss \(\Rightarrow\) lokale Sättigung
            \item \(\mu_r\) sinkt \(\Rightarrow\) Magnetisierungsanteil steigt
        \end{itemize}

        \vspace{0.2cm}
        \begin{beamercolorbox}[sep=0.5em,center,rounded=true,shadow=false]{alerted text}
            \textbf{Resultat:} Sekundärstrom sinkt \(\Rightarrow\) Messung zeigt zu wenig an.
        \end{beamercolorbox}
    \end{column}

    % --- Rechts: Simulation ---
    \begin{column}{0.58\textwidth}
        \centering
        \includegraphics[
            width=\linewidth,
            height=0.62\textheight,
            keepaspectratio
        ]{03_Ressourcen/Bilder/sim_2500A_magnetfelder.png}

        \vspace{0.1cm}
        {\tiny \color{gray}
        Simulation: Feldstärkeverteilung \(|H|\) bei \SI{2500}{A} \;—\;
        mittlere Phase (L2) wird von beiden Nachbarphasen überlagert.}
    \end{column}
\end{columns}

% Zeit: (07:15 - 08:00)
\slidetime{04:25}{05:05}
\keynote{ \begin{itemize}\item Fremdfeld koppelt ein \item lokale Sättigung \item Sekundärstrom sinkt \end{itemize} }

\note{
% deine ausführlichen Notizen bleiben wie gehabt
... (dein bestehender \note{}-Text) ...
}
\end{frame}


% --- Sammelfolie: 3 Technologien (Bilder + Überschrift, zentriert, Quelle + Notizen) ---
\begin{frame}{Technologievergleich}

\begin{columns}[T,onlytextwidth] % Überschriften oben bündig
    % --- Links: Kompensiert ---
    \begin{column}{0.33\textwidth}
        \centering
        \textbf{\footnotesize Kompensiert}
        \vspace{0.15cm}

        \begin{minipage}[c][0.72\textheight][c]{\linewidth}
            \centering
            \includegraphics[width=0.90\linewidth,keepaspectratio]{03_Ressourcen/zeichnungen/aufbau_wandler_kompensiert.drawio.pdf}

            % optional: falls du unten noch ein zweites Bild (Schaltbild) hast
            % \vspace{0.25cm}
            % \includegraphics[width=0.95\linewidth,keepaspectratio]{03_Ressourcen/zeichnungen/<DEIN_SCHALTBILD>.pdf}
        \end{minipage}
    \end{column}

    % --- Mitte: FFP (Redur) ---
    \begin{column}{0.33\textwidth}
        \centering
        \textbf{\footnotesize FFP}
        \vspace{0.15cm}

        \begin{minipage}[c][0.72\textheight][c]{\linewidth}
            \centering
            \includegraphics[width=\linewidth,keepaspectratio]{03_Ressourcen/Bilder/wandler-ffp-redur-patent.png}

            \vspace{0.2cm}
            {\tiny \color{gray} Quelle: Patent DE102021106843A1 (Redur)}
        \end{minipage}
    \end{column}

    % --- Rechts: Dreieck ---
    \begin{column}{0.33\textwidth}
        \centering
        \textbf{\footnotesize Dreieck (Standardwandler)}
        \vspace{0.15cm}

        \begin{minipage}[c][0.72\textheight][c]{\linewidth}
            \centering
            \includegraphics[width=\linewidth,keepaspectratio]{03_Ressourcen/zeichnungen/aufbau_wandler_FFP.1.drawio.pdf}
        \end{minipage}
    \end{column}
\end{columns}

\slidetime{05:05}{07:05}
\keynote{\begin{itemize}
    \item Drei Lösungswege: Wandlertechnik / Schirmung / Geometrie
\end{itemize}}

\note{
\textbf{Übersicht (Einordnung)}
Hier zeige ich drei Lösungsprinzipien nebeneinander – von „Wandler intern“ bis „Schaltschrank-Geometrie“.

\textbf{Links: Kompensiert}
Zusatzwicklungen erzeugen ein Gegenfeld. Das stabilisiert den Kern gegen lokale Sättigung und liefert die beste Genauigkeit,
ist aber typischerweise deutlich teurer (mehrere Faktoren gegenüber Standard) und oft schlechter verfügbar.

\textbf{Mitte: FFP}
Ein ferromagnetischer Protektor führt den Störfluss am Messkern vorbei (passive Schirmung/Umleitung).
Das ist ein guter Kompromiss: Standardwandler bleiben möglich und die Messung wird stabiler – allerdings mit Zusatzkosten und Bauraum.

\textbf{Rechts: Dreieck}
Die Geometrie wird so verändert, dass L2 weniger stark von beiden Nachbarphasen „in die Zange“ genommen wird.
Damit kann man Standardwandler weiter verwenden, bezahlt aber mit mehr Kupfer und höherer mechanischer Komplexität im Schaltschrank.

\textbf{Überleitung}
Im nächsten Schritt bewerte ich, welche Option die Klasse~1 unter Fremdfeld bei minimalen Mehrkosten am besten trifft.
}
\end{frame}


