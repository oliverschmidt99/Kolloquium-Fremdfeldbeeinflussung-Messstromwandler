\section{Problemstellung und Lösungsansätze}

\subsection{Wandlertechnologien}

% --- Folie 1: Funktionsprinzip und Aufbau (Standard) ---
\begin{frame}{Funktionsprinzip und Aufbau}
    \begin{columns}[c, onlytextwidth]
        \begin{column}{0.45\textwidth}
            \textbf{Aufgaben des Messstromwandlers}
            \begin{itemize}
                \item Transformation hoher Primärströme (\SI{1}{A} / \SI{5}{A})
                \item Galvanische Trennung
                \item Bündelung des magnetischen Flusses
            \end{itemize}
        \end{column}
        \begin{column}{0.50\textwidth}
            \centering
            % Pfad ist korrekt laut Tree
            \includegraphics[width=\linewidth]{03_Ressourcen/zeichnungen/aufbau_wandler.drawio.pdf}
            \par\vspace{0.2cm} {\tiny Prinzipieller Aufbau eines Aufsteckstromwandlers}
        \end{column}
    \end{columns}

    % Zeit angepasst an neuen Fluss: (05:00 - 05:45)
    \slidetime{03:45}{04:25}
    \keynote{ \begin{itemize} \item Transformator-Prinzip \item Schutzfunktion \end{itemize} }
    \note{
        \textbf{Kurz}
        Messstromwandler = Stromtransformator: \SI{1000}{A} \(\rightarrow\) normierter Sekundärstrom (\SI{1}{A}/\SI{5}{A}) mit galvanischer Trennung.

        \textbf{Prinzip}
        Aufsteckwandler: Kupferschiene entspricht \(N_\mathrm{p}=1\), Sekundärwicklung \(N_\mathrm{s}\).
        Sättigungsnähe erhöht den Magnetisierungsstrom \(I_\mathrm{m}\) \(\Rightarrow\) \(I_\mathrm{s}\) wird kleiner \(\Rightarrow\) Untererfassung.

        \textbf{Relevanz}
        Fremdfeld addiert Störfluss im Kern (Worst Case: L2) und treibt \(I_\mathrm{m}\) hoch.
        Bürde/Last beeinflusst, wie früh Sättigung einsetzt.
    }
\end{frame}

% --- Folie 2: Physik (mit Bild) ---
\begin{frame}{Physikalisches Problem: Fremdfeldeinfluss}

    \begin{columns}[T,onlytextwidth]
        % --- Links: Kernaussagen ---
        \begin{column}{0.42\textwidth}
            \textbf{Ursache: Räumliche Nähe}
            \begin{itemize}\scriptsize
                \item Starke Magnetfelder der Nachbarleiter koppeln ein
            \end{itemize}

            \vspace{0.2cm}
            \textbf{Wirkung: Partielle Sättigung}
            \begin{itemize}\scriptsize
                \item Nutzfluss + Störfluss \(\Rightarrow\) lokale Sättigung
                \item \(\mu_r\) sinkt \(\Rightarrow\) Magnetisierungsanteil steigt
            \end{itemize}

            \vspace{0.2cm}
            \begin{beamercolorbox}[sep=0.5em,center,rounded=true,shadow=false]{alerted text}
                \textbf{Resultat:} Sekundärstrom sinkt \(\Rightarrow\) Messung zeigt zu wenig an.
            \end{beamercolorbox}
        \end{column}

        % --- Rechts: Simulation ---
        \begin{column}{0.58\textwidth}
            \centering
            \includegraphics[
                width=\linewidth,
                height=0.62\textheight,
                keepaspectratio
            ]{03_Ressourcen/Bilder/sim_2500A_magnetfelder.png}

            \vspace{0.1cm}
            {\tiny \color{gray}
                Simulation: Feldstärkeverteilung \(|H|\) bei \SI{2500}{A} \;—\;
                mittlere Phase (L2) wird von beiden Nachbarphasen überlagert.}
        \end{column}
    \end{columns}

    % Zeit: (07:15 - 08:00)
    \slidetime{04:25}{05:05}
    \keynote{ \begin{itemize}\item Fremdfeld koppelt ein \item lokale Sättigung \item Sekundärstrom sinkt \end{itemize} }

    \note{
        Fremdfeld aus Nachbarleitern koppelt in den Eisenkern ein (Worst Case: mittlere Phase L2).
        Nutzfluss + Störfluss \(\Rightarrow\) lokale/partielle Sättigung, \(\mu_r\) sinkt.
        Folge: Magnetisierungsanteil steigt, Sekundärstrom sinkt \(\Rightarrow\) Messung zeigt zu wenig an.
    }
\end{frame}


% --- Sammelfolie: 3 Technologien (Bilder + Überschrift, zentriert, Quelle + Notizen) ---
\begin{frame}{Technologievergleich}

    \begin{columns}[T,onlytextwidth] % Überschriften oben bündig
        % --- Links: Kompensiert ---
        \begin{column}{0.33\textwidth}
            \centering
            \textbf{\footnotesize Kompensiert}
            \vspace{0.15cm}

            \begin{minipage}[c][0.72\textheight][c]{\linewidth}
                \centering
                \includegraphics[width=0.90\linewidth,keepaspectratio]{03_Ressourcen/zeichnungen/aufbau_wandler_kompensiert.drawio.pdf}

                % optional: falls du unten noch ein zweites Bild (Schaltbild) hast
                % \vspace{0.25cm}
                % \includegraphics[width=0.95\linewidth,keepaspectratio]{03_Ressourcen/zeichnungen/<DEIN_SCHALTBILD>.pdf}
            \end{minipage}
        \end{column}

        % --- Mitte: FFP (Redur) ---
        \begin{column}{0.33\textwidth}
            \centering
            \textbf{\footnotesize FFP}
            \vspace{0.15cm}

            \begin{minipage}[c][0.72\textheight][c]{\linewidth}
                \centering
                \includegraphics[width=\linewidth,keepaspectratio]{03_Ressourcen/Bilder/wandler-ffp-redur-patent.png}

                \vspace{0.2cm}
                {\tiny \color{gray} Quelle: Patent DE102021106843A1 (Redur)}
            \end{minipage}
        \end{column}

        % --- Rechts: Dreieck ---
        \begin{column}{0.33\textwidth}
            \centering
            \textbf{\footnotesize Dreieck (Standardwandler)}
            \vspace{0.15cm}

            \begin{minipage}[c][0.72\textheight][c]{\linewidth}
                \centering
                \includegraphics[width=\linewidth,keepaspectratio]{03_Ressourcen/zeichnungen/aufbau_wandler_FFP.1.drawio.pdf}
            \end{minipage}
        \end{column}
    \end{columns}

    \slidetime{05:05}{07:05}
    \keynote{\begin{itemize}
            \item Drei Lösungswege: Wandlertechnik / Schirmung / Geometrie
        \end{itemize}}

    \note{
        Drei Hebel gegen Fremdfeld:
        (1) \textbf{Kompensation} (Gegenfeld) \(\rightarrow\) beste Stabilität, aber teuer/Verfügbarkeit.
        (2) \textbf{FFP} (Störfluss umlenken) \(\rightarrow\) guter Kompromiss, braucht Bauraum.
        (3) \textbf{Dreieck} (Geometrie) \(\rightarrow\) reduziert L2-Einkopplung, nutzt Standardwandler.
        Überleitung: Bewertung nach Normrobustheit vs. Kosten.
    }
\end{frame}


