\section{Grundlagen der Arbeit}

% --- Folie 1: Niederspannungsschaltanlagen ---
\begin{frame}{Niederspannungsschaltanlagen}
    \textbf{Funktion und Aufbau}
    \begin{itemize} 
        \item Verteilung elektrischer Energie auf diverse Abgänge
        \item Kombination von Schutzfunktionen und Messaufgaben
        \item Modulare Anordnung von Einspeisung und Abgangsfeldern
    \end{itemize} 
    
    \vspace{0.5cm}

    \textbf{Einbausituation und Herausforderung}
    \begin{itemize} 
        \item Führung hoher Betriebsströme über Sammelschienen
        \item Ströme im Bereich von \SIrange{900}{6000}{\ampere}
        \item Entstehung magnetischer Störfelder durch kompakte Bauform
        \item Beeinflussung benachbarter Messstromwandler
    \end{itemize} 

    \slidetime{02:00}{02:30} 
    \keynote{ 
    \begin{itemize} 
        \item Hohe Packungsdichte
        \item Starke Magnetfelder
        \item Kritische Messumgebung
    \end{itemize} }
    \note{ 
    \textbf{Wichtig für diese Folie}
    \begin{itemize} 
        \item Die Anlage dient der Energieverteilung bis \SI{1000}{\volt}
        \item Das Hauptsammelschienensystem verbindet die Felder
        \item Kompakte Bauweise führt zu geringen Abständen der Phasen
        \item Hohe Ströme erzeugen starke magnetische Felder
        \item Diese Felder koppeln in benachbarte Wandler ein
    \end{itemize} } 
\end{frame}

% --- Folie 2: Grundlagen Messstromwandler ---
\begin{frame}{Grundlagen Messstromwandler}
    \begin{itemize}
        \item Transformation hoher Primärströme auf normierte Signale von \SI{1,00}{\ampere} oder \SI{5,00}{\ampere}
        \item Funktion als Bindeglied zwischen Hochstrombereich und Messtechnik
        \item Sicherstellung der galvanischen Trennung der Kreise
        \item Schutz der nachgelagerten Geräte vor hohem Potenzial
    \end{itemize}

    \slidetime{02:30}{03:30}
    \keynote{
        \begin{itemize}
            \item Transformatorisches Prinzip
            \item Sicherheit durch Trennung
            \item Normierung der Signale
        \end{itemize}
    }
    \note{
        
        \begin{itemize}
            \item Der Wandler transformiert hohe Wechselströme aus dem Primärnetz in kleine und messbare Ströme
            \item Er dient als Bindeglied zwischen der Primärenergie und den Schutzeinrichtungen
            \item Das physikalische Prinzip garantiert eine galvanische Trennung zwischen Primärkreis und Sekundärkreis
            \item Dies ermöglicht den gefahrlosen Anschluss von Standardmessgeräten
            \item Die Messtechnik wird so nicht dem hohen Potenzial des Primärleiters ausgesetzt
        \end{itemize}
    }
\end{frame}

% --- Folie 3: Konstruktiver Aufbau ---
\begin{frame}{Konstruktiver Aufbau}
    \begin{figure}[H]
        \centering
        \includegraphics[width=0.8\textwidth]{03_Ressourcen/zeichnungen/aufbau_wandler.drawio.pdf}
        \caption{Schematischer Aufbau eines Aufsteckstromwandlers}
        \label{pic:aufbau_wandler}
    \end{figure}

    \slidetime{03:30}{04:30}
    \keynote{
        \begin{itemize}
            \item Primärleiter als Einwindung
            \item Ringkern bündelt Fluss
            \item Sekundärwicklung
        \end{itemize}
    }
    \note{
        
        \begin{itemize}
            \item Die Kupferschiene dient als Primärleiter und entspricht einer Windungszahl von Eins
            \item Der Eisenkern bündelt den magnetischen Fluss und besteht oft aus Siliziumeisen
            \item Die Sekundärwicklung ist direkt auf den Ringkern aufgebracht
            \item Das Gehäuse gewährleistet die elektrische Isolation und den mechanischen Schutz
            \item Die Fensteröffnung definiert den maximalen Querschnitt der Stromschiene
        \end{itemize}
    }
\end{frame}

% --- Folie 4: Physikalische Grundlagen Magnetfelder ---
\begin{frame}{Physikalische Grundlagen Magnetfelder}
    \textbf{Entstehung und Ausbreitung}
    \begin{itemize}
        \item Ausbildung konzentrischer Feldlinien um stromdurchflossene Leiter
        \item Abhängigkeit der Feldstärke vom Abstand zum Leiter
        \item Verlauf des magnetischen Flusses entlang des geringsten Widerstands
    \end{itemize}
    
    \vspace{0.5cm}
    
    \textbf{Funktion des Eisenkerns}
    \begin{itemize}
        \item Bündelung der magnetischen Feldlinien im Kernmaterial
        \item Nutzung der hohen Permeabilität ferromagnetischer Stoffe
        \item Minimierung des magnetischen Widerstands im Kreis
    \end{itemize}

    \slidetime{04:30}{05:30}
    \keynote{
        \begin{itemize}
            \item Konzentrische Kreise
            \item Weg des geringsten Widerstands
            \item Eisen leitet besser als Luft
        \end{itemize}
    }
    \note{
        
        \begin{itemize}
            \item Jeder stromdurchflossene Leiter baut ein konzentrisches Magnetfeld auf
            \item Die Feldstärke nimmt mit dem Abstand zum Leiter ab
            \item Magnetischer Fluss verhält sich analog zum elektrischen Strom
            \item Der Eisenkern hat eine hohe magnetische Leitfähigkeit
            \item Der Fluss konzentriert sich im Kern statt in der Umgebungsluft
        \end{itemize}
    }
\end{frame}

% --- Folie 5: Einfluss magnetischer Fremdfelder ---
\begin{frame}{Einfluss magnetischer Fremdfelder}
    \textbf{Ursache der Störung}
    \begin{itemize}
        \item Räumliche Nähe benachbarter stromführender Leiter
        \item Überlagerung der Magnetfelder im Raum
        \item Einkopplung externer Streuflüsse in den Wandlerkern
    \end{itemize}
    
    \vspace{0.5cm}
    
    \textbf{Auswirkung auf die Messung}
    \begin{itemize}
        \item Addition von Nutzfluss und Störfluss im Material
        \item Verursachung lokaler Sättigungseffekte im Eisenweg
        \item Asymmetrische Verzerrung des Sekundärsignals
    \end{itemize}

    \slidetime{05:30}{06:30}
    \keynote{
        \begin{itemize}
            \item Superposition der Felder
            \item Einkopplung trotz Schirmung
            \item Lokale Sättigung
        \end{itemize}
    }
    \note{
        
        \begin{itemize}
            \item In der Schaltanlage liegen die Phasen sehr eng beieinander
            \item Die Felder der Nachbarleiter durchdringen den Kern des Messwandlers
            \item Diese Vektoraddition führt zu einer Erhöhung der Flussdichte
            \item Der Kern gerät partiell in die Sättigung
            \item Das Ergebnis ist eine messbare Abweichung des Sekundärstroms
        \end{itemize}
    }
\end{frame}


% --- Folie: Bedeutung des Ferromagnetismus ---
\begin{frame}{Bedeutung des Ferromagnetismus}
    \textbf{Funktion des Eisenkerns}
    \begin{itemize}
        \item Nutzung materialspezifischer Eigenschaften zur Flussbündelung
        \item Hohe relative Permeabilität verringert magnetischen Widerstand
        \item Effiziente Übertragung der Primärgröße auf die Sekundärseite
    \end{itemize}
    
    

    \vspace{0.5cm}
    
    \textbf{Problem der Nichtlinearität}
    \begin{itemize}
        \item Begrenzte Aufnahmekapazität der magnetischen Domänen
        \item Ausrichtung der Weißschen Bezirke in Feldrichtung
        \item Eintritt der Sättigung bei hohen Feldstärken
        \item Verlust der Linearität zwischen Primärstrom und Messsignal
    \end{itemize}

    \slidetime{06:30}{07:30}
    \keynote{
        \begin{itemize}
            \item Flussbündelung notwendig
            \item Domänen richten sich aus
            \item Sättigung begrenzt Messbereich
        \end{itemize}
    }
    \note{
        
        \begin{itemize}
            \item Der Kern besteht aus ferromagnetischem Material wie Siliziumeisen zur Flussführung
            \item Im unmagnetisierten Zustand sind die magnetischen Domänen statistisch verteilt
            \item Ein äußeres Feld erzwingt die Ausrichtung dieser Weißschen Bezirke
            \item Sind alle Bezirke ausgerichtet, tritt die magnetische Sättigung ein
            \item In der Sättigung sinkt die Permeabilität und der Übertragungsfehler steigt an
        \end{itemize}
    }
\end{frame}

% --- Folie: Quantifizierung Fremdfeld ---
\begin{frame}{Quantifizierung des Fremdfeldeinflusses}
    \textbf{Berechnungsansatz nach MBS AG}
    \begin{itemize}
        \item Analytische Bestimmung der induzierten Fremdflussdichte
        \item Berücksichtigung der geometrischen Anordnung
        \item Abhängigkeit von Stromstärke und Leiterabstand
    \end{itemize}
    
    

    \vspace{0.2cm}

    % Die Formel aus deinem Text
    \begin{equation*}
        B_{\text{Fremd}} \approx \num{1,00} \cdot 10^{-6} \cdot I_{p} \cdot \frac{R + \num{0,50} \cdot W}{A} \cdot \log_{10}\left(\frac{D+R}{D-R}\right)
    \end{equation*}

    \vspace{0.3cm}
    
    \textbf{Einflussgrößen}
    \begin{itemize}
        \item Linearer Anstieg der Störgröße durch Primärstrom $I_p$
        \item Exponentieller Einfluss durch Phasenabstand $D$
        \item Kompensation der Flussaufnahme durch Kernquerschnitt $A$
    \end{itemize}

    \slidetime{07:30}{08:30}
    \keynote{
        \begin{itemize}
            \item Strom treibt Fehler
            \item Abstand schützt
            \item Geometrie ist relevant
        \end{itemize}
    }
    \note{
        \begin{itemize}
            \item Die Formel dient zur Abschätzung der zusätzlichen Flussdichte durch Fremdfelder
            \item Der Term zeigt eine direkte Proportionalität zum störenden Primärstrom
            \item Der Abstand D befindet sich im logarithmischen Term und hat großen Einfluss
            \item Ein größerer Eisenquerschnitt A verringert die resultierende Flussdichte
            \item Dies belegt theoretisch die Notwendigkeit größerer Abstände in der Anlage
        \end{itemize}
    }
\end{frame}