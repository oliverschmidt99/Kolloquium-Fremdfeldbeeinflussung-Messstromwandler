\section{Grundlagen der Arbeit}

% --- Folie 1: Funktionsprinzip und Aufbau (Standard) ---
\begin{frame}{Funktionsprinzip und Aufbau}
    \begin{columns}[c]
        % Linke Spalte: Text
        \begin{column}{0.55\textwidth}
            \textbf{Aufgaben des Messstromwandlers}
            \begin{itemize}
                \item Transformation hoher Primärströme auf normierte Signale (\SI{1}{A} / \SI{5}{A})
                \item Galvanische Trennung zum Schutz der Messgeräte
                \item Bündelung des magnetischen Flusses durch Eisenkern
            \end{itemize}
        \end{column}

        % Rechte Spalte: Bild Standard
        \begin{column}{0.40\textwidth}
            \centering
            \includegraphics[width=1.2\textwidth]{03_Ressourcen/zeichnungen/aufbau_wandler.drawio.pdf}
            \par\vspace{0.2cm}
            {\tiny Prinzipieller Aufbau eines Aufsteckstromwandlers}
        \end{column}
    \end{columns}

    \slidetime{05:30}{06:15}
    \keynote{
        \begin{itemize}
            \item Transformator-Prinzip
            \item Schutzfunktion
            \item Normierte Signale
        \end{itemize}
    }
    \note{
        \textbf{Zum Aufbau:}
        Wir sehen hier rechts den schematischen Aufbau eines Standard-Wandlers. Die Kupferschiene ($P_1/P_2$) dient als Primärwicklung mit einer einzigen Windung.

        \textbf{Zur Funktion:}
        Der Eisenkern (grau) bündelt den magnetischen Fluss um den Leiter und induziert in der Sekundärwicklung ($S_1/S_2$) einen proportionalen, aber viel kleineren Strom.

        \textbf{Das Ziel:}
        Das ermöglicht uns, Ströme von mehreren tausend Ampere sicher vom Hochpotenzial zu trennen und für die Messgeräte auf 1 oder 5 Ampere zu normieren.
    }
\end{frame}

% --- Folie 2: Lösungsansatz 1 - Kompensation ---
\begin{frame}{Lösungsansatz: Kompensierte Wandler}
    \begin{columns}[c]
        % Linke Spalte: Text
        \begin{column}{0.55\textwidth}
            \textbf{Funktionsprinzip}
            \begin{itemize}
                \item Einsatz zusätzlicher Wicklungen auf dem Eisenkern
                \item Erzeugung eines magnetischen Gegenfeldes
                \item Aktive oder passive Korrektur des Übersetzungsfehlers
            \end{itemize}

            \vspace{0.4cm}

            \textbf{Vor- und Nachteile}
            \begin{itemize}
                \item Deutliche Reduktion der Sättigungseffekte
                \item Höherer konstruktiver Aufwand und Kostenfaktor
                \item Größerer Platzbedarf im Schaltschrank
            \end{itemize}
        \end{column}

        % Rechte Spalte: Bild Kompensation
        \begin{column}{0.40\textwidth}
            \centering
            \includegraphics[width=1\textwidth]{03_Ressourcen/zeichnungen/aufbau_wandler_kompensiert.drawio.pdf}
            \par\vspace{0.2cm}
            {\tiny Prinzip der Kompensationswicklung}
        \end{column}
    \end{columns}

    \slidetime{06:15}{07:00}
    \keynote{
        \begin{itemize}
            \item Gegenfeld erzeugen
            \item Fehler minimieren
            \item Aber: Teuer \& Groß
        \end{itemize}
    }
    \note{
        \textbf{Das Prinzip der Kompensation:}
        Eine technische Weiterentwicklung ist der kompensierte Wandler. Wie in der Grafik zu sehen, nutzen wir hier zusätzliche Wicklungen (W3/W4), um ein magnetisches Gegenfeld zu erzeugen.

        \textbf{Die Wirkung:}
        Dadurch wird der Arbeitspunkt des Kerns stabilisiert und Sättigungseffekte werden aktiv ausgeglichen. Der Wandler misst linearer.

        \textbf{Der Nachteil:}
        Allerdings ist dieser Aufbau deutlich komplexer. Das treibt die Kosten und – was fast noch kritischer ist – den Platzbedarf im Schaltschrank in die Höhe.
    }
\end{frame}

% --- Folie 3: Lösungsansatz 2 - FFP-Technologie ---
\begin{frame}{Lösungsansatz: Fremdfeld-Protektion (FFP)}

    \begin{columns}[c]
        % Linke Spalte: Text
        \begin{column}{0.55\textwidth}
            \textbf{Konstruktive Optimierung}
            \begin{itemize}
                \item Gezielte Schirmung des Messkerns
                \item Umleitung der magnetischen Störfeldlinien
                \item Prinzip des magnetischen Nebenschlusses
            \end{itemize}

            \vspace{0.5cm}

            \textbf{Zielsetzung}
            \begin{itemize}
                \item Einhaltung der Genauigkeitsklasse 1 trotz enger Phasenabstände
                \item Schutz vor partieller Sättigung durch Nachbarleiter
            \end{itemize}
        \end{column}

        % Rechte Spalte: Bild (Redur Patent)
        \begin{column}{0.40\textwidth}
            \centering
            \begin{beamercolorbox}[sep=0.5em,center,rounded=true,shadow=false]{white}
                \centering
                % Falls .png nicht existiert, Endung ggf. auf .pdf anpassen
                \includegraphics[width=0.95\textwidth]{03_Ressourcen/Bilder/wandler-ffp-redur-patent.png}
                \par\vspace{0.2cm}
                {\tiny \color{gray} Quelle: Patent DE102021106843A1 (Redur)}
            \end{beamercolorbox}
        \end{column}
    \end{columns}

    \slidetime{07:00}{07:45}
    \keynote{
        \begin{itemize}
            \item Schirmung als "Umleitung"
            \item Schutz des Kerns
            \item Klasse 1 sichern
        \end{itemize}
    }
    \note{
        \textbf{Die FFP-Technologie:}
        Eine Alternative ist die sogenannte Fremdfeld-Protektion (FFP). Das Bild rechts zeigt eine Patentschrift der Firma Redur dazu.

        \textbf{Das Prinzip (Schirmung):}
        Hier wird der eigentliche Messkern durch spezielle Leitbleche (im Bild die äußeren Schalen) abgeschirmt. Diese wirken wie ein magnetischer Nebenschluss und leiten die Feldlinien der Nachbarleiter am Kern vorbei.

        \textbf{Das Ziel:}
        Der innere Kern bleibt "sauber", gerät nicht in die Sättigung und soll so die Genauigkeitsklasse 1 auch bei engsten Abständen garantieren.
    }
\end{frame}

% --- Folie 4: Das Problem (Fremdfelder) ---
\begin{frame}{Physikalisches Problem: Fremdfeldeinfluss}

    \textbf{Ursache: Räumliche Nähe}
    \begin{itemize}
        \item Benachbarte Leiter erzeugen starke eigene Magnetfelder
        \item Diese Felder koppeln als \textbf{Störfluss} in den Wandlerkern ein
    \end{itemize}

    \vspace{0.4cm}

    \textbf{Wirkung: Partielle Sättigung}
    \begin{itemize}
        \item Der Störfluss addiert sich vektoriell zum Nutzfluss
        \item \textbf{Folge:} Der Kern gerät lokal in die magnetische Sättigung
        \item Die Permeabilität $\mu_r$ sinkt $\rightarrow$ Das Übertragungsverhältnis stimmt nicht mehr
    \end{itemize}

    \vspace{0.3cm}
    \begin{beamercolorbox}[sep=0.5em,center,rounded=true,shadow=false]{alerted text}
        \textbf{Resultat:} Der Sekundärstrom sinkt, die Messung zeigt zu wenig an.
    \end{beamercolorbox}

    \slidetime{07:45}{08:30}
    \keynote{
        \begin{itemize}
            \item Einkopplung von außen
            \item Kern ist „voll“ (Sättigung)
            \item Messwert sinkt ab
        \end{itemize}
    }
    \note{
        \textbf{Der physikalische Hintergrund:}
        Warum brauchen wir diese Technologien? Wenn wir Standard-Wandler in der engen Anlage verbauen, durchdringen die Magnetfelder der Nachbarleiter den Kern.

        \textbf{Der Mechanismus:}
        Der Eisenkern addiert den "Nutzfluss" und den "Störfluss". Ist die Summe zu groß, gerät das Eisen in die Sättigung.

        \textbf{Die Folge:}
        In der Sättigung verliert der Kern seine Leitfähigkeit für das Magnetfeld ($\mu_r$ sinkt). Er kann den Strom nicht mehr vollständig übertragen. Das Resultat ist genau der Einbruch, den wir bei Phase L2 gemessen haben.
    }
\end{frame}

% --- Folie 5: Normative Anforderungen (Klassen) ---
\begin{frame}{Normative Anforderungen (Klassen)}
    \textbf{Definition der Genauigkeit nach DIN EN 61869-2}
    \begin{itemize}
        \item Einteilung der Wandler in Genauigkeitsklassen
        \item Klasse definiert die maximal zulässige Messabweichung
    \end{itemize}

    \vspace{0.8cm}

    % Tabelle der Klassen (OHNE Anwendungsbereich)
    \centering
    \renewcommand{\arraystretch}{1.3}
    \begin{tabular}{l c}
        \toprule
        \textbf{Klasse} & \textbf{Fehler bei Nennstrom} \\
        \midrule
        \textbf{0,2 S}  & $\pm\, \SI{0,2}{\%}$          \\
        \rowcolor{cHSblue!10}
        \textbf{0,5}    & $\pm\, \SI{0,5}{\%}$          \\
        \rowcolor{cHSblue!10}
        \textbf{1}      & $\pm\, \SI{1,0}{\%}$          \\
        \textbf{3}      & $\pm\, \SI{3,0}{\%}$          \\
        \bottomrule
    \end{tabular}

    \vspace{0.8cm}
    \small \textit{Hinweis: Für diese Arbeit ist die Einhaltung der \textbf{Klasse 1} das Mindestziel.}

    \slidetime{08:30}{09:15}
    \keynote{
        \begin{itemize}
            \item Normierung nach IEC
            \item Ziel: Klasse 1
            \item Fehlergrenzen definiert
        \end{itemize}
    }
    \note{
        \textbf{Bewertungsgrundlage:}
        Um in den folgenden Ergebnissen zu entscheiden, ob eine Lösung tauglich ist, brauchen wir einen Maßstab. Die Norm definiert hierfür feste Genauigkeitsklassen.

        \textbf{Die Tabelle:}
        Wie wir sehen, darf ein Wandler der Klasse 1 bei Nennstrom maximal 1 Prozent abweichen.

        \textbf{Mein Ziel:}
        Das ist der Benchmark für meine Arbeit: Schaffen es die Wandler unter Fremdfeldeinfluss, diese \textbf{Klasse 1} zu halten? Alles, was schlechter ist, ist für eine Verrechnungsmessung inakzeptabel.
    }
\end{frame}