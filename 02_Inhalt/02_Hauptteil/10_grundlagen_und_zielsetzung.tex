\section{Grundlagen der Arbeit}

\begin{frame}{Niederspannungsschaltanlage}
    \begin{itemize} 
        \item \textbf{Niederspannungsschaltanlage:} Energieverteilung auf viele Abgänge
        \item \textbf{Funktionen:} Schalten/Schützen \,+\, Messen (Überwachung, Verrechnung) 
        \item \textbf{Einbausituation:} 
        \begin{itemize} 
            \item \SIrange{900}{6000}{\ampere} in Sammelschienen \item kompakt \(\rightarrow\) Messstromwandler nahe an benachbarten Phasen 
            \item \(\Rightarrow\) Fremdfelder \(\rightarrow\) Messabweichung 
        \end{itemize} 
    \end{itemize} 
    \slidetime{02:00}{02:30} 
    \keynote{ 
    \begin{itemize} 
        \item \textbf{A} 
    \end{itemize} }
    \note{ 
    \textbf{Wichtig für diese Folie:} 
    \begin{itemize} 
        \item 1 
    \end{itemize} } 
\end{frame}




\begin{frame}{Messstromwandler}
\begin{itemize}
    \item Funktioniert nach dem Transformatorprinzip
    \item Transformiert Primärströme auf messbare Sekundärströme
    \item Verbindet Hochstrombereich mit Messeinrichtungen
    \item Trennt Primärkreis und Sekundärkreis galvanisch
    \item Erlaubt Anschluss standardisierter Messgeräte mit \SI{1,00}{\ampere} oder \SI{5,00}{\ampere}
\end{itemize}

\slidetime{00:00}{00:00}
\keynote{
\begin{itemize}
        \item Transformatorprinzip nutzen
        \item Galvanische Trennung sicherstellen
        \item Normierte Signale bereitstellen
    \end{itemize}
}
\note{
    \textbf{Wichtig für diese Folie}
    \begin{itemize}
        \item Ein Stromwandler transformiert Wechselströme aus dem Primärnetz in messbare Ströme auf der Sekundärseite
        \item Er fungiert als Bindeglied zwischen dem Hochstrombereich und den Schutzeinrichtungen
        \item Das Funktionsprinzip beruht auf der galvanischen Trennung zwischen Primärkreis und Sekundärkreis
        \item Dies ermöglicht den Anschluss standardisierter Messgeräte für Nennströme von 1 A oder 5 A
        \item Die Geräte werden so nicht dem Potenzial des Primärleiters ausgesetzt
    \end{itemize}
}



\end{frame}




\begin{frame}{Aufbau Messstromwandler}

\begin{figure}[H]
    \centering
    \includegraphics[width=1.0\textwidth]{03_Ressourcen/zeichnungen/aufbau_wandler.drawio.pdf}
    \caption{Schematischer Aufbau eines Aufsteckstromwandlers}
    \label{pic:aufbau_wandler}
\end{figure}


\slidetime{00:00}{00:00}
\keynote{
\begin{itemize}
        \item \textbf{empty}
    \end{itemize}
}
\note{
    \textbf{Wichtig für diese Folie:}
    \begin{itemize}
        \item empty
    \end{itemize}
}
\end{frame}



\begin{frame}{template}
\begin{itemize}
    \item empty
\end{itemize}

\slidetime{00:00}{00:00}
\keynote{
\begin{itemize}
        \item \textbf{empty}
    \end{itemize}
}
\note{
    \textbf{Wichtig für diese Folie:}
    \begin{itemize}
        \item empty
    \end{itemize}
}
\end{frame}