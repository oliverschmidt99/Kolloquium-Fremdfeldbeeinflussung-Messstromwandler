\section{Messergebnisse 3000 A}

% --- 3000 A: Fehlerkurven ---
\begin{frame}{Fehlerkurven bei \SI{3000}{A}}
    \centering
    % Pfad: dia_3000A_kosten/dia_3000A_kosten-Zusammenfassung_MultiCurrent.png
    \includegraphics[width=\textwidth, height=0.85\textheight, keepaspectratio]{03_Ressourcen/diagramme/dia_3000A_kosten/dia_3000A_kosten-Zusammenfassung_MultiCurrent.png}

    % Zeitplanung
    \slidetime{10:00}{11:00}
    \keynote{
        \begin{itemize}
            \item Bunt (Parallel): Starke Spreizung
            \item Grau (Dreieck): Viel engeres Fehlerband
            \item Sättigung beginnt
        \end{itemize}
    }
    \note{
        \textbf{Szenario 3000 A - Fehlerverlauf:}
        Hier sehen wir die Fehlerkurven bei 3000 Ampere.

        \textbf{Beobachtung:}
        Die bunten Linien (Parallelanordnung) zeigen eine deutliche Spreizung zwischen den Phasen. Man sieht, wie der Fehler bei steigender Last (x-Achse) stark abfällt – ein Zeichen für beginnende Sättigung.

        \textbf{Vergleich:}
        Die grauen Linien (Dreiecksanordnung) liegen deutlich enger beisammen und verlaufen stabiler. Die geometrische Optimierung wirkt hier bereits sehr gut.
    }
\end{frame}

% --- 3000 A: Ranking ---
\begin{frame}{Ökonomisches Ranking bei \SI{3000}{A}}
    \centering
    % Pfad: dia_3000A_kosten/dia_3000A_kosten-Oekonomie_Ranking_Dual.png
    \includegraphics[width=\textwidth, height=0.85\textheight, keepaspectratio]{03_Ressourcen/diagramme/dia_3000A_kosten/dia_3000A_kosten-Oekonomie_Ranking_Dual.png}

    \slidetime{11:00}{11:45}
    \keynote{
        \begin{itemize}
            \item Preis vs. Fehler
            \item Parallel (Bunt): Hohe Fehler
            \item Dreieck (Orange/Gelb): Gutes P/L-Verhältnis
        \end{itemize}
    }
    \note{
        \textbf{Ranking 3000 A:}
        Hier sind die Kosten (grüner Balken) gegen die Fehlerarten (bunt) aufgetragen.

        \textbf{Ergebnis:}
        Während die Parallel-Varianten (ganz unten) zwar günstig sind, weisen sie sehr hohe Fehlerbalken auf (rot/orange).
        Die Dreiecksvarianten (oben) bieten hier den besten Kompromiss: Die Fehler sind minimal, ohne dass die Kosten so explodieren wie bei vollkompensierten Wandlern.
    }
\end{frame}

\section{Messergebnisse 4000 A}

% --- 4000 A: Fehlerkurven ---
\begin{frame}{Fehlerkurven bei \SI{4000}{A} (Kritischer Bereich)}
    \centering
    % Pfad: dia_4000A_kosten/dia_4000A_kosten-Zusammenfassung_MultiCurrent.png
    \includegraphics[width=\textwidth, height=0.85\textheight, keepaspectratio]{03_Ressourcen/diagramme/dia_4000A_kosten/dia_4000A_kosten-Zusammenfassung_MultiCurrent.png}

    \slidetime{11:45}{12:45}
    \keynote{
        \begin{itemize}
            \item 4000 A = Extremfall
            \item Parallel: Totalausfall (Absturz der Kurve)
            \item Dreieck: Hält noch stand
        \end{itemize}
    }
    \note{
        \textbf{Eskalation auf 4000 A:}
        Bei 4000 Ampere sehen wir drastische Effekte.

        \textbf{Analyse:}
        Achten Sie auf die orange Linie in der Mitte (Phase L2, Parallel). Der Fehler stürzt komplett ab auf unter -6\%. Das ist messtechnisch ein Totalausfall.
        Selbst die Dreiecksanordnung (gestrichelt, grau/blau) kommt hier an ihre Grenzen, bleibt aber noch deutlich stabiler als die Standardanordnung.
    }
\end{frame}

% --- 4000 A: Ranking ---
\begin{frame}{Ökonomisches Ranking bei \SI{4000}{A}}
    \centering
    % Pfad: dia_4000A_kosten/dia_4000A_kosten-Oekonomie_Ranking_Dual.png
    \includegraphics[width=\textwidth, height=0.85\textheight, keepaspectratio]{03_Ressourcen/diagramme/dia_4000A_kosten/dia_4000A_kosten-Oekonomie_Ranking_Dual.png}

    \slidetime{12:45}{13:30}
    \keynote{
        \begin{itemize}
            \item Fehlerbalken dominieren
            \item Standardtechnik nicht mehr zulässig
            \item FFP oder Dreieck zwingend nötig
        \end{itemize}
    }
    \note{
        \textbf{Ranking 4000 A:}
        Im ökonomischen Vergleich bei 4000 A wird deutlich:

        \textbf{Rot (Überlastfehler):}
        Die roten Balken bei den unteren Varimport pandas as pd
import numpy as np
import matplotlib.pyplot as plt

# --- KONFIGURATION ---
CSV_DATEI = '2026-02-11T11-12_export.csv'
SPALTEN_NAMEN = ['5% In', '20% In', '50% In', '80% In', '90% In', '100% In', '120% In']

def lade_und_plotte_vergleich():
    # 1. Daten laden
    df = pd.read_csv(CSV_DATEI)
    
    # --- PREIS BEREINIGUNG ---
    if 'Preis (€)' in df.columns:
        # Funktion zur robusten Bereinigung deutscher Zahlenformate
        def parse_german_float(s):
            if pd.isna(s): return np.nan
            s = str(s)
            # Entferne Tausender-Punkte (z.B. 1.200,00 -> 1200,00)
            if '.' in s and ',' in s: # Nur entfernen wenn beides da ist, um sicher zu gehen
                 s = s.replace('.', '')
            elif '.' in s and ',' not in s:
                 # Fall: 1.200 (könnte 1200 sein oder 1.2) - Kontextabhängig.
                 # Hier gehen wir davon aus, dass . Tausender ist, wenn kein Komma da ist
                 # Aber in deinem File war es bisher immer mit Komma.
                 pass 
            
            # Ersetze Dezimal-Komma durch Punkt (z.B. 43,670 -> 43.670)
            s = s.replace(',', '.')
            return float(s)

        df['Preis (€)'] = df['Preis (€)'].apply(parse_german_float)

    # Fehler-Spalten konvertieren
    for col in SPALTEN_NAMEN:
        if col in df.columns:
            if df[col].dtype == 'object':
                df[col] = df[col].astype(str).str.replace(',', '.', regex=False)
            df[col] = pd.to_numeric(df[col], errors='coerce')

    # 2. Max Fehler pro Zeile (Worst Case)
    df['row_max_error'] = df[SPALTEN_NAMEN].abs().max(axis=1)

    # 3. Aggregation pro Gerät (Summe L1+L2+L3)
    # Wir summieren Fehler UND Preise für das Gesamtsystem (3 Phasen)
    device_stats = df.groupby(['final_legend', 'nennstrom', 'technologie', 'geometrie']).agg({
        'row_max_error': 'sum', 
        'Preis (€)': 'sum'      
    }).reset_index()

    # 4. Normierung: Fehler pro Euro
    device_stats['error_per_euro'] = device_stats.apply(
        lambda row: row['row_max_error'] / row['Preis (€)'] if row['Preis (€)'] > 0 else np.nan, 
        axis=1
    )

    # 5. Zusammenfassung pro Konfiguration (Max/Worst Case)
    summary = device_stats.groupby(['nennstrom', 'technologie', 'geometrie']).agg({
        'row_max_error': 'max',
        'error_per_euro': 'max'
    }).reset_index()

    summary['label'] = summary['technologie'] + ' ' + summary['geometrie']

    # --- PLOTTEN ---
    farben_map = {
        'Standard Parallel': '#d62728',     # Rot
        'Standard Dreieck': '#ff9896',      # Hellrot
        'Kompensiert Parallel': '#1f77b4',  # Blau
        'Kompensiert Dreieck': '#aec7e8',   # Hellblau
        'FFP Parallel': '#2ca02c',          # Grün
        'FFP Dreieck': '#98df8a'            # Hellgrün
    }

    def erstelle_balkendiagramm(data_col, title, ylabel, filename, fmt='%.1f'):
        pivot_data = summary.pivot(index='nennstrom', columns='label', values=data_col)
        colors = [farben_map.get(c, '#333333') for c in pivot_data.columns]
        
        plt.figure(figsize=(12, 7))
        ax = pivot_data.plot(kind='bar', figsize=(14, 8), width=0.8, color=colors, edgecolor='black', linewidth=0.5)
        
        plt.title(title, fontsize=14)
        plt.ylabel(ylabel, fontsize=12)
        plt.xlabel('Nennstrom [A]', fontsize=12)
        plt.xticks(rotation=0)
        plt.grid(axis='y', linestyle='--', alpha=0.7)
        plt.legend(title='Konfiguration', bbox_to_anchor=(1.01, 1), loc='upper left')

        for container in ax.containers:
            ax.bar_label(container, fmt=fmt, padding=3, rotation=90, fontsize=9)

        plt.tight_layout()
        plt.savefig(filename, dpi=300)
        print(f"Grafik gespeichert: {filename}")
        plt.close()

    # Plot 1: Absolut
    erstelle_balkendiagramm('row_max_error', 
                            'Vergleich Gesamtfehler (Summe L1+L2+L3)', 
                            'Gesamtfehler [Summe Absolutbeträge]', 
                            'vergleich_fehler_absolut.png')

    # Plot 2: Normiert
    erstelle_balkendiagramm('error_per_euro', 
                            'Fehler auf Preis normiert (Gesamtfehler / Gesamtpreis)', 
                            'Normierter Fehler [1/€]', 
                            'vergleich_fehler_pro_euro.png', 
                            fmt='%.4f')

if __name__ == "__main__":
    lade_und_plotte_vergleich()ianten (Parallel) sind riesig. Das bedeutet, bei Überstrom messen wir fast nur noch "Mist".

        \textbf{Fazit:}
        Wer hier spart, misst falsch. Die Investition in die optimierte Anordnung (obere Balken) ist hier technisch zwingend erforderlich, um überhaupt noch verwertbare Daten zu erhalten.
    }
\end{frame}


