% ==========================================
% TEIL 1: VERSUCHSAUFBAU
% ==========================================
\section{Versuchsaufbau und Durchführung}

% --- Folie 6: Messstrecke ---
\begin{frame}{Versuchsaufbau: Hochstrom-Prüfstand}
    \centering
    \textbf{Schematischer Aufbau der automatisierten Messstrecke}

    \vspace{0.2cm}

    \includegraphics[height=0.75\textheight, width=0.95\textwidth, keepaspectratio]{03_Ressourcen/zeichnungen/aufbau_hochstrom_pruefstand_new.drawio.pdf}

    \slidetime{07:05}{07:55}
    \keynote{
        \begin{itemize}
            \item Referenz -> Trafo -> 4kA
            \item Synchrone Messung
            \item WinCC Erfassung
        \end{itemize}
    }
    \note{
        \textbf{Ziel des Aufbaus}
        Damit die Unterschiede zwischen Wandlern, FFP und Geometrie wirklich vergleichbar sind, muss die Messung reproduzierbar und synchron sein: \emph{gleicher Strom, gleiche Temperaturbedingungen, gleiche Auswertung}.

        \textbf{Messkette / Signalfluss}
        Wir erzeugen den hohen Prüfstrom über den Hochstromtrafo und regeln ihn stufenweise bis \SI{4000}{A}. Parallel dazu messen wir synchron:
        \begin{itemize}
            \item \textbf{Referenz:} PAC 4220 (hohe Genauigkeit) als ``Wahrheit''.
            \item \textbf{Prüfling:} PAC 3220 + der jeweils getestete Stromwandler.
        \end{itemize}
        Aus der Differenz zwischen Referenz und Prüfling erhalten wir direkt den Übersetzungsfehler und können Normgrenzen (Klasse 1) bewerten.

        \textbf{Automatisierung und Datenqualität}
        Die komplette Sequenz läuft SPS-geführt (Profinet), inklusive Haltezeiten zum Einschwingen. Dadurch eliminieren wir typische Fehlerquellen wie manuelles Nachregeln, Ablesefehler oder variable Einschwingzeiten. Die Messwerte werden in WinCC mit Zeitstempel archiviert, was eine saubere Nachauswertung ermöglicht.

        \textbf{Wichtiger Punkt für die späteren Ergebnisse}
        So stellen wir sicher, dass ein Kurveneinbruch (z.\,B. bei L2/Parallel) \emph{wirklich} aus der Kernphysik/Sättigung kommt und nicht aus einem Messartefakt des Prüfstands.
    }
\end{frame}

% --- Folie 7: Lastprofil ---
\begin{frame}{Lastprofil und Prüfablauf}
    \centering
    \textbf{Visualisierung der automatisierten Messsequenz}

    \includegraphics[width=\textwidth, height=0.65\textheight, keepaspectratio]{03_Ressourcen/Bilder/verlauf_gesamt.png}


    %\begin{columns}[t, onlytextwidth]
    %    \begin{column}{0.48\textwidth}
    %        \textbf{Ablauf}
    %        \begin{itemize}
    %            \item Stufenweise Erhöhung (\SI{5}{\%} bis \SI{120}{\%})
    %            \item Vollautomatisierte Ansteuerung
    %        \end{itemize}
    %    \end{column}
    %    \begin{column}{0.48\textwidth}
    %        \textbf{Vorteil}
    %        \begin{itemize}
    %            \item Identische thermische Belastung
    %            \item 100\,\% Reproduzierbarkeit
    %        \end{itemize}
    %    \end{column}
    %\end{columns}

    \slidetime{07:55}{08:40}
    \keynote{
        \begin{itemize}
            \item Treppenstufen = Lastpunkte
            \item Reproduzierbarkeit
            \item Menschlicher Fehler ausgeschlossen
        \end{itemize}
    }
    \note{
        \textbf{Das Lastprofil}
        Hier sehen Sie den \frqq Herzschlag\flqq\ unserer Messung – die direkte Aufzeichnung des Primärstroms aus dem Leitsystem.

        \textbf{Verfahren}
        Wir fahren vollautomatisiert feste Lastpunkte an: Von 5\,\% Teillast bis hinauf zu 120\,\% Überlast. Die Plateaus zeigen die Haltezeiten, in denen sich die Messwerte einschwingen.

        \textbf{Relevanz}
        Das ist entscheidend für die wissenschaftliche Qualität: Jeder Wandler wird exakt demselben Stressprofil ausgesetzt. Menschliche Ablesefehler oder Schwankungen am Regeltrafo sind durch die SPS-Steuerung ausgeschlossen.
    }
\end{frame}


% ==========================================
% TEIL 2: DETAILLIERTE MESSERGEBNISSE
% ==========================================
\section{Exemplarische Messergebnisse}

\subsection{Messung bei 2000 A}

% --- 2000 A: Genauigkeitsmessung ---
\begin{frame}{Genauigkeitsmessung bei \SI{2000}{A} (Linearer Bereich)}
    \centering
    \includegraphics[width=\textwidth, height=0.85\textheight, keepaspectratio]{03_Ressourcen/diagramme/verlauf_2000A_presentation.png}

    \slidetime{08:40}{10:05}
    \keynote{
        \begin{itemize}
            \item Referenzmessung im Nennbereich
            \item Alle Wandler arbeiten präzise
            \item Kaum Unterschiede zwischen Parallel und Dreieck
        \end{itemize}
    }
    \note{
    \textbf{Baseline bei \SI{2000}{A}}
    Diese Messung dient als Referenz im weitgehend linearen Bereich: Die Kurven liegen eng um 0\,\%,
    Fremdfeld- und Sättigungseffekte sind hier noch klein.

    \textbf{So liest man die Kurven}
    Durchgezogen = Parallel, gestrichelt = Dreieck (gleicher Wandler, andere Geometrie).
    Wichtig ist besonders L2, weil dort später der stärkste Fremdfeldeinfluss zu erwarten ist.

    \textbf{Überleitung}
    Wenn bei \SI{4000}{A} starke Abweichungen auftreten, ist das damit klar als physikalischer Nichtlinearitäts-/Sättigungseffekt
    einzuordnen – nicht als Messstreuung.
    }
\end{frame}

\begin{frame}{Aufbau FFP-Wandler (kompensiert)}
    \centering
    \includegraphics[width=\textwidth, keepaspectratio]
            {03_Ressourcen/zeichnungen/aufbau_wandler_FFP.drawio.pdf}

    \slidetime{05:40}{06:10}

    \note{
        \textbf{Was sieht man in der Skizze?}
        Der Primärleiter erzeugt das Hauptmagnetfeld im Ringkern.
        Die zusätzlichen Wicklungen detektieren Streufelder und führen über die Kompensationswicklung
        ein gezieltes Gegenfeld ein.

        \textbf{Physikalische Idee}
        Dadurch bleibt der resultierende Fluss im Eisenkern näher am linearen Bereich.
        Die lokale Sättigung – insbesondere bei asymmetrischen Feldverteilungen – wird reduziert.

        \textbf{Einordnung}
        Das ist eine aktive, konstruktive Lösung.
        Im späteren Vergleich prüfen wir, ob eine rein geometrische Maßnahme
        (Dreiecksanordnung der Leiter) einen ähnlichen Effekt erzielen kann –
        jedoch ohne Zusatzwicklung und ohne Mehrkosten.
    }

\end{frame}





\subsection{Messung bei 4000 A}

% --- 4000 A: Genauigkeitsmessung ---
\begin{frame}{Genauigkeitsmessung bei \SI{4000}{A} (Kritischer Bereich)}
    \centering
    \includegraphics[width=\textwidth, height=0.85\textheight, keepaspectratio]{03_Ressourcen/diagramme/verlauf_4000A_presentation.png}

    \slidetime{10:05}{11:45}
    \keynote{
        \begin{itemize}
            \item 4000 A = Extremfall
            \item Parallel zeigt Totalausfall (Absturz der Kurve)
            \item Dreieck hält die Genauigkeit deutlich länger
        \end{itemize}
    }
    \note{
        \textbf{Eskalation auf 4000 A}
        Im direkten Vergleich sehen wir nun die drastischen Effekte bei 4000 Ampere.

        \textbf{Analyse}
        Achten Sie auf die hellblaue Linie in der Mitte (Phase L2, Parallel). Der Fehler stürzt komplett ab auf unter -6\,\%. Das ist messtechnisch ein Totalausfall durch Sättigung.

        \textbf{Lösung}
        Die Dreiecksanordnung (dunkelblau gestrichelt) kommt zwar auch an ihre Grenzen, bleibt aber noch stabil genug, um nutzbare Werte zu liefern. Der Unterschied zur vorherigen Folie ist massiv.
    }
\end{frame}


% ==========================================
% TEIL 3: ANALYSE UND WIRTSCHAFTLICHKEIT
% ==========================================
\section{Vergleichende Analyse}

\subsection{Methodik und Kosten}

% --- Metriken ---
\begin{frame}{Berechnungsgrundlagen der Analyse}
    \small
    Um die Diagramme korrekt zu interpretieren, hier die Methodik:

    \vspace{0.2cm}

    \textbf{1. Mittlerer Gesamtfehler (Basiswert)}
    \begin{equation*}
        E_{\text{total}} = \frac{1}{3} \sum_{\text{Phasen}} \left( \frac{1}{n} \sum_{\text{Last}} |F_{\text{Messwert}}| \right)
    \end{equation*}
    \footnotesize{Durchschnitt der \textbf{Beträge} über alle Lastpunkte (5\%--120\%) und Phasen.}

    \vspace{0.3cm}
    \hrule
    \vspace{0.3cm}

    \begin{columns}[t]
        \begin{column}{0.48\textwidth}
            \centering
            \textbf{2. Geom. Verbesserung (\%)}
            \begin{equation*}
                \eta_{\text{geo}} = \left( 1 - \frac{E_{\text{Dreieck}}}{E_{\text{Parallel}}} \right) \cdot 100
            \end{equation*}
            \footnotesize{Anteil des eliminierten Fehlers durch Geometrie-Wechsel.}
        \end{column}
        \begin{column}{0.48\textwidth}
            \centering
            \textbf{3. Wirtschaftlichkeit (\%/€)}
            \begin{equation*}
                \eta_{\text{eco}} = \frac{\eta_{\text{geo}}}{\text{Preis (€)}}
            \end{equation*}
            \footnotesize{Technischer Gewinn normiert auf Investitionskosten.}
        \end{column}
    \end{columns}

    \slidetime{11:45}{12:40}
    \note{
        \textbf{Warum diese Kennzahlen?}
        Die Diagramme enthalten viele Lastpunkte und drei Phasen. Um daraus eine \emph{vergleichbare} Aussage pro Wandler und Geometrie abzuleiten, verdichten wir die Kurven auf wenige robuste Kennzahlen.

        \textbf{1) Mittlerer Gesamtfehler \(E_{\text{total}}\)}
        Wir mitteln über:
        \begin{itemize}
            \item alle \textbf{Lastpunkte} (5\,\% bis 120\,\%),
            \item alle \textbf{Phasen} (L1, L2, L3),
        \end{itemize}
        und verwenden den \textbf{Betrag} \(|F_{\text{Messwert}}|\), damit sich positive/negative Fehler nicht gegenseitig ``schönrechnen''.
        Das Ergebnis ist ein einziger Basiswert pro Messaufbau: \emph{Wie groß ist der typische Fehler im Betrieb?}

        \textbf{2) Geometrische Verbesserung \(\eta_{\text{geo}}\)}
        \[
        \eta_{\text{geo}}=\left(1-\frac{E_{\Delta}}{E_{\parallel}}\right)\cdot 100
        \]
        Diese Kennzahl sagt: \emph{Wie viel Prozent des ursprünglichen Fehlers (Parallel) wird durch Dreieck eliminiert?}
        Beispiel: \(E_{\parallel}=6\) und \(E_{\Delta}=1\) \(\Rightarrow\) \(\eta_{\text{geo}}=83\,\%\).

        \textbf{3) Wirtschaftlichkeit \(\eta_{\text{eco}}\) in \(\%\!/\text{€}\)}
        \[
        \eta_{\text{eco}}=\frac{\eta_{\text{geo}}}{\text{Preis}}
        \]
        Damit vergleichen wir nicht nur ``was ist technisch gut?'', sondern \emph{wie viel Verbesserung bekomme ich pro investiertem Euro?}
        Genau diese Sicht ist später entscheidend, um Standardwandler + Dreieck gegen teure Spezialwandler einzuordnen.
    }
\end{frame}

% --- Kostenübersicht ---
\begin{frame}{Wirtschaftliche Einordnung der Wandler}
    \centering
    \includegraphics[width=\textwidth, height=0.85\textheight, keepaspectratio]{03_Ressourcen/diagramme/diag_kosten_horizontal.png}

    \slidetime{12:40}{13:30}
    \keynote{
        \begin{itemize}
            \item Standardwandler sind sehr günstig
            \item Kompensierte Wandler kosten ein Vielfaches
            \item Preisschere öffnet sich bei hohen Strömen
        \end{itemize}
    }
    \note{
        \textbf{Der Kostenfaktor}
        Bevor wir die Effizienz bewerten, müssen wir die Investitionskosten betrachten.

        \textbf{Vergleich}
        Die hellblauen und roten Balken zeigen die Standardwandler. Diese liegen preislich weit unter 100 Euro.
        Ganz unten sehen sie die dunkelblauen Balken für die kompensierten Wandler bei 3000 und 4000 Ampere. Diese kosten mit über 300 bis 400 Euro ein Vielfaches der Standardgeräte.

        \textbf{Bedeutung}
        Wenn wir mit der Dreiecksanordnung die billigen Wandler auf das Niveau der teuren Wandler heben können, ist der wirtschaftliche Hebel enorm.
    }
\end{frame}


\subsection{Gesamtvergleich und Effizienz}

% --- DIAGRAMM 0: Wahrer Fehler ---
\begin{frame}{Gesamtfehler im Vergleich: Parallel vs. Dreieck}
    \centering
    \includegraphics[width=\textwidth, height=0.85\textheight, keepaspectratio]{03_Ressourcen/diagramme/diag0_wahre_fehler_horizontal.png}

    \slidetime{13:30}{14:50}
    \keynote{
        \begin{itemize}
            \item Volle Balken sind Parallelanordnung
            \item Schraffierte Balken sind Dreiecksanordnung
            \item Massive Reduktion der Balkenlänge durch Dreieck
        \end{itemize}
    }
    \note{
        \textbf{Analyse der absoluten Fehler}
        Dieses Diagramm zeigt die summierten Fehlerwerte über alle Messpunkte.

        \textbf{Lesehilfe}
        Die vollen Balken repräsentieren die klassische Parallelanordnung. Die schraffierten Balken zeigen denselben Wandler in der optimierten Dreiecksanordnung.

        \textbf{Erkenntnis}
        Man sieht auf den ersten Blick, dass die schraffierten Balken fast überall deutlich kürzer sind. Besonders beim Celsa Standardwandler bei 2500 Ampere fällt der Fehler von über 7 auf unter 2 Einheiten. Die Geometrie wirkt hier wie ein physischer Filter gegen Störfelder.
    }
\end{frame}

% --- DIAGRAMM 1: Verbesserung Dreieck (%) ---
\begin{frame}{Wirksamkeit der geometrischen Optimierung}
    \centering
    \includegraphics[width=\textwidth, height=0.85\textheight, keepaspectratio]{03_Ressourcen/diagramme/diag1_dreieck_pct.png}

    \slidetime{14:50}{15:55}
    \keynote{
        \begin{itemize}
            \item Relative Fehlerreduktion durch Geometrie
            \item Über 90 Prozent Verbesserung bei 3000 A
            \item Standardwandler zeigen größten Effekt
        \end{itemize}
    }
    \note{
        \textbf{Geometrische Optimierung}
        Hier sehen wir die prozentuale Fehlerreduktion durch den Wechsel von Parallelanordnung auf Dreiecksanordnung.

        \textbf{Beobachtung}
        Besonders bei 3000 Ampere zeigen sich hohe Werte. Das bedeutet hier eliminiert die Dreiecksanordnung fast den gesamten Fehler der in der Parallelanordnung auftrat.

        \textbf{Schlussfolgerung}
        Der Standardwandler profitiert am stärksten von der geometrischen Anordnung und erreicht dadurch fast die Güte der teureren Systeme.
    }
\end{frame}

% --- DIAGRAMM 2: Wirtschaftlichkeit Dreieck (%/€) ---
\begin{frame}{Ökonomische Bewertung der Maßnahmen}
    \centering
    \includegraphics[width=\textwidth, height=0.85\textheight, keepaspectratio]{03_Ressourcen/diagramme/diag2_dreieck_pct_eur.png}

    \slidetime{15:55}{16:50}
    \keynote{
        \begin{itemize}
            \item Verbesserung normiert auf Investitionskosten
            \item Hohe Effizienz bei günstigen Wandlern
            \item Teure Systeme profitieren weniger stark
        \end{itemize}
    }
    \note{
        \textbf{Wirtschaftliche Analyse}
        Wenn wir die technische Verbesserung in Relation zum Kaufpreis setzen ändert sich das Bild.

        \textbf{Ergebnis}
        Da die Standardwandler sehr günstig sind ist ihr Gewinn an Genauigkeit pro investiertem Euro sehr hoch.
        Die FFP Wandler sind zwar technisch führend aber der relative Gewinn durch die Geometrie fällt kaufmännisch weniger ins Gewicht da die Basiskosten bereits höher sind.
    }
\end{frame}