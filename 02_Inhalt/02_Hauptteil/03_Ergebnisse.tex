\section{Messergebnisse 3000 A}

% --- 3000 A: Fehlerkurven ---
\begin{frame}{Fehlerkurven bei \SI{3000}{A}}
    \centering
    % Pfad geprüft: 03_Ressourcen/diagramme/dia_3000A_kosten/dia_3000A_kosten-Zusammenfassung_MultiCurrent.png
    \includegraphics[width=\textwidth, height=0.85\textheight, keepaspectratio]{03_Ressourcen/diagramme/dia_3000A_kosten/dia_3000A_kosten-Zusammenfassung_MultiCurrent.png}

    % Zeitplanung
    \slidetime{10:00}{11:00}
    \keynote{
        \begin{itemize}
            \item Bunt (Parallel): Starke Spreizung
            \item Grau (Dreieck): Viel engeres Fehlerband
            \item Sättigung beginnt
        \end{itemize}
    }
    \note{
        \textbf{Szenario 3000 A - Fehlerverlauf:}
        Hier sehen wir die Fehlerkurven bei 3000 Ampere.

        \textbf{Beobachtung:}
        Die bunten Linien (Parallelanordnung) zeigen eine deutliche Spreizung zwischen den Phasen. Man sieht, wie der Fehler bei steigender Last (x-Achse) stark abfällt – ein Zeichen für beginnende Sättigung.

        \textbf{Vergleich:}
        Die grauen Linien (Dreiecksanordnung) liegen deutlich enger beisammen und verlaufen stabiler. Die geometrische Optimierung wirkt hier bereits sehr gut.
    }
\end{frame}


\section{Messergebnisse 4000 A}

% --- 4000 A: Fehlerkurven ---
\begin{frame}{Fehlerkurven bei \SI{4000}{A} (Kritischer Bereich)}
    \centering
    % Pfad geprüft: 03_Ressourcen/diagramme/dia_4000A_kosten/dia_4000A_kosten-Zusammenfassung_MultiCurrent.png
    \includegraphics[width=\textwidth, height=0.85\textheight, keepaspectratio]{03_Ressourcen/diagramme/dia_4000A_kosten/dia_4000A_kosten-Zusammenfassung_MultiCurrent.png}

    \slidetime{11:45}{12:45}
    \keynote{
        \begin{itemize}
            \item 4000 A = Extremfall
            \item Parallel: Totalausfall (Absturz der Kurve)
            \item Dreieck: Hält noch stand
        \end{itemize}
    }
    \note{
        \textbf{Eskalation auf 4000 A:}
        Bei 4000 Ampere sehen wir drastische Effekte.

        \textbf{Analyse:}
        Achten Sie auf die orange Linie in der Mitte (Phase L2, Parallel). Der Fehler stürzt komplett ab auf unter -6\%. Das ist messtechnisch ein Totalausfall.
        Selbst die Dreiecksanordnung (gestrichelt, grau/blau) kommt hier an ihre Grenzen, bleibt aber noch deutlich stabiler als die Standardanordnung.
    }
\end{frame}



\section{Definition der Vergleichsmetriken}

\begin{frame}{Berechnungsgrundlagen der Analyse}
    \small % Schrift etwas kleiner
    Um die Diagramme korrekt zu interpretieren, hier die Methodik:

    \vspace{0.2cm}

    \textbf{1. Mittlerer Gesamtfehler (Basiswert)}
    \begin{equation*}
        E_{\text{total}} = \frac{1}{3} \sum_{\text{Phasen}} \left( \frac{1}{n} \sum_{\text{Last}} |F_{\text{Messwert}}| \right)
    \end{equation*}
    \footnotesize{Durchschnitt der \textbf{Beträge} über alle Lastpunkte (5\%--120\%) und Phasen.}

    \vspace{0.3cm}
    \hrule
    \vspace{0.3cm}

    \begin{columns}[t]
        \begin{column}{0.48\textwidth}
            \centering
            \textbf{2. Geom. Verbesserung (\%)}
            \begin{equation*}
                \eta_{\text{geo}} = \left( 1 - \frac{E_{\text{Dreieck}}}{E_{\text{Parallel}}} \right) \cdot 100
            \end{equation*}
            \footnotesize{Anteil des eliminierten Fehlers durch Geometrie-Wechsel.}
        \end{column}
        \begin{column}{0.48\textwidth}
            \centering
            \textbf{3. Wirtschaftlichkeit (\%/€)}
            \begin{equation*}
                \eta_{\text{eco}} = \frac{\eta_{\text{geo}}}{\text{Preis (€)}}
            \end{equation*}
            \footnotesize{Technischer Gewinn normiert auf Investitionskosten.}
        \end{column}
    \end{columns}

    \slidetime{12:45}{12:45}
    \note{
        \textbf{Folie 2: Die Vergleichsmetriken}
        
        Darauf aufbauend haben wir zwei Kennzahlen für die Diagramme:
        
        1. Die **Geometrische Verbesserung**: Wir vergleichen den eben gezeigten Fehler der Parallelanordnung mit dem der Dreiecksanordnung. Ein Wert von 90\% heißt: 90\% des Fehlers sind weg.
        
        2. Die **Wirtschaftlichkeit**: Hier teilen wir die technische Verbesserung durch den Kaufpreis. Das zeigt uns, "wie viel Verbesserung ich pro investiertem Euro" bekomme.
    }
\end{frame}
\section{Gesamtvergleich und Effizienz}

% --- DIAGRAMM 1: Verbesserung Dreieck (%) ---
\begin{frame}{Verbesserung durch Dreiecksanordnung (in \%)}
    \centering
    % KORRIGIERT: diag1_dreieck_pct.png statt diagramm1...
    \includegraphics[width=\textwidth, height=0.85\textheight, keepaspectratio]{03_Ressourcen/diagramme/vergleiche/diag1_dreieck_pct.png}

    \slidetime{12:45}{13:30}
    \keynote{
        \begin{itemize}
            \item Zeigt: Wie viel \% besser ist Dreieck als Parallel?
            \item 3000 A: Enorme Verbesserung (>90\%)
            \item Standard-Wandler (Rot) profitieren am meisten
        \end{itemize}
    }
    \note{
        \textbf{Geometrische Optimierung (Diagramm 1):}
        
        Hier sehen wir, wie viel Prozent Fehler wir allein durch die Änderung der Geometrie von Parallel auf Dreieck einsparen.
        
        \textbf{Auffällig:}
        Besonders bei 3000 A (in der Mitte) sehen wir riesige Balken. Das bedeutet, hier eliminiert die Dreiecksanordnung fast den gesamten Fehler, der in der Parallelanordnung auftrat.
        
        \textbf{Fazit:}
        Der "billige" Standard-Wandler (Rot) profitiert am stärksten von der geometrischen Anordnung.
    }
\end{frame}

% --- DIAGRAMM 2: Wirtschaftlichkeit Dreieck (%/€) ---
\begin{frame}{Wirtschaftlichkeit der Geometrie (\% pro €)}
    \centering
    % KORRIGIERT: diag2_dreieck_pct_eur.png statt diagramm2...
    \includegraphics[width=\textwidth, height=0.85\textheight, keepaspectratio]{03_Ressourcen/diagramme/vergleiche/diag2_dreieck_pct_eur.png}

    \slidetime{13:30}{14:15}
    \keynote{
        \begin{itemize}
            \item Verbesserung normiert auf den Preis
            \item Hohe Balken = Viel Leistung für wenig Geld
            \item Standard (Rot) schlägt High-Tech
        \end{itemize}
    }
    \note{
        \textbf{Ökonomischer Blick (Diagramm 2):}
        
        Wenn wir die Verbesserung in Relation zum Kaufpreis setzen, ändert sich das Bild.
        
        \textbf{Ergebnis:}
        Da die Standard-Wandler (MBS, Rot) sehr günstig sind, ist ihr "Verbesserungs-Gewinn pro investiertem Euro" extrem hoch.
        Die teuren FFP-Wandler (Orange) sind zwar technisch top, aber der relative Gewinn durch die Geometrie fällt kaufmännisch weniger ins Gewicht, da sie ohnehin schon teurer sind.
    }
\end{frame}

% --- DIAGRAMM 4: Absoluter Fehler Parallel ---
\begin{frame}{Technologie-Vergleich: Absoluter Fehler (Parallel)}
    \centering
    % KORRIGIERT: diag4_parallel_error.png statt diagramm4...
    \includegraphics[width=\textwidth, height=0.85\textheight, keepaspectratio]{03_Ressourcen/diagramme/vergleiche/diag4_parallel_error.png}

    \slidetime{14:15}{15:00}
    \keynote{
        \begin{itemize}
            \item Vergleich der Technologien (Worst-Case Parallel)
            \item Rot (Standard): Fehler explodiert bei 2500/3000 A
            \item Orange (FFP): Konstant niedrig
        \end{itemize}
    }
    \note{
        \textbf{Technologischer Vergleich (Diagramm 4):}
        
        Abschließend der Blick auf die reinen Technologien in der kritischen Parallelanordnung.
        
        \textbf{Analyse:}
        Man sieht deutlich, wie der Standard-Wandler (Rot) bei 2500 A und 3000 A massive Fehlerwerte aufweist.
        Im Gegensatz dazu bleibt der FFP-Wandler (Orange, ganz rechts in den Gruppen) fast über das gesamte Spektrum kaum sichtbar, also nahe Null.
        
        \textbf{Schlussfolgerung:}
        Wer keine Geometrie-Optimierung vornehmen kann (Platzmangel), \textit{muss} zu FFP oder kompensierten Wandlern greifen.
    }
\end{frame}