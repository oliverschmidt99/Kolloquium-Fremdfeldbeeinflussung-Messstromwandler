% ==========================================
% TEIL 1: VERSUCHSAUFBAU
% ==========================================
\section{Versuchsaufbau und Durchführung}

% --- Folie 6: Messstrecke ---
\begin{frame}{Versuchsaufbau: Hochstrom-Prüfstand}

    \begin{center}
        \includegraphics[height=0.75\textheight, width=0.95\textwidth, keepaspectratio]{03_Ressourcen/zeichnungen/aufbaz_hochstrom_pruefstand_neu.drawio}
    \end{center}

    \slidetime{08:00}{08:50}
    \keynote{
        \begin{itemize}
            \item Referenz -> Trafo -> 4kA
            \item Synchrone Messung
            \item WinCC Erfassung
        \end{itemize}
    }
    \note{
        \begin{itemize}
            \item \textbf{Ziel:} Reproduzierbare, \textbf{synchrone} Vergleichsmessung (gleicher Strom, gleiche Bedingungen).
            \item \textbf{Messkette:} Referenz (PAC 4220) vs. Prüfling (PAC 3220 + Wandler) \(\Rightarrow\) direkte Abweichung \(\varepsilon\) gegen \textbf{Klasse~1}.
            \item \textbf{Automatisierung:} SPS/WinCC steuert Stromstufen und Haltezeiten \(\Rightarrow\) Einbrüche sind \textbf{physikalisch} (Sättigung), kein Messartefakt.
        \end{itemize}
}
\end{frame}



% ==========================================
% TEIL 2: DETAILLIERTE MESSERGEBNISSE
% ==========================================
\section{Exemplarische Messergebnisse}

\subsection{Messung bei 2000 A}

% --- 2000 A: Referenzanalyse (Redur 13A1030.ffp mit FFP) ---
\begin{frame}{Redur 13A1030.ffp (FFP)}
    \centering
    \includegraphics[width=\textwidth, height=0.85\textheight, keepaspectratio]{03_Ressourcen/diagramme/redur_2000A_special.png}

    \slidetime{09:35}{11:00}
    \keynote{
        \begin{itemize}
            \item Referenzmessung im weitgehend linearen Bereich (\SI{2000}{A})
            \item Parallel: durch seitliche FFP in allen Phasen normkonform (Klasse 1)
            \item Dreieck: Ausreißer bei L2 / 120\,\%~$I_\mathrm{n}$ (ca.\,$\varepsilon=-1{,}50\,\%$) \(\rightarrow\) außerhalb Klasse 1
        \end{itemize}
    }
    \note{
        \begin{itemize}
            \item \textbf{\SI{2000}{A} = Referenz (linear):} Abweichungen nahe \(\SI{0}{\percent}\).
            \item \textbf{Parallel:} seitliche \textbf{FFP} \(\Rightarrow\) normkonform (Klasse~1) in allen Phasen.
            \item \textbf{Dreieck:} \textbf{L2} bei \(120\,\%~I_\mathrm{n}\) ca.\ \(\varepsilon\approx -1{,}50\,\%\) \(\Rightarrow\) außerhalb Klasse~1.
            \item \textbf{Interpretation:} ungeschirmte Rückseite \(\rightarrow\) Fremdfeld-Einkopplung \(\rightarrow\) partielle \textbf{Kernsättigung}.
            \item \textbf{Überleitung:} Bei \SI{4000}{A} wird der Sättigungseffekt dominant.
        \end{itemize}
}
\end{frame}






\subsection{Messung bei 4000 A}


% --- 4000 A: Vergleich Standard (Dreieck) vs. Kompensiert ---
\begin{frame}{Standard (Dreieck) vs. Kompensiert}
    \centering
    \includegraphics[width=\textwidth, height=0.85\textheight, keepaspectratio]{03_Ressourcen/diagramme/verlauf_4000A_presentation.png}

    \slidetime{11:00}{12:15}
    \keynote{
        \begin{itemize}
            \item Fokus: \textbf{Standard in Dreieck} vs. \textbf{kompensierter Wandler}
            \item Standard (Dreieck): bis ca.\ 80--100\,\% relativ stabil, am oberen Lastpunkt deutliche Abweichungen (teils außerhalb Klasse 1)
            \item Kompensiert: in allen Phasen deutlich näher an 0\,\% \(\rightarrow\) Sättigungs-/Fremdfeldeinfluss stark reduziert
            \item Kernaussage: \textbf{Kompensation} ist bei \SI{4000}{A} wirksamer als die reine Geometrieoptimierung
        \end{itemize}
    }
    \note{
        \begin{itemize}
            \item \textbf{Blickführung:} hellblau gestrichelt = \textbf{Standard (Dreieck)}, dunkelblau = \textbf{kompensiert}.
            \item \textbf{Standard (Dreieck):} bis ca.\ 80--100\,\% stabil, am oberen Lastpunkt Ausreißer \(\Rightarrow\) Klasse~1 nicht durchgängig.
            \item \textbf{Kompensiert:} Abweichungen nahe \(\SI{0}{\percent}\) \(\Rightarrow\) Fremdfeld-/Sättigungseinfluss stark reduziert.
            \item \textbf{Take-away:} Bei \SI{4000}{A} ist \textbf{Kompensation} der dominante Hebel; Geometrie ist sekundär.
        \end{itemize}
}
\end{frame}






% --- Zusammenfassung: Parallel vs. Dreieck (Verbesserung) ---
\begin{frame}{Parallel vs. Dreieck: Verbesserung}
    \centering
    \includegraphics[width=\textwidth, height=0.85\textheight, keepaspectratio]{03_Ressourcen/diagramme/diag1_dreieck_pct.png}

    \slidetime{12:15}{12:55}
    \keynote{
        \begin{itemize}
            \item Standard-Wandler profitieren deutlich von \textbf{Parallel}-Anordnung (z.\,B. ALO~10030/12070: typ. \(\approx\)\,\(+70\) bis \(+80\)\,\% bei 2--4\,kA)
            \item \textbf{Kompensierte} Varianten: geringere, teils negative Effekte \(\rightarrow\) primärer Hebel bleibt die Kompensation
            \item Auffälligkeit: Redur~13A1030.3ffp mit FFP bei \SI{2000}{A} zeigt eine negative Verbesserung \(\rightarrow\) hohe Sensitivität auf Feldführung/Anordnung
        \end{itemize}
    }
    \note{
        \begin{itemize}
            \item \textbf{Definition:} „Verbesserung“ = relative Annäherung an \(\varepsilon=0\,\%\) beim Wechsel von \textbf{Dreieck} zu \textbf{Parallel}.
            \item \textbf{Kernaussage:} Für \textbf{Standard}-Wandler ist die \textbf{Anordnung} ein sehr wirksamer Hebel (große Zugewinne in mehreren Stromstufen).
            \item \textbf{Kompensierte} Wandler sind bereits gegen Fremdfeld/Sättigung gehärtet \(\Rightarrow\) weniger Zusatznutzen durch reine Geometrieänderung.
            \item \textbf{Randfall (FFP):} Das Ergebnis deutet auf \textbf{wechselseitige Effekte} zwischen FFP-Positionierung und Leitergeometrie hin.
        \end{itemize}
    }
\end{frame}


% --- Kostenvergleich der Wandler ---
\begin{frame}{Übersicht der Kosten der Wandler}
    \centering
    \includegraphics[width=\textwidth, height=0.85\textheight, keepaspectratio]{03_Ressourcen/diagramme/diag_kosten.png}

    \slidetime{12:55}{13:25}
    \keynote{
        \begin{itemize}
            \item Standard-Wandler liegen im Bereich von ca. \(\SI{44}{\euro}\) bis \(\SI{72}{\euro}\) (je nach Typ/Stromstufe)
            \item FFP-Lösung (Redur~13A1030.3ffp) im mittleren Kostenniveau (\(\approx\SI{146}{\euro}\))
            \item Kompensation ist der Kostentreiber: ALO~12070K bei 3--4\,kA (\(\approx\SI{346}{\euro}\) bis \(\SI{402}{\euro}\))
        \end{itemize}
    }
    \note{
        \begin{itemize}
            \item \textbf{Take-away:} Die \textbf{Anordnung} ist eine \textbf{kostenneutrale} Optimierung (keine Mehrkosten), sofern derselbe Wandler genutzt wird.
            \item \textbf{Trade-off:} Kompensierte Wandler liefern die stabilsten Ergebnisse bei hohen Strömen, sind aber deutlich teurer.
            \item \textbf{Praktischer Schluss:} Für Anwendungen bis ~2--3\,kA kann \textbf{Parallel} bei Standard-Wandlern ein sehr gutes Preis/Leistungs-Verhältnis liefern.
        \end{itemize}
    }
\end{frame}
