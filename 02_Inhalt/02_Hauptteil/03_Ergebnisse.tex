% ==========================================
% TEIL 1: VERSUCHSAUFBAU
% ==========================================
\section{Versuchsaufbau und Durchführung}

% --- Folie 6: Messstrecke ---
\begin{frame}{Versuchsaufbau: Hochstrom-Prüfstand}

    \begin{center}
        \includegraphics[height=0.75\textheight, width=0.95\textwidth, keepaspectratio]{03_Ressourcen/zeichnungen/aufbaz_hochstrom_pruefstand_neu.drawio}
    \end{center}

    \slidetime{07:05}{07:55}
    \keynote{
        \begin{itemize}
            \item Referenz -> Trafo -> 4kA
            \item Synchrone Messung
            \item WinCC Erfassung
        \end{itemize}
    }
    \note{
        \textbf{Ziel des Aufbaus}
        Damit die Unterschiede zwischen Wandlern, FFP und Geometrie wirklich vergleichbar sind, muss die Messung reproduzierbar und synchron sein: \emph{gleicher Strom, gleiche Temperaturbedingungen, gleiche Auswertung}.

        \textbf{Messkette / Signalfluss}
        Wir erzeugen den hohen Prüfstrom über den Hochstromtrafo und regeln ihn stufenweise bis \SI{4000}{A}. Parallel dazu messen wir synchron:
        \begin{itemize}
            \item \textbf{Referenz:} PAC 4220 (hohe Genauigkeit) als ``Wahrheit''.
            \item \textbf{Prüfling:} PAC 3220 + der jeweils getestete Stromwandler.
        \end{itemize}
        Aus der Differenz zwischen Referenz und Prüfling erhalten wir direkt den Übersetzungsfehler und können Normgrenzen (Klasse 1) bewerten.

        \textbf{Automatisierung und Datenqualität}
        Die komplette Sequenz läuft SPS-geführt (Profinet), inklusive Haltezeiten zum Einschwingen. Dadurch eliminieren wir typische Fehlerquellen wie manuelles Nachregeln, Ablesefehler oder variable Einschwingzeiten. Die Messwerte werden in WinCC mit Zeitstempel archiviert, was eine saubere Nachauswertung ermöglicht.

        \textbf{Wichtiger Punkt für die späteren Ergebnisse}
        So stellen wir sicher, dass ein Kurveneinbruch (z.\,B. bei L2/Parallel) \emph{wirklich} aus der Kernphysik/Sättigung kommt und nicht aus einem Messartefakt des Prüfstands.
    }
\end{frame}



% ==========================================
% TEIL 2: DETAILLIERTE MESSERGEBNISSE
% ==========================================
\section{Exemplarische Messergebnisse}

\subsection{Messung bei 2000 A}

% --- 2000 A: Genauigkeitsmessung ---
\begin{frame}{Genauigkeitsmessung bei \SI{2000}{A} (Linearer Bereich)}
    \centering
    \includegraphics[width=\textwidth, height=0.85\textheight, keepaspectratio]{03_Ressourcen/diagramme/verlauf_2000A_presentation.png}

    \slidetime{08:40}{10:05}
    \keynote{
        \begin{itemize}
            \item Referenzmessung im Nennbereich
            \item Alle Wandler arbeiten präzise
            \item Kaum Unterschiede zwischen Parallel und Dreieck
        \end{itemize}
    }
    \note{
        \textbf{Baseline bei \SI{2000}{A}}
        Diese Messung dient als Referenz im weitgehend linearen Bereich: Die Kurven liegen eng um 0\,\%,
        Fremdfeld- und Sättigungseffekte sind hier noch klein.

        \textbf{So liest man die Kurven}
        Durchgezogen = Parallel, gestrichelt = Dreieck (gleicher Wandler, andere Geometrie).
        Wichtig ist besonders L2, weil dort später der stärkste Fremdfeldeinfluss zu erwarten ist.

        \textbf{Überleitung}
        Wenn bei \SI{4000}{A} starke Abweichungen auftreten, ist das damit klar als physikalischer Nichtlinearitäts-/Sättigungseffekt
        einzuordnen – nicht als Messstreuung.
    }
\end{frame}

\begin{frame}{Aufbau FFP-Wandler (kompensiert)}
    \centering
    \includegraphics[width=\textwidth, keepaspectratio]
    {03_Ressourcen/zeichnungen/aufbau_wandler_FFP.drawio.pdf}

    \slidetime{05:40}{06:10}

    \note{
        \textbf{Was sieht man in der Skizze?}
        Der Primärleiter erzeugt das Hauptmagnetfeld im Ringkern.
        Die zusätzlichen Wicklungen detektieren Streufelder und führen über die Kompensationswicklung
        ein gezieltes Gegenfeld ein.

        \textbf{Physikalische Idee}
        Dadurch bleibt der resultierende Fluss im Eisenkern näher am linearen Bereich.
        Die lokale Sättigung – insbesondere bei asymmetrischen Feldverteilungen – wird reduziert.

        \textbf{Einordnung}
        Das ist eine aktive, konstruktive Lösung.
        Im späteren Vergleich prüfen wir, ob eine rein geometrische Maßnahme
        (Dreiecksanordnung der Leiter) einen ähnlichen Effekt erzielen kann –
        jedoch ohne Zusatzwicklung und ohne Mehrkosten.
    }

\end{frame}





\subsection{Messung bei 4000 A}

% --- 4000 A: Genauigkeitsmessung ---
\begin{frame}{Genauigkeitsmessung bei \SI{4000}{A} (Kritischer Bereich)}
    \centering
    \includegraphics[width=\textwidth, height=0.85\textheight, keepaspectratio]{03_Ressourcen/diagramme/verlauf_4000A_presentation.png}

    \slidetime{10:05}{11:45}
    \keynote{
        \begin{itemize}
            \item 4000 A = Extremfall
            \item Parallel zeigt Totalausfall (Absturz der Kurve)
            \item Dreieck hält die Genauigkeit deutlich länger
        \end{itemize}
    }
    \note{
        \textbf{Eskalation auf 4000 A}
        Im direkten Vergleich sehen wir nun die drastischen Effekte bei 4000 Ampere.

        \textbf{Analyse}
        Achten Sie auf die hellblaue Linie in der Mitte (Phase L2, Parallel). Der Fehler stürzt komplett ab auf unter -6\,\%. Das ist messtechnisch ein Totalausfall durch Sättigung.

        \textbf{Lösung}
        Die Dreiecksanordnung (dunkelblau gestrichelt) kommt zwar auch an ihre Grenzen, bleibt aber noch stabil genug, um nutzbare Werte zu liefern. Der Unterschied zur vorherigen Folie ist massiv.
    }
\end{frame}


% ==========================================
% TEIL 3: ANALYSE UND WIRTSCHAFTLICHKEIT
% ==========================================
\section{Vergleichende Analyse}



