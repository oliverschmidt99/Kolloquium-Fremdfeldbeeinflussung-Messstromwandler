\section{Messergebnisse 3000 A}

% --- 3000 A: Fehlerkurven ---
\begin{frame}{Fehlerkurven bei \SI{3000}{A}}
    \centering
    % Pfad: dia_3000A_kosten/dia_3000A_kosten-Zusammenfassung_MultiCurrent.png
    \includegraphics[width=\textwidth, height=0.85\textheight, keepaspectratio]{03_Ressourcen/diagramme/dia_3000A_kosten/dia_3000A_kosten-Zusammenfassung_MultiCurrent.png}

    % Zeitplanung
    \slidetime{10:00}{11:00}
    \keynote{
        \begin{itemize}
            \item Bunt (Parallel): Starke Spreizung
            \item Grau (Dreieck): Viel engeres Fehlerband
            \item Sättigung beginnt
        \end{itemize}
    }
    \note{
        \textbf{Szenario 3000 A - Fehlerverlauf:}
        Hier sehen wir die Fehlerkurven bei 3000 Ampere.

        \textbf{Beobachtung:}
        Die bunten Linien (Parallelanordnung) zeigen eine deutliche Spreizung zwischen den Phasen. Man sieht, wie der Fehler bei steigender Last (x-Achse) stark abfällt – ein Zeichen für beginnende Sättigung.

        \textbf{Vergleich:}
        Die grauen Linien (Dreiecksanordnung) liegen deutlich enger beisammen und verlaufen stabiler. Die geometrische Optimierung wirkt hier bereits sehr gut.
    }
\end{frame}

% --- 3000 A: Ranking ---
\begin{frame}{Ökonomisches Ranking bei \SI{3000}{A}}
    \centering
    % Pfad: dia_3000A_kosten/dia_3000A_kosten-Oekonomie_Ranking_Dual.png
    \includegraphics[width=\textwidth, height=0.85\textheight, keepaspectratio]{03_Ressourcen/diagramme/dia_3000A_kosten/dia_3000A_kosten-Oekonomie_Ranking_Dual.png}

    \slidetime{11:00}{11:45}
    \keynote{
        \begin{itemize}
            \item Preis vs. Fehler
            \item Parallel (Bunt): Hohe Fehler
            \item Dreieck (Orange/Gelb): Gutes P/L-Verhältnis
        \end{itemize}
    }
    \note{
        \textbf{Ranking 3000 A:}
        Hier sind die Kosten (grüner Balken) gegen die Fehlerarten (bunt) aufgetragen.

        \textbf{Ergebnis:}
        Während die Parallel-Varianten (ganz unten) zwar günstig sind, weisen sie sehr hohe Fehlerbalken auf (rot/orange).
        Die Dreiecksvarianten (oben) bieten hier den besten Kompromiss: Die Fehler sind minimal, ohne dass die Kosten so explodieren wie bei vollkompensierten Wandlern.
    }
\end{frame}

\section{Messergebnisse 4000 A}

% --- 4000 A: Fehlerkurven ---
\begin{frame}{Fehlerkurven bei \SI{4000}{A} (Kritischer Bereich)}
    \centering
    % Pfad: dia_4000A_kosten/dia_4000A_kosten-Zusammenfassung_MultiCurrent.png
    \includegraphics[width=\textwidth, height=0.85\textheight, keepaspectratio]{03_Ressourcen/diagramme/dia_4000A_kosten/dia_4000A_kosten-Zusammenfassung_MultiCurrent.png}

    \slidetime{11:45}{12:45}
    \keynote{
        \begin{itemize}
            \item 4000 A = Extremfall
            \item Parallel: Totalausfall (Absturz der Kurve)
            \item Dreieck: Hält noch stand
        \end{itemize}
    }
    \note{
        \textbf{Eskalation auf 4000 A:}
        Bei 4000 Ampere sehen wir drastische Effekte.

        \textbf{Analyse:}
        Achten Sie auf die orange Linie in der Mitte (Phase L2, Parallel). Der Fehler stürzt komplett ab auf unter -6\%. Das ist messtechnisch ein Totalausfall.
        Selbst die Dreiecksanordnung (gestrichelt, grau/blau) kommt hier an ihre Grenzen, bleibt aber noch deutlich stabiler als die Standardanordnung.
    }
\end{frame}



\section{Gesamtvergleich und Effizienz}

% --- Gesamtfehler Absolut ---
\begin{frame}{Vergleich Gesamtfehler (Summe L1+L2+L3)}
    \centering
    % Dateiname: vergleich_fehler_absolut.png
    \includegraphics[width=\textwidth, height=0.85\textheight, keepaspectratio]{vergleich_fehler_absolut.png}

    \note{
        \textbf{Gesamtbetrachtung der absoluten Fehler}
        
        Die Grafik zeigt die aufsummierten Absolutbeträge der Fehler über alle Phasen
        
        \textbf{Auffälligkeiten}
        Besonders bei 2500 A sticht der rote Balken der Standard-Parallel-Konfiguration hervor
        Dies deutet auf eine ungünstige Lastverteilung oder Sättigung hin
        
        \textbf{Vergleich}
        Die blauen und grünen Balken (Kompensiert und FFP) bleiben über den gesamten Bereich niedrig
        Die geometrische Anordnung allein (rosa) bringt bereits eine Verbesserung
    }
\end{frame}

% --- Fehler pro Euro ---
\begin{frame}{Ökonomische Bewertung (Fehler pro Euro)}
    \centering
    % Dateiname: vergleich_fehler_pro_euro.png
    \includegraphics[width=\textwidth, height=0.85\textheight, keepaspectratio]{vergleich_fehler_pro_euro.png}

    \note{
        \textbf{Normierung auf die Kosten}
        
        Hier wird der Fehler ins Verhältnis zum Preis gesetzt
        Ein niedriger Balken bedeutet viel Präzision für das investierte Geld
        
        \textbf{Ergebnis}
        Trotz höherer Anschaffungskosten schneiden die kompensierten Varianten oft besser ab
        Der hohe Fehler der Standard-Variante bei 2500 A verschlechtert das Preis-Leistungs-Verhältnis deutlich
    }
\end{frame}

% --- Effizienzfaktoren ---
\begin{frame}{Effizienz-Steigerung gegenüber Standard Parallel}
    \centering
    % Dateiname: effizienz_faktoren_alle.png
    \includegraphics[width=\textwidth, height=0.85\textheight, keepaspectratio]{effizienz_faktoren_alle.png}

    \note{
        \textbf{Verbesserungsfaktor}
        
        Dieses Diagramm zeigt, um welchen Faktor die anderen Methoden besser sind als der Standard
        
        \textbf{Highlights}
        Bei 2500 A und 4000 A erreichen die technischen Kompensationen (blaue Balken) sehr hohe Werte
        Faktoren von über 60 zeigen das Potential der Technik in kritischen Bereichen
        In unkritischen Bereichen (z B 5000 A) ist der Gewinn geringer
    }
\end{frame}