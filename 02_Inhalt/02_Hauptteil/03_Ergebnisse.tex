\section{Messergebnisse}

% --- Szenario 1: Der "Standard"-Bereich ---
\subsection{Vergleich bei 2000 A}

\begin{frame}{Marktüberblick bei \SI{2000}{A}}
    \vspace{-0.2cm}
    \begin{columns}[c]
        \begin{column}{0.65\textwidth}
            \centering
            % Pfad zum Diagramm 2000A Zusammenfassung
            \includegraphics[height=0.75\textheight, keepaspectratio]{03_Ressourcen/diagramme/dia_2000A_kosten/dia_2000A_kosten-Zusammenfassung_MultiCurrent.pdf}
        \end{column}
        \begin{column}{0.35\textwidth}
            \textbf{Beobachtungen}
            \small
            \begin{itemize}
                \item \textbf{Parallel (bunt):} Starke Spreizung der Fehlerkurven.
                \item \textbf{Dreieck (grau):} Fehlerband deutlich schmaler.
                \item \textbf{FFP (Redur):} Zeigt bereits hier die stabilsten Werte.
            \end{itemize}
        \end{column}
    \end{columns}

    % Zeitplanung: ca. 1 Minute
    \slidetime{10:00}{11:00}
    \keynote{
        \begin{itemize}
            \item Bunte Linien = Parallel (Schlecht)
            \item Graue Linien = Dreieck (Besser)
            \item FFP stabil
        \end{itemize}
    }
    \note{
        \textbf{Szenario 2000 A:}
        Hier sehen wir den Überblick über verschiedene Hersteller bei 2000 Ampere.

        \textbf{Erkenntnis:}
        Die bunten Linien zeigen die klassische Parallelanordnung. Man sieht hier schon eine deutliche Spreizung der Messfehler zwischen den Phasen.

        \textbf{Geometrie-Effekt:}
        Sobald wir auf die Dreiecksanordnung wechseln (die grauen Linien), rücken die Fehlerkurven deutlich enger zusammen. Die rein geometrische Maßnahme zeigt also bereits Wirkung.
    }
\end{frame}


% --- Szenario 2: Der Übergangsbereich ---
\subsection{Vergleich bei 3000 A}

\begin{frame}{Validierung bei \SI{3000}{A}}
    \vspace{-0.2cm}
    \begin{columns}[c]
        \begin{column}{0.65\textwidth}
            \centering
            % Pfad zum Diagramm 3000A Zusammenfassung
            \includegraphics[height=0.75\textheight, keepaspectratio]{03_Ressourcen/diagramme/dia_3000A_kosten/dia_3000A_kosten-Zusammenfassung_MultiCurrent.pdf}
        \end{column}
        \begin{column}{0.35\textwidth}
            \textbf{Entwicklung des Fehlers}
            \small
            \begin{itemize}
                \item Sättigungseffekte nehmen zu.
                \item \textbf{Parallel:} Fehler wachsen überproportional an.
                \item \textbf{Dreieck:} Bleibt stabil, aber erste Anzeichen von Sättigung.
            \end{itemize}
        \end{column}
    \end{columns}

    % Zeitplanung: ca. 45 Sekunden
    \slidetime{11:00}{11:45}
    \keynote{
        \begin{itemize}
            \item Trend setzt sich fort
            \item Sättigung beginnt früher
            \item Parallelanordnung kritisch
        \end{itemize}
    }
    \note{
        \textbf{3000 Ampere:}
        Bei Erhöhung des Stroms auf 3000 Ampere bestätigt sich der Trend.

        \textbf{Beobachtung:}
        Die Fehler in der Parallelanordnung wachsen überproportional an, da der Eisenkern durch die stärkeren Fremdfelder früher in die Sättigung getrieben wird.

        \textbf{Zwischenfazit:}
        Die Dreiecksanordnung hält das System noch stabil, aber wir nähern uns den Grenzen der Standardtechnik.
    }
\end{frame}


% --- Szenario 3: Spezialuntersuchung Bürde ---
\subsection{Einfluss der Bürde}

\begin{frame}{Spezialuntersuchung: Einfluss der Bürde (\SI{3000}{A})}
    \vspace{-0.2cm}
    \begin{columns}[c]
        \begin{column}{0.65\textwidth}
            \centering
            % Pfad zum Diagramm 3000A Bürde
            \includegraphics[height=0.75\textheight, keepaspectratio]{03_Ressourcen/diagramme/dia_3000A_buerde/dia_3000A_buerde-Zusammenfassung_MultiCurrent.pdf}
        \end{column}
        \begin{column}{0.35\textwidth}
            \textbf{Varianten der Sekundärlast}
            \small
            \begin{itemize}
                \item \textbf{0 $\Omega$:} Minimale Belastung (Idealfall).
                \item \textbf{Ref:} Nennbürde (Realfall).
                \item \textbf{Erkenntnis:} Hohe Bürde verschlechtert das Ergebnis signifikant (frühere Sättigung).
            \end{itemize}
        \end{column}
    \end{columns}

    % Zeitplanung: ca. 45 Sekunden
    \slidetime{11:45}{12:30}
    \keynote{
        \begin{itemize}
            \item Test: Kabellänge/Querschnitt
            \item Hohe Bürde = Schlecht
            \item Niedrige Bürde hilft
        \end{itemize}
    }
    \note{
        \textbf{Untersuchung:}
        Hier habe ich geprüft, ob man durch Optimierung der Sekundärseite (z.B. kürzere Kabel oder größerer Querschnitt) den Fehler reduzieren kann.

        \textbf{Ergebnis:}
        Die Kurven zeigen deutlich: Eine niedrige Bürde (nahe 0 Ohm) verzögert die Sättigung und verbessert das Messergebnis. Eine hohe Bürde verschlechtert es massiv.

        \textbf{Praxis:}
        Das bedeutet für die Praxis: Leitungswege kurz halten und Querschnitte großzügig dimensionieren ist eine wirksame, flankierende Maßnahme.
    }
\end{frame}


% --- Szenario 4: Der Extremfall ---
\subsection{Analyse bei 4000 A}

\begin{frame}{Hochstrom-Szenario bei \SI{4000}{A}}
    \vspace{-0.2cm}
    \begin{columns}[c]
        \begin{column}{0.65\textwidth}
            \centering
            % Pfad zum Diagramm 4000A Zusammenfassung
            \includegraphics[height=0.75\textheight, keepaspectratio]{03_Ressourcen/diagramme/dia_4000A_kosten/dia_4000A_kosten-Zusammenfassung_MultiCurrent.pdf}
        \end{column}
        \begin{column}{0.35\textwidth}
            \textbf{Kritischer Bereich}
            \small
            \begin{itemize}
                \item Standardwandler in Parallelanordnung versagen (Klasse 3).
                \item \textbf{Dreieck:} Bringt Verbesserung, reicht aber nicht für Kl. 0,5.
                \item \textbf{FFP:} Hält als einzige Lösung die Klasse 1 sicher ein.
            \end{itemize}
        \end{column}
    \end{columns}

    % Zeitplanung: ca. 1 Minute
    \slidetime{12:30}{13:30}
    \keynote{
        \begin{itemize}
            \item 4000 A = Extrem
            \item Parallel = Totalausfall
            \item Nur FFP sicher
        \end{itemize}
    }
    \note{
        \textbf{Eskalation:}
        Bei 4000 Ampere verschärft sich die Situation drastisch.

        \textbf{Parallelanordnung:}
        Die Standardwandler in Parallelanordnung (bunt) brechen komplett aus und verletzen sogar die Klasse 3. Das ist messtechnisch unbrauchbar.

        \textbf{Lösung:}
        Selbst die Dreiecksanordnung reicht hier für Präzisionsmessungen kaum noch aus. Lediglich die Kombination mit FFP-Wandlern garantiert hier noch die geforderte Genauigkeitsklasse 1.
    }
\end{frame}


% --- Szenario 5: Die Lösung / Ökonomie ---
\subsection{Ökonomische Bewertung}

\begin{frame}{Kosten-Nutzen-Analyse (Ranking)}
    \vspace{-0.2cm}
    \centering
    % Das Diagramm: Preis über mittlerem Fehler (4000A Ranking als Basis)
    \includegraphics[height=0.68\textheight, keepaspectratio]{03_Ressourcen/diagramme/dia_4000A_kosten/dia_4000A_kosten-Oekonomie_Ranking_Dual.pdf}

    \vspace{0.3cm}

    % Definition als Fußnote
    \begin{beamercolorbox}[sep=0.3em,center,rounded=true,shadow=false]{cHSblue!5}
        \tiny \color{gray}
        Datengrundlage: Mittelwert der Messabweichung $\varepsilon$ über drei Lastbereiche: \\
        \textbf{Niederstrom} ($<20\,\% I_n$), \textbf{Nennstrom} ($20\dots120\,\% I_n$) und \textbf{Überlast} ($>120\,\% I_n$).
    \end{beamercolorbox}

    % Zeitplanung: ca. 1 Minute
    \slidetime{13:30}{14:30}
    \keynote{
        \begin{itemize}
            \item Ziel: Unten Links (Günstig \& Gut)
            \item X-Achse = Mittlerer Fehler
            \item Sieger: Celsa Dreieck
        \end{itemize}
    }
    \note{
        \textbf{Entscheidungsgrundlage:}
        Zum Abschluss habe ich die technischen Fehlerwerte in einen wirtschaftlichen Kontext gesetzt.

        \textbf{Methodik:}
        Auf der X-Achse sehen Sie den \textbf{mittleren Fehler} über den gesamten Arbeitsbereich. Auf der Y-Achse den Preis.

        \textbf{Ziel:}
        Wir suchen eine Lösung im Bereich "Links Unten" – also minimaler Fehler bei minimalen Kosten.

        \textbf{Ergebnis:}
        Es zeigt sich klar: Die Celsa-Lösung in \textbf{Dreiecksanordnung (Grau)} bietet hier das Optimum. Sie hält die Klasse 1 ein und kostet nur einen Bruchteil der vollkompensierten Systeme.
    }
\end{frame}