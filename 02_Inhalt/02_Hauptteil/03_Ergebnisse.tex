% ==========================================
% TEIL 1: VERSUCHSAUFBAU
% ==========================================
\section{Versuchsaufbau und Durchführung}

% --- Folie 6: Messstrecke ---
\begin{frame}{Versuchsaufbau: Hochstrom-Prüfstand}
    \centering
    \textbf{Schematischer Aufbau der automatisierten Messstrecke}

    \vspace{0.2cm}

    \includegraphics[height=0.75\textheight, width=0.95\textwidth, keepaspectratio]{03_Ressourcen/zeichnungen/aufbau_hochstrom_pruefstand_new.drawio.pdf}

    \slidetime{08:00}{08:45}
    \keynote{
        \begin{itemize}
            \item Referenz -> Trafo -> 4kA
            \item Synchrone Messung
            \item WinCC Erfassung
        \end{itemize}
    }
    \note{
        \textbf{Aufbau}
        Hier sehen Sie das schematische Diagramm unseres automatisierten Hochstrom-Prüfstands für die Versuchsreihen.

        \textbf{Signalfluss}
        Wir speisen niederspannungsseitig ein und transformieren den Strom über Hochstromtrafos auf bis zu 4000 Ampere hoch.

        \textbf{Vergleich}
        Das Kernstück ist die synchrone Messung zwischen dem Referenzgerät PAC 4220 und dem jeweiligen Prüfling PAC 3220.

        \textbf{Daten}
        Alle Messwerte werden zentral über Profinet erfasst und automatisch in WinCC archiviert, um Ablesefehler sicher auszuschließen.
    }
\end{frame}

% --- Folie 7: Lastprofil ---
\begin{frame}{Lastprofil und Prüfablauf}
    \centering
    \textbf{Visualisierung der automatisierten Messsequenz}

    \vspace{0.2cm}

    \includegraphics[width=\textwidth, height=0.65\textheight, keepaspectratio]{03_Ressourcen/Bilder/verlauf_gesamt.png}

    \vspace{0.3cm}

    \begin{columns}[t, onlytextwidth]
        \begin{column}{0.48\textwidth}
            \textbf{Ablauf}
            \begin{itemize}
                \item Stufenweise Erhöhung (\SI{5}{\%} bis \SI{120}{\%})
                \item Vollautomatisierte Ansteuerung
            \end{itemize}
        \end{column}
        \begin{column}{0.48\textwidth}
            \textbf{Vorteil}
            \begin{itemize}
                \item Identische thermische Belastung
                \item 100\,\% Reproduzierbarkeit
            \end{itemize}
        \end{column}
    \end{columns}

    \slidetime{08:00}{08:45}
    \keynote{
        \begin{itemize}
            \item Treppenstufen = Lastpunkte
            \item Reproduzierbarkeit
            \item Menschlicher Fehler ausgeschlossen
        \end{itemize}
    }
    \note{
        \textbf{Das Lastprofil}
        Hier sehen Sie den \frqq Herzschlag\flqq\ unserer Messung – die direkte Aufzeichnung des Primärstroms aus dem Leitsystem.

        \textbf{Verfahren}
        Wir fahren vollautomatisiert feste Lastpunkte an: Von 5\,\% Teillast bis hinauf zu 120\,\% Überlast. Die Plateaus zeigen die Haltezeiten, in denen sich die Messwerte einschwingen.

        \textbf{Relevanz}
        Das ist entscheidend für die wissenschaftliche Qualität: Jeder Wandler wird exakt demselben Stressprofil ausgesetzt. Menschliche Ablesefehler oder Schwankungen am Regeltrafo sind durch die SPS-Steuerung ausgeschlossen.
    }
\end{frame}


% ==========================================
% TEIL 2: DETAILLIERTE MESSERGEBNISSE
% ==========================================
\section{Exemplarische Messergebnisse}

\subsection{Messung bei 2000 A}

% --- 2000 A: Genauigkeitsmessung ---
\begin{frame}{Genauigkeitsmessung bei \SI{2000}{A} (Linearer Bereich)}
    \centering
    \includegraphics[width=\textwidth, height=0.85\textheight, keepaspectratio]{03_Ressourcen/diagramme/verlauf_2000A_presentation.png}

    \slidetime{10:00}{11:00}
    \keynote{
        \begin{itemize}
            \item Referenzmessung im Nennbereich
            \item Alle Wandler arbeiten präzise
            \item Kaum Unterschiede zwischen Parallel und Dreieck
        \end{itemize}
    }
    \note{
        \textbf{Ausgangslage bei 2000 A}
        Zunächst betrachten wir den unkritischen Bereich bei 2000 Ampere.

        \textbf{Beobachtung}
        Hier arbeiten alle Wandler noch weit unterhalb ihrer Sättigungsgrenze. Die Linien verlaufen fast horizontal und liegen eng beieinander.

        \textbf{Erkenntnis}
        Sowohl die Parallelanordnung (Bunt) als auch die Dreiecksanordnung (Gestrichelt) liefern hier normgerechte Ergebnisse. Das System ist magnetisch noch nicht überlastet.
    }
\end{frame}


\subsection{Messung bei 2000 A}

% --- 2000 A: Genauigkeitsmessung ---
\begin{frame}{Redur}
    \centering
    \includegraphics[width=\textwidth, height=0.85\textheight, keepaspectratio]{03_Ressourcen/zeichnungen/aufbau_wandler_FFP.drawio.pdf}

    \slidetime{10:00}{11:00}
    \keynote{

    }
    \note{

    }
\end{frame}

\subsection{Messung bei 4000 A}

% --- 4000 A: Genauigkeitsmessung ---
\begin{frame}{Genauigkeitsmessung bei \SI{4000}{A} (Kritischer Bereich)}
    \centering
    \includegraphics[width=\textwidth, height=0.85\textheight, keepaspectratio]{03_Ressourcen/diagramme/verlauf_4000A_presentation.png}

    \slidetime{11:45}{12:45}
    \keynote{
        \begin{itemize}
            \item 4000 A = Extremfall
            \item Parallel zeigt Totalausfall (Absturz der Kurve)
            \item Dreieck hält die Genauigkeit deutlich länger
        \end{itemize}
    }
    \note{
        \textbf{Eskalation auf 4000 A}
        Im direkten Vergleich sehen wir nun die drastischen Effekte bei 4000 Ampere.

        \textbf{Analyse}
        Achten Sie auf die hellblaue Linie in der Mitte (Phase L2, Parallel). Der Fehler stürzt komplett ab auf unter -6\,\%. Das ist messtechnisch ein Totalausfall durch Sättigung.

        \textbf{Lösung}
        Die Dreiecksanordnung (dunkelblau gestrichelt) kommt zwar auch an ihre Grenzen, bleibt aber noch stabil genug, um nutzbare Werte zu liefern. Der Unterschied zur vorherigen Folie ist massiv.
    }
\end{frame}


% ==========================================
% TEIL 3: ANALYSE UND WIRTSCHAFTLICHKEIT
% ==========================================
\section{Vergleichende Analyse}

\subsection{Methodik und Kosten}

% --- Metriken ---
\begin{frame}{Berechnungsgrundlagen der Analyse}
    \small
    Um die Diagramme korrekt zu interpretieren, hier die Methodik:

    \vspace{0.2cm}

    \textbf{1. Mittlerer Gesamtfehler (Basiswert)}
    \begin{equation*}
        E_{\text{total}} = \frac{1}{3} \sum_{\text{Phasen}} \left( \frac{1}{n} \sum_{\text{Last}} |F_{\text{Messwert}}| \right)
    \end{equation*}
    \footnotesize{Durchschnitt der \textbf{Beträge} über alle Lastpunkte (5\%--120\%) und Phasen.}

    \vspace{0.3cm}
    \hrule
    \vspace{0.3cm}

    \begin{columns}[t]
        \begin{column}{0.48\textwidth}
            \centering
            \textbf{2. Geom. Verbesserung (\%)}
            \begin{equation*}
                \eta_{\text{geo}} = \left( 1 - \frac{E_{\text{Dreieck}}}{E_{\text{Parallel}}} \right) \cdot 100
            \end{equation*}
            \footnotesize{Anteil des eliminierten Fehlers durch Geometrie-Wechsel.}
        \end{column}
        \begin{column}{0.48\textwidth}
            \centering
            \textbf{3. Wirtschaftlichkeit (\%/€)}
            \begin{equation*}
                \eta_{\text{eco}} = \frac{\eta_{\text{geo}}}{\text{Preis (€)}}
            \end{equation*}
            \footnotesize{Technischer Gewinn normiert auf Investitionskosten.}
        \end{column}
    \end{columns}

    \slidetime{12:45}{12:45}
    \note{
        \textbf{Folie 2: Die Vergleichsmetriken}
        Darauf aufbauend haben wir zwei Kennzahlen für die Diagramme:

        1. Die **geometrische Verbesserung** Wir vergleichen den eben gezeigten Fehler der Parallelanordnung mit dem der Dreiecksanordnung. Ein Wert von 90\% heißt: 90\% des Fehlers sind weg.

        2. Die **Wirtschaftlichkeit** Hier teilen wir die technische Verbesserung durch den Kaufpreis. Das zeigt uns, \frqq wie viel Verbesserung ich pro investiertem Euro\flqq bekomme.
    }
\end{frame}

% --- Kostenübersicht ---
\begin{frame}{Wirtschaftliche Einordnung der Wandler}
    \centering
    \includegraphics[width=\textwidth, height=0.85\textheight, keepaspectratio]{03_Ressourcen/diagramme/diag_kosten_horizontal.png}

    \slidetime{13:30}{14:00}
    \keynote{
        \begin{itemize}
            \item Standardwandler sind sehr günstig
            \item Kompensierte Wandler kosten ein Vielfaches
            \item Preisschere öffnet sich bei hohen Strömen
        \end{itemize}
    }
    \note{
        \textbf{Der Kostenfaktor}
        Bevor wir die Effizienz bewerten, müssen wir die Investitionskosten betrachten.

        \textbf{Vergleich}
        Die hellblauen und roten Balken zeigen die Standardwandler. Diese liegen preislich weit unter 100 Euro.
        Ganz unten sehen sie die dunkelblauen Balken für die kompensierten Wandler bei 3000 und 4000 Ampere. Diese kosten mit über 300 bis 400 Euro ein Vielfaches der Standardgeräte.

        \textbf{Bedeutung}
        Wenn wir mit der Dreiecksanordnung die billigen Wandler auf das Niveau der teuren Wandler heben können, ist der wirtschaftliche Hebel enorm.
    }
\end{frame}


\subsection{Gesamtvergleich und Effizienz}

% --- DIAGRAMM 0: Wahrer Fehler ---
\begin{frame}{Gesamtfehler im Vergleich: Parallel vs. Dreieck}
    \centering
    \includegraphics[width=\textwidth, height=0.85\textheight, keepaspectratio]{03_Ressourcen/diagramme/diag0_wahre_fehler_horizontal.png}

    \slidetime{12:45}{13:30}
    \keynote{
        \begin{itemize}
            \item Volle Balken sind Parallelanordnung
            \item Schraffierte Balken sind Dreiecksanordnung
            \item Massive Reduktion der Balkenlänge durch Dreieck
        \end{itemize}
    }
    \note{
        \textbf{Analyse der absoluten Fehler}
        Dieses Diagramm zeigt die summierten Fehlerwerte über alle Messpunkte.

        \textbf{Lesehilfe}
        Die vollen Balken repräsentieren die klassische Parallelanordnung. Die schraffierten Balken zeigen denselben Wandler in der optimierten Dreiecksanordnung.

        \textbf{Erkenntnis}
        Man sieht auf den ersten Blick, dass die schraffierten Balken fast überall deutlich kürzer sind. Besonders beim Celsa Standardwandler bei 2500 Ampere fällt der Fehler von über 7 auf unter 2 Einheiten. Die Geometrie wirkt hier wie ein physischer Filter gegen Störfelder.
    }
\end{frame}

% --- DIAGRAMM 1: Verbesserung Dreieck (%) ---
\begin{frame}{Wirksamkeit der geometrischen Optimierung}
    \centering
    \includegraphics[width=\textwidth, height=0.85\textheight, keepaspectratio]{03_Ressourcen/diagramme/diag1_dreieck_pct.png}

    \slidetime{12:45}{13:30}
    \keynote{
        \begin{itemize}
            \item Relative Fehlerreduktion durch Geometrie
            \item Über 90 Prozent Verbesserung bei 3000 A
            \item Standardwandler zeigen größten Effekt
        \end{itemize}
    }
    \note{
        \textbf{Geometrische Optimierung}
        Hier sehen wir die prozentuale Fehlerreduktion durch den Wechsel von Parallelanordnung auf Dreiecksanordnung.

        \textbf{Beobachtung}
        Besonders bei 3000 Ampere zeigen sich hohe Werte. Das bedeutet hier eliminiert die Dreiecksanordnung fast den gesamten Fehler der in der Parallelanordnung auftrat.

        \textbf{Schlussfolgerung}
        Der Standardwandler profitiert am stärksten von der geometrischen Anordnung und erreicht dadurch fast die Güte der teureren Systeme.
    }
\end{frame}

% --- DIAGRAMM 2: Wirtschaftlichkeit Dreieck (%/€) ---
\begin{frame}{Ökonomische Bewertung der Maßnahmen}
    \centering
    \includegraphics[width=\textwidth, height=0.85\textheight, keepaspectratio]{03_Ressourcen/diagramme/diag2_dreieck_pct_eur.png}

    \slidetime{13:30}{14:15}
    \keynote{
        \begin{itemize}
            \item Verbesserung normiert auf Investitionskosten
            \item Hohe Effizienz bei günstigen Wandlern
            \item Teure Systeme profitieren weniger stark
        \end{itemize}
    }
    \note{
        \textbf{Wirtschaftliche Analyse}
        Wenn wir die technische Verbesserung in Relation zum Kaufpreis setzen ändert sich das Bild.

        \textbf{Ergebnis}
        Da die Standardwandler sehr günstig sind ist ihr Gewinn an Genauigkeit pro investiertem Euro sehr hoch.
        Die FFP Wandler sind zwar technisch führend aber der relative Gewinn durch die Geometrie fällt kaufmännisch weniger ins Gewicht da die Basiskosten bereits höher sind.
    }
\end{frame}