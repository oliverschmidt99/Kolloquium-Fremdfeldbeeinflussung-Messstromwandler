% ==========================================
% TEIL 1: VERSUCHSAUFBAU
% ==========================================
\section{Versuchsaufbau und Durchführung}

% --- Folie 6: Messstrecke ---
\begin{frame}{Versuchsaufbau: Hochstrom-Prüfstand}

    \begin{center}
        \includegraphics[height=0.75\textheight, width=0.95\textwidth, keepaspectratio]{03_Ressourcen/zeichnungen/aufbaz_hochstrom_pruefstand_neu.drawio}
    \end{center}

    \slidetime{07:05}{07:55}
    \keynote{
        \begin{itemize}
            \item Referenz -> Trafo -> 4kA
            \item Synchrone Messung
            \item WinCC Erfassung
        \end{itemize}
    }
    \note{
        Ziel: reproduzierbare, synchrone Vergleichsmessung (gleicher Strom, gleiche Bedingungen).
        Messkette: Referenz (PAC 4220) vs. Prüfling (PAC 3220 + Wandler) \(\Rightarrow\) direkt \(\varepsilon\) gegen Klasse~1.
        SPS/WinCC steuern Stufen + Haltezeiten \(\Rightarrow\) Kurveneinbruch ist physikalisch (Sättigung), kein Messartefakt.
    }
\end{frame}



% ==========================================
% TEIL 2: DETAILLIERTE MESSERGEBNISSE
% ==========================================
\section{Exemplarische Messergebnisse}

\subsection{Messung bei 2000 A}

% --- 2000 A: Referenzanalyse (Redur 13A1030.ffp mit FFP) ---
\begin{frame}{Redur 13A1030.ffp (FFP)}
    \centering
    \includegraphics[width=\textwidth, height=0.85\textheight, keepaspectratio]{03_Ressourcen/diagramme/redur_2000A_special.png}

    \slidetime{08:40}{10:05}
    \keynote{
        \begin{itemize}
            \item Referenzmessung im weitgehend linearen Bereich (\SI{2000}{A})
            \item Parallel: durch seitliche FFP in allen Phasen normkonform (Klasse 1)
            \item Dreieck: Ausreißer bei L2 / 120\,\%~$I_\mathrm{n}$ (ca.\,$\varepsilon=-1{,}50\,\%$) \(\rightarrow\) außerhalb Klasse 1
            \item Interpretation: ungeschirmte Rückseite \(\rightarrow\) Feld-Einkopplung benachbarter Leiter, partielle Kernsättigung
        \end{itemize}
    }
    \note{
    \textbf{2000 A = Referenz (linear):} Abweichungen nahe 0\,\%.
    \textbf{Parallel:} seitliche FFP \(\Rightarrow\) normkonform.
    \textbf{Dreieck:} L2 bei 120\,\%~$I_\mathrm{n}$ ca.\ \(-1{,}50\,\%\) (außerhalb) \(\Rightarrow\) ungeschirmte Rückseite, Einkopplung/partielle Sättigung.
    Überleitung: bei \SI{4000}{A} wird der Sättigungseffekt dominant.
    }
\end{frame}






\subsection{Messung bei 4000 A}


% --- 4000 A: Vergleich Standard (Dreieck) vs. Kompensiert ---
\begin{frame}{Standard (Dreieck) vs. Kompensiert}
    \centering
    \includegraphics[width=\textwidth, height=0.85\textheight, keepaspectratio]{03_Ressourcen/diagramme/verlauf_4000A_presentation.png}

    \slidetime{10:05}{11:20}
    \keynote{
        \begin{itemize}
            \item Fokus: \textbf{Standard in Dreieck} vs. \textbf{kompensierter Wandler}
            \item Standard (Dreieck): bis ca.\ 80--100\,\% relativ stabil, am oberen Lastpunkt deutliche Abweichungen (teils außerhalb Klasse 1)
            \item Kompensiert: in allen Phasen deutlich näher an 0\,\% \(\rightarrow\) Sättigungs-/Fremdfeldeinfluss stark reduziert
            \item Kernaussage: \textbf{Kompensation} ist bei \SI{4000}{A} wirksamer als die reine Geometrieoptimierung
        \end{itemize}
    }
    \note{
        Blickführung: hellblau gestrichelt = \textbf{Standard (Dreieck)}, dunkelblau = \textbf{kompensiert}.
        Standard (Dreieck): bis ca.\ 80--100\,\% stabil, am oberen Lastpunkt Ausreißer \(\Rightarrow\) Klasse~1 nicht durchgängig.
        Kompensiert: nahe 0\,\% \(\Rightarrow\) Fremdfeld-/Sättigungseinfluss stark reduziert.
        Take-away: bei \SI{4000}{A} ist \textbf{Kompensation} der dominante Hebel; Geometrie sekundär.
    }
\end{frame}






