% =============================================================================
% EIGENE BEFEHLE & GLOBALE EINSTELLUNGEN (befehle.tex)
% =============================================================================

% --- 1. GLOBALE DEFINITIONEN FÜR DIE TITELSEITE ---
% (Diese werden von 00_deckblatt.tex UND layout.tex genutzt)

% --- Titel & Autor (Variablen aus dem Repo) ---
\newcommand{\praktikumstitel}{Kolloquium}
\newcommand{\versuchstitel}{Fremdfeldbeeinflussung auf Messstromwandler in der Niederspannung} % Dein \subtitle
\newcommand{\autorenname}{Oliver-Luca Schmidt}


\newcommand{\betreuerEins}{Dr.-Ing. Sandro Günter}
\newcommand{\betreuerZwei}{Dipl.-Ing. Holger Kuhlemann}
\newcommand{\betreuerDrei}{Dipl.-Ing. Rainer Ludewig}
\newcommand{\betreuerVier}{Simon Westerbur, B. Eng.}

% --- Dein Kurztitel für die Fußzeile (aus Beamer) ---
\newcommand{\kurztitel}{Kolloquium} 

% --- Institut & Datum (aus Beamer) ---
\newcommand{\institut}{FB Technik | Abteilung Elektrotechnik und Informatik}
\newcommand{\company}{Rolf Janssen GmbH Elektrotechnische Werke}
\newcommand{\datum}{\today}

% (Platzhalter-Variablen aus dem Repo)
\newcommand{\semester}{7}
\newcommand{\versuchsnummer}{}
\newcommand{\gruppe}{}
\newcommand{\studiengang}{Elektrotechnik}


% =============================================================================
% --- 2. GLOBALE PFAD- UND BILD-EINSTELLUNGEN ---
% =============================================================================

% Definiert die Standard-Pfade für \includegraphics
% (Pfade relativ von main.tex, da \graphicspath global gilt)
\graphicspath{ 
  {03_Ressourcen/Bilder/},
  {03_Ressourcen/Logo/} 
}

% =============================================================================
% --- 3. BEREICH FÜR DEINE EIGENEN BEFEHLE ---
% =============================================================================
% \newcommand{\OPV}{Operationsverstärker}

% --- HIER: Angepasste Gliederung ---

% 1. Schriftart (normal/nicht-fett UND nicht-kursiv)
\setbeamerfont{section in toc}{series=\mdseries, shape=\upshape}

% 2. Textfarbe für Gliederung auf Schwarz setzen
\setbeamercolor{section in toc}{fg=black}

% 3. Symbol-Templates (mit mattem Pfeil)
\setbeamertemplate{section in toc}{%
 \vspace{-5cm}%
  {\textcolor{cHSblue}{$\blacktriangleright$}} % <--- HIER WAR DER FEHLER (das "\ " ist weg)
  \usebeamerfont{section in toc}% <--- Holt Schriftart (nicht-kursiv)
  \usebeamercolor[fg]{section in toc}%<--- Holt Textfarbe (jetzt schwarz)
  \inserttocsection%
   }

\setbeamertemplate{itemize subitem}{\textbullet}

% ###### Navigationssymbole entfernen ######
\setbeamertemplate{navigation symbols}{}