% --- 01_Einstellungen/praeambel.tex ---
% Lädt alle Pakete für die BEAMER-Präsentation

\usepackage[T1]{fontenc}
\usepackage[utf8]{inputenc}
\usepackage[ngerman]{babel}      % German style for date, sections ...
\usepackage{amsmath}
\usepackage{xcolor}
\usepackage{graphicx} % Wichtig für \includegraphics
\usepackage[export]{adjustbox}

% Nützliche Pakete aus der Repo-Vorlage (Beamer-kompatibel)
\usepackage{siunitx}  % Für korrekten Satz von Einheiten, z.B. \SI{10}{\volt}
\usepackage{booktabs} % Für schöne Tabellen
\usepackage{svg}      % Um SVG-Grafiken einzubinden
\usepackage{csquotes} % Für \enquote{}



% Du kannst hier alle weiteren \usepackage-Befehle hinzufügen
\usepackage{hyperref} % (Lädt Beamer i.d.R. automatisch)
\usepackage{listings}

% --- NEUE DEFINITIONEN (AUS BEISPIELCODE) ---

% Farben für den Fortschrittsbalken definieren
\setbeamercolor{progress bar background}{bg=gray!30}
\setbeamercolor{progress bar foreground}{bg=blue!80!black}

% NEUE FARBDEFINITION FÜR DIE BEISPIEL-COLORBOX
\definecolor{myexamplecolor}{RGB}{220,230,250} % Ein helles Beispiel-Blau
\setbeamercolor{examplebox}{bg=myexamplecolor,fg=black}