% --- Dateiname: 01_Einstellungen/praeambel.tex ---
% Beschreibung: Lädt alle Pakete und setzt technische Grundeinstellungen.

% 1. BASIS-EINSTELLUNGEN & KODIERUNG
\usepackage[T1]{fontenc}
\usepackage[utf8]{inputenc}
\usepackage[ngerman]{babel}      % Deutsche Bezeichnungen

% 2. BEAMER-SPEZIFISCHE PAKETE & FONTS
\usepackage{pgfpages}            % Nötig für Notizen auf dem zweiten Bildschirm
\usepackage{bookmark}            % Verbessertes Lesezeichen-Management
\usefonttheme[onlymath]{serif}   % Mathe-Schriften mit Serifen

% 3. MATHEMATIK & EINHEITEN
\usepackage{amsmath}
\usepackage{siunitx}  % Für korrekten Satz von Einheiten
\sisetup{
    locale = DE,
    range-units = single,
    list-units = single
}


\usepackage{textcomp} % für \texteuro
\DeclareSIUnit{\euro}{\text{\texteuro}}
% 4. GRAFIK & FARBEN
\usepackage{xcolor}
\usepackage{colortbl} % Ermöglicht \rowcolor in Tabellen
\usepackage{graphicx} 
\usepackage[export]{adjustbox} % Wichtig für 'valign=c'
\usepackage{svg}      % Um SVG-Grafiken einzubinden (optional)

% 5. TABELLEN & LAYOUT
\usepackage{booktabs} % Für schöne Tabellen
\usepackage{ragged2e} % Ermöglicht Silbentrennung im Flattersatz (\RaggedRight)

% 6. UTILITIES & SONSTIGES
\usepackage{csquotes} % Für \enquote{}
\usepackage{xurl}     % Bessere Umbrüche für URLs
\Urlmuskip=0mu plus 1mu

\usepackage{hyperref} 
\usepackage{listings} % Für Code-Listings
\lstset{
  breaklines=true,
  breakatwhitespace=true,
  columns=fullflexible,
  basicstyle=\ttfamily\small
}

% --- DUMMY-BEFEHLE (FALLBACK) ---
% Verhindert Fehler, falls layout.tex noch nicht geladen ist
\providecommand{\slidetime}[2]{}
\providecommand{\keynote}[1]{}