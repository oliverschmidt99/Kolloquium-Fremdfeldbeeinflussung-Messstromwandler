% --- 01_Einstellungen/praeambel.tex ---
% Lädt alle Pakete für die BEAMER-Präsentation

\usepackage[T1]{fontenc}
\usepackage[utf8]{inputenc}
\usepackage[ngerman]{babel}      % German style for date, sections ...
\usepackage{amsmath}
\usepackage{xcolor}
\usepackage{graphicx} % Wichtig für \includegraphics
\usepackage[export]{adjustbox}

% Nützliche Pakete aus der Repo-Vorlage (Beamer-kompatibel)
\usepackage{siunitx}  % Für korrekten Satz von Einheiten, z.B. \SI{10}{\volt}
\usepackage{booktabs} % Für schöne Tabellen
\usepackage{svg}      % Um SVG-Grafiken einzubinden
\usepackage{csquotes} % Für \enquote{}

% Du kannst hier alle weiteren \usepackage-Befehle hinzufügen
\usepackage{hyperref} % (Lädt Beamer i.d.R. automatisch)
\usepackage{listings}

% --- NEUE DEFINITIONEN (AUS BEISPIELCODE) ---

% Farben für den Fortschrittsbalken definieren
\setbeamercolor{progress bar background}{bg=gray!30}
\setbeamercolor{progress bar foreground}{bg=blue!80!black}

% NEUE FARBDEFINITION FÜR DIE BEISPIEL-COLORBOX
\definecolor{myexamplecolor}{RGB}{220,230,250} % Ein helles Beispiel-Blau
\setbeamercolor{examplebox}{bg=myexamplecolor,fg=black}


\makeatletter

% --- 1. FARBEN ---
\definecolor{MyPastelBackground}{RGB}{225, 250, 225} 
\definecolor{MySeparator}{RGB}{0, 0, 0} % Aktuell Schwarz (wie in deinem Code)
\definecolor{CrimsonRed}{RGB}{220, 20, 60}

% --- 2. LOGIK FÜR INHALTE (Keynote & Zeit) ---
% Variablen initialisieren
\gdef\insertkeynote{}
\gdef\inserttimestart{--:--} % Platzhalter, falls keine Zeit angegeben
\gdef\inserttimeend{--:--}

% Befehl 1: Wichtige Notiz (Links oben)
\newcommand{\keynote}[1]{\gdef\insertkeynote{#1}}

% Befehl 2: Zeitplanung (Rechts oben) - Nutzung: \slidetime{05:00}{07:30}
\newcommand{\slidetime}[2]{%
  \gdef\inserttimestart{#1}%
  \gdef\inserttimeend{#2}%
}

% AUTOMATISCHER RESET NACH JEDER SEITE
% Löscht Text und setzt Zeit auf --:-- zurück
\addtobeamertemplate{note page}{}{
   \gdef\insertkeynote{}
   \gdef\inserttimestart{--:--}
   \gdef\inserttimeend{--:--}
}


% --- 3. DAS NEUE LAYOUT (T-Form: Keynote (breit) | Zeit (schmal) // Notes) ---
\setbeamertemplate{note page}{
  % A) Hintergrund färben
  \insertvrule{\paperheight}{MyPastelBackground}%
  \vskip-\paperheight%
  \nointerlineskip%

  % B) Hauptcontainer
  \vbox to \paperheight{%
    
    % === REIHE 1: OBEN (45% der Höhe) ===
    \hbox to \paperwidth{%
        
        % --- FELD 1: LINKS OBEN (Keynote) -> JETZT BREITER (ca. 65%) ---
        \begin{minipage}[t][0.45\paperheight]{0.65\paperwidth}
            \vspace{0.5em}
            \centering
            \textbf{\large \color{CrimsonRed} Wichtig / Note 1:}
            \par \vspace{0.5em}
            % Text linksbündig (Breite anpassen auf 0.6)
            \begin{minipage}[t]{0.6\paperwidth}
                \raggedright \normalfont \small
                \insertkeynote
            \end{minipage}
        \end{minipage}%
        
        % ROTE LINIE (Vertikal)
        \textcolor{MySeparator}{\vrule width 1.5pt}%
        
        % --- FELD 2: RECHTS OBEN (Zeitplanung) -> JETZT SCHMALER (ca. 33%) ---
        \begin{minipage}[t][0.45\paperheight]{0.33\paperwidth}
            \vspace{1em}
            \centering
            \textbf{\large Zeitplanung}
            \par \vspace{1.em} % Abstand nach unten, damit Zeit mittig wirkt
            
            % Zeitanzeige (Sehr groß und gut lesbar)
            \begingroup
                \color{black}
                \Huge \textbf{\inserttimestart} \\[0.2em]
                \large \color{gray} bis \\[0.2em]
                \Huge \textbf{\inserttimeend}
            \endgroup
        \end{minipage}%
    }%
    
    \nointerlineskip
    % === ROTE LINIE (Horizontal) ===
    \textcolor{MySeparator}{\hrule height 1.5pt}%
    \nointerlineskip
    
    % === REIHE 2: UNTEN (Details / Note 2) ===
    \begin{minipage}[t][0.54\paperheight]{\paperwidth}
        \vspace{1em}
        \centering
        \textbf{\large Details / Notes 2:}
        \par \vspace{0.5em}
        \begin{minipage}[t]{0.95\paperwidth}
            \raggedright \normalfont \small \setlength{\parskip}{0.5em}
            \insertnote
        \end{minipage}
    \end{minipage}%
  }%
}
\makeatother