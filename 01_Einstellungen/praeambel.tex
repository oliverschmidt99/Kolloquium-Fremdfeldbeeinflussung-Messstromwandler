% --- 01_Einstellungen/praeambel.tex ---
% Lädt alle Pakete für die BEAMER-Präsentation

% 1. BASIS-EINSTELLUNGEN & KODIERUNG
\usepackage[T1]{fontenc}
\usepackage[utf8]{inputenc}
\usepackage[ngerman]{babel}      % Deutsche Bezeichnungen für Datum, Abschnitte etc.

% 2. BEAMER-SPEZIFISCHE PAKETE & FONTS
\usepackage{pgfpages}            % Nötig für Notizen auf dem zweiten Bildschirm
\usepackage{bookmark}            % Verbessertes Lesezeichen-Management
\usefonttheme[onlymath]{serif}   % Mathe-Schriften mit Serifen

% 3. MATHEMATIK & EINHEITEN
\usepackage{amsmath}
\usepackage{siunitx}  % Für korrekten Satz von Einheiten
% Konfiguration für siunitx
\sisetup{
    locale = DE,
    range-units = single,
    list-units = single
}

% 4. GRAFIK & FARBEN
\usepackage{xcolor}
\usepackage{colortbl} % <--- WICHTIG: Ermöglicht \rowcolor in Tabellen

\usepackage{graphicx} % Wichtig für \includegraphics
\usepackage[export]{adjustbox}
\usepackage{svg}      % Um SVG-Grafiken einzubinden

% 5. TABELLEN & LAYOUT
\usepackage{booktabs} % Für schöne Tabellen
\usepackage{ragged2e} % Ermöglicht Silbentrennung im Flattersatz (\RaggedRight)

% 6. UTILITIES & SONSTIGES
\usepackage{csquotes} % Für \enquote{}
\usepackage{xurl}     % Bessere Umbrüche für URLs
\Urlmuskip=0mu plus 1mu

\usepackage{hyperref} % (Lädt Beamer i.d.R. automatisch)
\usepackage{listings} % Für Code-Listings
\lstset{
  breaklines=true,
  breakatwhitespace=true,
  columns=fullflexible
}

% --- KONFIGURATION DER NOTIZEN ---
% Option: Notizen auf zweitem Bildschirm (Rechts)
\setbeameroption{show notes on second screen=right}

% --- NEUE DEFINITIONEN ---

% Farben für den Fortschrittsbalken definieren
\setbeamercolor{progress bar background}{bg=gray!30}
\setbeamercolor{progress bar foreground}{bg=blue!80!black}

% FARBDEFINITION FÜR DIE BEISPIEL-COLORBOX
\definecolor{myexamplecolor}{RGB}{220,230,250} 
\setbeamercolor{examplebox}{bg=myexamplecolor,fg=black}

\makeatletter

% --- 1. FARBEN FÜR NOTIZSEITEN ---
\definecolor{MyPastelBackground}{RGB}{225, 250, 225} 
\definecolor{MySeparator}{RGB}{0, 0, 0}
\definecolor{CrimsonRed}{RGB}{220, 20, 60}

% --- 2. LOGIK FÜR INHALTE (Keynote & Zeit) ---
\gdef\insertkeynote{}
\gdef\inserttimestart{--:--}
\gdef\inserttimeend{--:--}

% Befehl 1: Wichtige Notiz
\newcommand{\keynote}[1]{\gdef\insertkeynote{#1}}

% Befehl 2: Zeitplanung
\newcommand{\slidetime}[2]{%
  \gdef\inserttimestart{#1}%
  \gdef\inserttimeend{#2}%
}

% AUTOMATISCHER RESET NACH JEDER SEITE
\addtobeamertemplate{note page}{}{
   \gdef\insertkeynote{}
   \gdef\inserttimestart{--:--}
   \gdef\inserttimeend{--:--}
}

% TEMPLATE FÜR NOTIZSEITEN
\setbeamertemplate{note page}{
  \insertvrule{\paperheight}{MyPastelBackground}%
  \vskip-\paperheight%
  \nointerlineskip%

  \vbox to \paperheight{%
    % === REIHE 1: OBEN (Keynote & Zeit) ===
    \hbox to \paperwidth{%
      % FELD 1: LINKS OBEN (Keynote)
      \begin{minipage}[t][0.45\paperheight]{0.65\paperwidth}
        \vspace{0.5em}
        \centering
        \textbf{\large \color{CrimsonRed} Wichtig / Note 1}
        \par \vspace{0.5em}
        \begin{minipage}[t]{0.6\paperwidth}
          \adjustbox{max width=\linewidth, max height=0.30\paperheight}{%
            \parbox[t]{\linewidth}{%
              \RaggedRight\normalfont\small
              \setlength{\parindent}{0pt}%
              \setlength{\parskip}{0.4em}%
              \sloppy
              \insertkeynote
            }%
          }%
        \end{minipage}
      \end{minipage}%
      \textcolor{MySeparator}{\vrule width 1.5pt}%
      % FELD 2: RECHTS OBEN (Zeitplanung)
      \begin{minipage}[t][0.45\paperheight]{0.33\paperwidth}
        \vspace{1em}
        \centering
        \textbf{\large Zeitplanung}
        \par \vspace{1.em}
        \begingroup
          \color{black}
          \Huge \textbf{\inserttimestart} \\[0.2em]
          \large \color{gray} bis \\[0.2em]
          \Huge \textbf{\inserttimeend}
        \endgroup
      \end{minipage}%
    }%
    \nointerlineskip
    \textcolor{MySeparator}{\hrule height 1.5pt}%
    \nointerlineskip
    % === REIHE 2: UNTEN (Details / Note 2) ===
    \begin{minipage}[t][\textwidth]{\textwidth}
      \vspace{1em}
      \centering
      \begin{minipage}[t]{0.99\textwidth}
        \adjustbox{max height=0.50\textwidth, max width=\textwidth}{%
          \parbox[t]{\linewidth}{%
            \RaggedRight\sloppy\small
            \setlength{\parindent}{0pt}%
            \setlength{\parskip}{0.2em}%
            \insertnote
          }%
        }%
      \end{minipage}
    \end{minipage}
  }%
}
\makeatother