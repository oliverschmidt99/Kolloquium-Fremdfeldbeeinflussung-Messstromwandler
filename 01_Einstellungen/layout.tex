% --- Dateiname: 01_Einstellungen/layout.tex ---
% Beschreibung: Zentrales Design (Corporate Design), Farben, Templates und Notizen-Logik.

% ==========================================
% 1. BEAMER GRUNDEINSTELLUNGEN
% ==========================================
\clubpenalty = 10000
\widowpenalty = 10000
\hyphenpenalty = 1000
\setbeamercovered{transparent}

\useoutertheme{split}
\useinnertheme{default}
\usefonttheme{professionalfonts}

\setbeamertemplate{navigation symbols}{}
\setbeamertemplate{blocks}[rounded][shadow=false]

% Abstände
\parindent0.0cm
\parskip1.5ex plus0.5ex minus0.5ex

% ==========================================
% 2. FARBDEFINITIONEN & ZUWEISUNGEN
% ==========================================
% -- Definitionen --
\definecolor{JanssenDarkBlue}{HTML}{001F5F} 
\definecolor{cHSmint}{RGB}{138,198,203}
\definecolor{cHSfont}{RGB}{88,88,90}
\definecolor{logobg}{rgb}{0.537,0.7765,0.796}
\definecolor{fbtechnik}{gray}{.2} % Legacy

% -- Aliase --
\definecolor{cHSblue}{named}{JanssenDarkBlue}

% -- Zuweisungen (Beamer Colors) --
\setbeamercolor{normal text}{bg=white,fg=black}
\setbeamercolor{structure}{bg=white,fg=JanssenDarkBlue}

\setbeamercolor{title}{bg=white,fg=JanssenDarkBlue}
\setbeamercolor{subtitle}{bg=white,fg=JanssenDarkBlue}
\setbeamercolor{frametitle}{bg=JanssenDarkBlue,fg=white}
\setbeamercolor{framesubtitle}{bg=JanssenDarkBlue,fg=cHSmint}

\setbeamercolor{section in head/foot}{bg=white,fg=black}
\setbeamercolor{subsection in head/foot}{bg=cHSmint,fg=white}
\setbeamercolor{title in head/foot}{bg=white,fg=black}
\setbeamercolor{author in head/foot}{bg=JanssenDarkBlue,fg=white}

\setbeamercolor{block title}{bg=white,fg=JanssenDarkBlue}
\setbeamercolor{block body}{bg=white,fg=black}

\setbeamercolor{item projected}{fg=white,bg=JanssenDarkBlue}
\setbeamercolor{item}{fg=JanssenDarkBlue,bg=white}
\setbeamercolor{logo}{fg=logobg,bg=logobg}

\setbeamercolor{progress bar background}{bg=gray!30}
\setbeamercolor{progress bar foreground}{bg=JanssenDarkBlue}

% ==========================================
% 3. SCHRIFTARTEN (FONTS)
% ==========================================
\setbeamerfont{title}{size=\Huge, series=\bfseries}
\setbeamerfont{frametitle}{size=\Large, series=\bfseries}
\setbeamerfont{title in head/foot}{size=\normalsize}
\setbeamerfont{footline content}{size=\normalsize}

% ==========================================
% 4. LAYOUT: KOPF- UND FUSSZEILE
% ==========================================
\setbeamertemplate{headline}{}

\setbeamertemplate{footline}{%
  \leavevmode\hbox{%
    % Linker Teil: Sektionsname
    \begin{beamercolorbox}[wd=0.9\paperwidth,ht=2.25ex,dp=1ex,leftskip=1cm,left]{section in head/foot}%
      \usebeamerfont{footline content}\scriptsize\insertsectionhead%
    \end{beamercolorbox}%
    % Rechter Teil: Seitenzahl
    \begin{beamercolorbox}[wd=0.1\paperwidth,ht=2.25ex,dp=1ex,rightskip=1cm,right]{section in head/foot}%
      \usebeamerfont{footline content}\scriptsize\insertframenumber/\inserttotalframenumber%
    \end{beamercolorbox}%
  }%
  \vskip0pt%
  % Progress Bar
  \begin{beamercolorbox}[wd=\paperwidth,ht=1ex,dp=0ex]{progress bar background}%
    \ifnum\inserttotalframenumber>0%
      \begin{beamercolorbox}[wd=\dimexpr\paperwidth*\insertframenumber/\inserttotalframenumber\relax,ht=1ex,dp=0ex]{progress bar foreground}%
      \end{beamercolorbox}%
    \fi
  \end{beamercolorbox}%
}

% ==========================================
% 5. LAYOUT: FOLIENTITEL (FRAMETITLE)
% ==========================================
\setbeamertemplate{frametitle}{%
    \nointerlineskip%
    \begin{beamercolorbox}[wd=\paperwidth,ht=1.0cm,dp=0.7cm]{frametitle}%
        \begin{minipage}[c][1.4cm][c]{\paperwidth}%
            \hspace{0.5cm}%
            \begin{minipage}[c]{0.80\paperwidth}%
                \usebeamerfont{frametitle}\insertframetitle%
            \end{minipage}%
            \hfill%
            \begin{minipage}[c]{1.5cm}%
                \centering
                \includegraphics[height=1.2cm, valign=c]{03_Ressourcen/Logo/logo-rolf_janssen-ohne_text.pdf}%
            \end{minipage}%
            \hspace{0.5cm}%
        \end{minipage}%
    \end{beamercolorbox}%
}

% ==========================================
% 6. LAYOUT: DECKBLATT (TITLEFRAME)
% ==========================================
\newcommand{\titleframe}[1][]{
    \begin{frame}[plain]
        % Header Balken
        \vspace{-2px} 
        \begin{beamercolorbox}[wd=\paperwidth, ht=1.6cm, dp=1.1cm]{frametitle}
             \begin{minipage}[c][2.0cm][c]{\paperwidth}
                 \hspace{0.5cm}%
                 \raisebox{0.3cm}{\includegraphics[height=2.0cm, valign=c]{03_Ressourcen/Logo/hsel-logo-removebg-preview.png}}%
                 \hfill%
                 \raisebox{0.3cm}{\includegraphics[height=3.0cm, valign=c]{03_Ressourcen/Logo/logo-rolf_janssen-mit_text.pdf}}%
                 \hspace{0.5cm}%
             \end{minipage}
        \end{beamercolorbox}
        
        \vspace{0.5cm}
        
        % Titel
        \begin{center}
            \usebeamerfont{title}\usebeamercolor[fg]{title}
            \LARGE \inserttitle\par
            \vspace{0.1cm}
            \usebeamerfont{subtitle}\large\insertsubtitle\par
        \end{center}
        
        \vspace{0.2cm}
        
        % Info
        \begin{center}
            \insertinstitute
        \end{center}
        
        \vfill
        
        % Datum
        \centering
        \small{\insertdate}
        \vspace{0.3cm} 

        % Notizen-Argument
        #1 
    \end{frame}
    \setbeamertemplate{background canvas}{}
}

% ==========================================
% 7. INHALT: STANDARD-ELEMENTE (TOC, Listen)
% ==========================================
\setbeamertemplate{section in toc}{%
  \textcolor{JanssenDarkBlue}{$\blacktriangleright$}\ \inserttocsection\par\vskip0.1em
}
\setbeamertemplate{itemize subitem}{\textbullet}

% ==========================================
% 8. INHALT: SPEZIELLE BLÖCKE
% ==========================================
% E-Block (Blau)
\newenvironment{eblock}{
    \setbeamercolor{block body}{bg=cHSblue!40,fg=black}
    \begin{block}{}
    \vspace{-3px}
}{
    \end{block}
}

% W-Block (Hell)
\newenvironment{wblock}{
    \setbeamercolor{block body}{bg=logobg!2,fg=black}
    \begin{block}{}
    \vspace{-3px}
}{
    \end{block}
}

% ==========================================
% 9. FUNKTIONEN: NOTIZEN-LOGIK
% ==========================================
\definecolor{MyPastelBackground}{RGB}{225, 250, 225} 
\definecolor{MySeparator}{RGB}{0, 0, 0}
\definecolor{CrimsonRed}{RGB}{220, 20, 60}

\gdef\insertkeynote{}
\gdef\inserttimestart{--:--}
\gdef\inserttimeend{--:--}

\renewcommand{\keynote}[1]{\gdef\insertkeynote{#1}}
\renewcommand{\slidetime}[2]{\gdef\inserttimestart{#1}\gdef\inserttimeend{#2}}

\addtobeamertemplate{note page}{}{
   \gdef\insertkeynote{}
   \gdef\inserttimestart{--:--}
   \gdef\inserttimeend{--:--}
}

\makeatletter
\setbeamertemplate{note page}{
  \insertvrule{\paperheight}{MyPastelBackground}%
  \vskip-\paperheight%
  \nointerlineskip%
  \vbox to \paperheight{%
    % OBEN: Keynote & Zeit
    \hbox to \paperwidth{%
      \begin{minipage}[t][0.45\paperheight]{0.65\paperwidth}
        \vspace{0.5em}\centering
        \textbf{\large \color{CrimsonRed} Wichtig / Note 1}
        \par \vspace{0.5em}
        \begin{minipage}[t]{0.6\paperwidth}
          \adjustbox{max width=\linewidth, max height=0.30\paperheight}{%
            \parbox[t]{\linewidth}{%
              \RaggedRight\normalfont\small
              \setlength{\parindent}{0pt}\setlength{\parskip}{0.4em}\sloppy
              \insertkeynote
            }%
          }%
        \end{minipage}
      \end{minipage}%
      \textcolor{MySeparator}{\vrule width 1.5pt}%
      \begin{minipage}[t][0.45\paperheight]{0.33\paperwidth}
        \vspace{1em}\centering
        \textbf{\large Zeitplanung}
        \par \vspace{1.em}
        \begingroup
          \color{black}
          \Huge \textbf{\inserttimestart} \\[0.2em]
          \large \color{gray} bis \\[0.2em]
          \Huge \textbf{\inserttimeend}
        \endgroup
      \end{minipage}%
    }%
    \nointerlineskip
    \textcolor{MySeparator}{\hrule height 1.5pt}%
    \nointerlineskip
    % UNTEN: Details
    \begin{minipage}[t][\textwidth]{\textwidth}
      \vspace{1em}\centering
      \begin{minipage}[t]{0.99\textwidth}
        \adjustbox{max height=0.50\textwidth, max width=\textwidth}{%
          \parbox[t]{\linewidth}{%
            \RaggedRight\sloppy\small
            \setlength{\parindent}{0pt}\setlength{\parskip}{0.2em}
            \insertnote
          }%
        }%
      \end{minipage}
    \end{minipage}
  }%
}
\makeatother