% --- Dateiname: 01_Einstellungen/layout.tex ---
% Beschreibung: Definiert das Aussehen (Beamer-Theme, Farben, Kopf-/Fußzeile) und den Titleframe.

% --- Beamer Grundeinstellungen ---
\clubpenalty = 10000
\widowpenalty = 10000
\hyphenpenalty = 1000
\setbeamercovered{transparent}

\useoutertheme{split}
\useinnertheme{default}
\setbeamertemplate{blocks}[rounded][shadow=false]
\usefonttheme{professionalfonts}
\setbeamertemplate{navigation symbols}{}

% --- Farbdefinitionen ---
\definecolor{fbtechnik}{gray}{.2}
\definecolor{logobg}{rgb}{0.537,0.7765,0.796}
\definecolor{cHSblue}{RGB}{0,93,164}
\definecolor{cHSmint}{RGB}{138,198,203}
\definecolor{cHSfont}{RGB}{88,88,90}
\setbeamercolor{headline left}{bg=white}
\setbeamercolor{headline right}{bg=white}

% --- Farbzuweisungen ---
\setbeamercolor{section in head/foot}{bg=white,fg=black}
\setbeamercolor{subsection in head/foot}{bg=cHSmint,fg=white}
\setbeamercolor{block body}{bg=white,fg=black}
\setbeamercolor{block title}{bg=white,fg=cHSblue}
\setbeamercolor{item projected}{fg=black,bg=black!20}
\setbeamercolor{item}{fg=cHSblue,bg=black!20}
\setbeamercolor{frametitle}{bg=cHSblue,fg=white}
\setbeamercolor{framesubtitle}{bg=cHSblue!25,fg=black!80}
\setbeamercolor{title}{bg=cHSblue,fg=white}
\setbeamercolor{subtitle}{bg=cHSblue,fg=white}
\setbeamercolor{logo}{fg=logobg,bg=logobg}
\setbeamercolor{title in head/foot}{bg=white,fg=black}
\setbeamercolor{author in head/foot}{bg=cHSblue,fg=red}
\setbeamercolor{structure}{bg=white,fg=black}
\setbeamercolor{normal text}{bg=white,fg=black}
\setbeamercolor{progress bar background}{bg=gray!30}
\setbeamercolor{progress bar foreground}{bg=cHSblue}

% --- Schriftarten Kopf-/Fußzeile ---
\setbeamerfont{title in head/foot}{size=\normalsize}
\setbeamerfont{footline content}{size=\normalsize}
\setbeamerfont{section in toc}{series=\mdseries, shape=\upshape}
\setbeamercolor{section in toc}{fg=black}

% --- Templates (TOC, Itemize) ---
\setbeamertemplate{section in toc}{%
  \vspace{-5cm}%
  {\textcolor{cHSblue}{$\blacktriangleright$}}%
  \usebeamerfont{section in toc}%
  \usebeamercolor[fg]{section in toc}%
  \inserttocsection%
}
\setbeamertemplate{itemize subitem}{\textbullet}

% --- Kopfzeile (Headline) ---
\setbeamertemplate{headline}{
  \leavevmode%
  \hbox{%
    % Linke Box: Logo
    \begin{beamercolorbox}[wd=0.4\paperwidth,ht=1.0cm,dp=0.5cm,left]{headline left}
      \hspace{2cm} \vspace{-0.75cm}
      \texttt{\includegraphics[width=5cm, valign=c]{03_Ressourcen/Logo/logo_rolf-janssen_2024.pdf}}
    \end{beamercolorbox}%
    % Rechte Box: Titel
    \begin{beamercolorbox}[wd=0.6\paperwidth,ht=1.0cm,dp=0.5cm,right]{headline right}
      \usebeamerfont{title in head/foot}%
      \insertshorttitle%
      \hspace{1cm}
    \end{beamercolorbox}%
  }%
  \vspace{1mm}
}

% --- Fußzeile (Footline mit Progressbar) ---
\setbeamertemplate{footline}{%
  \leavevmode\hbox{%
    \begin{beamercolorbox}[wd=0.9\paperwidth,ht=2.25ex,dp=1ex,leftskip=1cm,left]{section in head/foot}%
      \usebeamerfont{footline content}\scriptsize\insertsectionhead%
    \end{beamercolorbox}%
    \begin{beamercolorbox}[wd=-0.9\paperwidth,ht=2.25ex,dp=1ex,rightskip=1cm,right]{section in head/foot}%
      \usebeamerfont{footline content}\scriptsize\insertframenumber/\inserttotalframenumber%
    \end{beamercolorbox}%
  }%
  \vskip0pt%
  \begin{beamercolorbox}[wd=\paperwidth,ht=1ex,dp=0ex]{progress bar background}%
    \ifnum\inserttotalframenumber>0%
      \begin{beamercolorbox}[wd=\dimexpr\paperwidth*\insertframenumber/\inserttotalframenumber\relax,ht=1ex,dp=0ex]{progress bar foreground}%
      \end{beamercolorbox}%
    \fi
  \end{beamercolorbox}%
}

% --- Folientitel (Frametitle) ---
\setbeamertemplate{frametitle}{\leavevmode\vbox{%
    \begin{beamercolorbox}[wd=1\paperwidth, ht=0.8cm, dp=0.3cm, center, center]{frametitle}
        \usebeamerfont{frametitle}\Large
        \strut\insertframetitle\strut\par%
        {%
            \ifx\insertframesubtitle\@empty%
            \else%
                \vskip-1.5ex 
                {\usebeamerfont{framesubtitle}\usebeamercolor[fg]{framesubtitle}\insertframesubtitle\strut\par}%
            \fi
        }%
    \end{beamercolorbox}%
}\vskip0pt}

% --- Definition: Titleframe ---
% Akzeptiert optionales Argument [1] für Notizen
\newcommand{\titleframe}[1][]{
    \begin{frame}[plain]
        % 1. Logos
        \begin{center}
            \includegraphics[width=5cm,valign=c]{03_Ressourcen/Logo/logo_rolf-janssen_2024.pdf}
            \hspace{0.5cm}
            \includegraphics[width=4cm,valign=c]{03_Ressourcen/Logo/logo.pdf}
        \end{center}
        
        \vspace{0.5cm}
        
        % 2. Titel-Block
        \begin{beamercolorbox}[wd=\paperwidth, sep=8pt, center]{title}
            \usebeamerfont{title}\Large \inserttitle\par
            \vspace{0.2cm}
            \usebeamerfont{subtitle}\large \insertsubtitle
        \end{beamercolorbox}
        
        \vspace{0.2cm} % Abstandshalter
        
        % 3. Info-Block
        \begin{center}
            \large{\autorenname} \\
            \vspace{1em}
            \large{\company} \\
            \vspace{1em}
            \small{
                Betreuung: \\
                \betreuerEins \\
                \betreuerDrei \\
            }
        \end{center}

        % 4. Notizen einfügen
        #1 

    \end{frame}
    \setbeamertemplate{background canvas}{}
}