\documentclass{beamer}
\usefonttheme[onlymath]{serif}
\usepackage{bookmark}
\usepackage{pgfpages}

% 1 EINSTELLUNGEN
% Zuerst die Pakete laden! (Hier drin steht vermutlich \usepackage{siunitx})
% --- 01_Einstellungen/praeambel.tex ---
% Lädt alle Pakete für die BEAMER-Präsentation

% 1. BASIS-EINSTELLUNGEN & KODIERUNG
\usepackage[T1]{fontenc}
\usepackage[utf8]{inputenc}
\usepackage[ngerman]{babel}      % Deutsche Bezeichnungen für Datum, Abschnitte etc.

% 2. BEAMER-SPEZIFISCHE PAKETE & FONTS
\usepackage{pgfpages}            % Nötig für Notizen auf dem zweiten Bildschirm
\usepackage{bookmark}            % Verbessertes Lesezeichen-Management
\usefonttheme[onlymath]{serif}   % Mathe-Schriften mit Serifen

% 3. MATHEMATIK & EINHEITEN
\usepackage{amsmath}
\usepackage{siunitx}  % Für korrekten Satz von Einheiten
% Konfiguration für siunitx
\sisetup{
    locale = DE,
    range-units = single,
    list-units = single
}

% 4. GRAFIK & FARBEN
\usepackage{xcolor}
\usepackage{colortbl} % <--- WICHTIG: Ermöglicht \rowcolor in Tabellen

\usepackage{graphicx} % Wichtig für \includegraphics
\usepackage[export]{adjustbox}
\usepackage{svg}      % Um SVG-Grafiken einzubinden

% 5. TABELLEN & LAYOUT
\usepackage{booktabs} % Für schöne Tabellen
\usepackage{ragged2e} % Ermöglicht Silbentrennung im Flattersatz (\RaggedRight)

% 6. UTILITIES & SONSTIGES
\usepackage{csquotes} % Für \enquote{}
\usepackage{xurl}     % Bessere Umbrüche für URLs
\Urlmuskip=0mu plus 1mu

\usepackage{hyperref} % (Lädt Beamer i.d.R. automatisch)
\usepackage{listings} % Für Code-Listings
\lstset{
  breaklines=true,
  breakatwhitespace=true,
  columns=fullflexible
}

% --- KONFIGURATION DER NOTIZEN ---
% Option: Notizen auf zweitem Bildschirm (Rechts)
\setbeameroption{show notes on second screen=right}

% --- NEUE DEFINITIONEN ---

% Farben für den Fortschrittsbalken definieren
\setbeamercolor{progress bar background}{bg=gray!30}
\setbeamercolor{progress bar foreground}{bg=blue!80!black}

% FARBDEFINITION FÜR DIE BEISPIEL-COLORBOX
\definecolor{myexamplecolor}{RGB}{220,230,250} 
\setbeamercolor{examplebox}{bg=myexamplecolor,fg=black}

\makeatletter

% --- 1. FARBEN FÜR NOTIZSEITEN ---
\definecolor{MyPastelBackground}{RGB}{225, 250, 225} 
\definecolor{MySeparator}{RGB}{0, 0, 0}
\definecolor{CrimsonRed}{RGB}{220, 20, 60}

% --- 2. LOGIK FÜR INHALTE (Keynote & Zeit) ---
\gdef\insertkeynote{}
\gdef\inserttimestart{--:--}
\gdef\inserttimeend{--:--}

% Befehl 1: Wichtige Notiz
\newcommand{\keynote}[1]{\gdef\insertkeynote{#1}}

% Befehl 2: Zeitplanung
\newcommand{\slidetime}[2]{%
  \gdef\inserttimestart{#1}%
  \gdef\inserttimeend{#2}%
}

% AUTOMATISCHER RESET NACH JEDER SEITE
\addtobeamertemplate{note page}{}{
   \gdef\insertkeynote{}
   \gdef\inserttimestart{--:--}
   \gdef\inserttimeend{--:--}
}

% TEMPLATE FÜR NOTIZSEITEN
\setbeamertemplate{note page}{
  \insertvrule{\paperheight}{MyPastelBackground}%
  \vskip-\paperheight%
  \nointerlineskip%

  \vbox to \paperheight{%
    % === REIHE 1: OBEN (Keynote & Zeit) ===
    \hbox to \paperwidth{%
      % FELD 1: LINKS OBEN (Keynote)
      \begin{minipage}[t][0.45\paperheight]{0.65\paperwidth}
        \vspace{0.5em}
        \centering
        \textbf{\large \color{CrimsonRed} Wichtig / Note 1}
        \par \vspace{0.5em}
        \begin{minipage}[t]{0.6\paperwidth}
          \adjustbox{max width=\linewidth, max height=0.30\paperheight}{%
            \parbox[t]{\linewidth}{%
              \RaggedRight\normalfont\small
              \setlength{\parindent}{0pt}%
              \setlength{\parskip}{0.4em}%
              \sloppy
              \insertkeynote
            }%
          }%
        \end{minipage}
      \end{minipage}%
      \textcolor{MySeparator}{\vrule width 1.5pt}%
      % FELD 2: RECHTS OBEN (Zeitplanung)
      \begin{minipage}[t][0.45\paperheight]{0.33\paperwidth}
        \vspace{1em}
        \centering
        \textbf{\large Zeitplanung}
        \par \vspace{1.em}
        \begingroup
          \color{black}
          \Huge \textbf{\inserttimestart} \\[0.2em]
          \large \color{gray} bis \\[0.2em]
          \Huge \textbf{\inserttimeend}
        \endgroup
      \end{minipage}%
    }%
    \nointerlineskip
    \textcolor{MySeparator}{\hrule height 1.5pt}%
    \nointerlineskip
    % === REIHE 2: UNTEN (Details / Note 2) ===
    \begin{minipage}[t][\textwidth]{\textwidth}
      \vspace{1em}
      \centering
      \begin{minipage}[t]{0.99\textwidth}
        \adjustbox{max height=0.50\textwidth, max width=\textwidth}{%
          \parbox[t]{\linewidth}{%
            \RaggedRight\sloppy\small
            \setlength{\parindent}{0pt}%
            \setlength{\parskip}{0.2em}%
            \insertnote
          }%
        }%
      \end{minipage}
    \end{minipage}
  }%
}
\makeatother
% --- Dateiname: 01_Einstellungen/layout.tex ---
% Beschreibung: Zentrales Design (Corporate Design), Farben, Templates und Notizen-Logik.

% ==========================================
% 1. BEAMER GRUNDEINSTELLUNGEN
% ==========================================
\clubpenalty = 10000
\widowpenalty = 10000
\hyphenpenalty = 1000
\setbeamercovered{transparent}

\useoutertheme{split}
\useinnertheme{default}
\usefonttheme{professionalfonts}

\setbeamertemplate{navigation symbols}{}
\setbeamertemplate{blocks}[rounded][shadow=false]

% Abstände
\parindent0.0cm
\parskip1.5ex plus0.5ex minus0.5ex

% ==========================================
% 2. FARBDEFINITIONEN & ZUWEISUNGEN
% ==========================================
% -- Definitionen --
\definecolor{JanssenDarkBlue}{HTML}{001F5F} 
\definecolor{cHSmint}{RGB}{138,198,203}
\definecolor{cHSfont}{RGB}{88,88,90}
\definecolor{logobg}{rgb}{0.537,0.7765,0.796}
\definecolor{fbtechnik}{gray}{.2} % Legacy

% -- Aliase --
\definecolor{cHSblue}{named}{JanssenDarkBlue}

% -- Zuweisungen (Beamer Colors) --
\setbeamercolor{normal text}{bg=white,fg=black}
\setbeamercolor{structure}{bg=white,fg=JanssenDarkBlue}

\setbeamercolor{title}{bg=white,fg=JanssenDarkBlue}
\setbeamercolor{subtitle}{bg=white,fg=JanssenDarkBlue}
\setbeamercolor{frametitle}{bg=JanssenDarkBlue,fg=white}
\setbeamercolor{framesubtitle}{bg=JanssenDarkBlue,fg=cHSmint}

\setbeamercolor{section in head/foot}{bg=white,fg=black}
\setbeamercolor{subsection in head/foot}{bg=cHSmint,fg=white}
\setbeamercolor{title in head/foot}{bg=white,fg=black}
\setbeamercolor{author in head/foot}{bg=JanssenDarkBlue,fg=white}

\setbeamercolor{block title}{bg=white,fg=JanssenDarkBlue}
\setbeamercolor{block body}{bg=white,fg=black}

\setbeamercolor{item projected}{fg=white,bg=JanssenDarkBlue}
\setbeamercolor{item}{fg=JanssenDarkBlue,bg=white}
\setbeamercolor{logo}{fg=logobg,bg=logobg}

\setbeamercolor{progress bar background}{bg=gray!30}
\setbeamercolor{progress bar foreground}{bg=JanssenDarkBlue}

% ==========================================
% 3. SCHRIFTARTEN (FONTS)
% ==========================================
\setbeamerfont{title}{size=\Huge, series=\bfseries}
\setbeamerfont{frametitle}{size=\Large, series=\bfseries}
\setbeamerfont{title in head/foot}{size=\normalsize}
\setbeamerfont{footline content}{size=\normalsize}

% ==========================================
% 4. LAYOUT: KOPF- UND FUSSZEILE
% ==========================================
\setbeamertemplate{headline}{}

\setbeamertemplate{footline}{%
  \leavevmode\hbox{%
    % Linker Teil: Sektionsname
    \begin{beamercolorbox}[wd=0.9\paperwidth,ht=2.25ex,dp=1ex,leftskip=1cm,left]{section in head/foot}%
      \usebeamerfont{footline content}\scriptsize\insertsectionhead%
    \end{beamercolorbox}%
    % Rechter Teil: Seitenzahl
    \begin{beamercolorbox}[wd=0.1\paperwidth,ht=2.25ex,dp=1ex,rightskip=1cm,right]{section in head/foot}%
      \usebeamerfont{footline content}\scriptsize\insertframenumber/\inserttotalframenumber%
    \end{beamercolorbox}%
  }%
  \vskip0pt%
  % Progress Bar
  \begin{beamercolorbox}[wd=\paperwidth,ht=1ex,dp=0ex]{progress bar background}%
    \ifnum\inserttotalframenumber>0%
      \begin{beamercolorbox}[wd=\dimexpr\paperwidth*\insertframenumber/\inserttotalframenumber\relax,ht=1ex,dp=0ex]{progress bar foreground}%
      \end{beamercolorbox}%
    \fi
  \end{beamercolorbox}%
}

% ==========================================
% 5. LAYOUT: FOLIENTITEL (FRAMETITLE)
% ==========================================
\setbeamertemplate{frametitle}{%
    \nointerlineskip%
    \begin{beamercolorbox}[wd=\paperwidth,ht=1.0cm,dp=0.7cm]{frametitle}%
        \begin{minipage}[c][1.4cm][c]{\paperwidth}%
            \hspace{0.5cm}%
            \begin{minipage}[c]{0.80\paperwidth}%
                \usebeamerfont{frametitle}\insertframetitle%
            \end{minipage}%
            \hfill%
            \begin{minipage}[c]{1.5cm}%
                \centering
                \includegraphics[height=1.2cm, valign=c]{03_Ressourcen/Logo/logo-rolf_janssen-ohne_text.pdf}%
            \end{minipage}%
            \hspace{0.5cm}%
        \end{minipage}%
    \end{beamercolorbox}%
}

% ==========================================
% 6. LAYOUT: DECKBLATT (TITLEFRAME)
% ==========================================
\newcommand{\titleframe}[1][]{
    \begin{frame}[plain]
        % Header Balken
        \vspace{-2px} 
        \begin{beamercolorbox}[wd=\paperwidth, ht=1.6cm, dp=1.1cm]{frametitle}
             \begin{minipage}[c][2.0cm][c]{\paperwidth}
                 \hspace{0.5cm}%
                 \raisebox{0.3cm}{\includegraphics[height=2.0cm, valign=c]{03_Ressourcen/Logo/hsel-logo-removebg-preview.png}}%
                 \hfill%
                 \raisebox{0.3cm}{\includegraphics[height=3.0cm, valign=c]{03_Ressourcen/Logo/logo-rolf_janssen-mit_text.pdf}}%
                 \hspace{0.5cm}%
             \end{minipage}
        \end{beamercolorbox}
        
        \vspace{0.5cm}
        
        % Titel
        \begin{center}
            \usebeamerfont{title}\usebeamercolor[fg]{title}
            \LARGE \inserttitle\par
            \vspace{0.1cm}
            \usebeamerfont{subtitle}\large\insertsubtitle\par
        \end{center}
        
        \vspace{0.2cm}
        
        % Info
        \begin{center}
            \insertinstitute
        \end{center}
        
        \vfill
        
        % Datum
        \centering
        \small{\insertdate}
        \vspace{0.3cm} 

        % Notizen-Argument
        #1 
    \end{frame}
    \setbeamertemplate{background canvas}{}
}

% ==========================================
% 7. INHALT: STANDARD-ELEMENTE (TOC, Listen)
% ==========================================
\setbeamertemplate{section in toc}{%
  \textcolor{JanssenDarkBlue}{$\blacktriangleright$}\ \inserttocsection\par\vskip0.1em
}
\setbeamertemplate{itemize subitem}{\textbullet}

% ==========================================
% 8. INHALT: SPEZIELLE BLÖCKE
% ==========================================
% E-Block (Blau)
\newenvironment{eblock}{
    \setbeamercolor{block body}{bg=cHSblue!40,fg=black}
    \begin{block}{}
    \vspace{-3px}
}{
    \end{block}
}

% W-Block (Hell)
\newenvironment{wblock}{
    \setbeamercolor{block body}{bg=logobg!2,fg=black}
    \begin{block}{}
    \vspace{-3px}
}{
    \end{block}
}

% ==========================================
% 9. FUNKTIONEN: NOTIZEN-LOGIK
% ==========================================
\definecolor{MyPastelBackground}{RGB}{225, 250, 225} 
\definecolor{MySeparator}{RGB}{0, 0, 0}
\definecolor{CrimsonRed}{RGB}{220, 20, 60}

\gdef\insertkeynote{}
\gdef\inserttimestart{--:--}
\gdef\inserttimeend{--:--}

\renewcommand{\keynote}[1]{\gdef\insertkeynote{#1}}
\renewcommand{\slidetime}[2]{\gdef\inserttimestart{#1}\gdef\inserttimeend{#2}}

\addtobeamertemplate{note page}{}{
   \gdef\insertkeynote{}
   \gdef\inserttimestart{--:--}
   \gdef\inserttimeend{--:--}
}

\makeatletter
\setbeamertemplate{note page}{
  \insertvrule{\paperheight}{MyPastelBackground}%
  \vskip-\paperheight%
  \nointerlineskip%
  \vbox to \paperheight{%
    % OBEN: Keynote & Zeit
    \hbox to \paperwidth{%
      \begin{minipage}[t][0.45\paperheight]{0.65\paperwidth}
        \vspace{0.5em}\centering
        \textbf{\large \color{CrimsonRed} Wichtig / Note 1}
        \par \vspace{0.5em}
        \begin{minipage}[t]{0.6\paperwidth}
          \adjustbox{max width=\linewidth, max height=0.30\paperheight}{%
            \parbox[t]{\linewidth}{%
              \RaggedRight\normalfont\small
              \setlength{\parindent}{0pt}\setlength{\parskip}{0.4em}\sloppy
              \insertkeynote
            }%
          }%
        \end{minipage}
      \end{minipage}%
      \textcolor{MySeparator}{\vrule width 1.5pt}%
      \begin{minipage}[t][0.45\paperheight]{0.33\paperwidth}
        \vspace{1em}\centering
        \textbf{\large Zeitplanung}
        \par \vspace{1.em}
        \begingroup
          \color{black}
          \Huge \textbf{\inserttimestart} \\[0.2em]
          \large \color{gray} bis \\[0.2em]
          \Huge \textbf{\inserttimeend}
        \endgroup
      \end{minipage}%
    }%
    \nointerlineskip
    \textcolor{MySeparator}{\hrule height 1.5pt}%
    \nointerlineskip
    % UNTEN: Details
    \begin{minipage}[t][\textwidth]{\textwidth}
      \vspace{1em}\centering
      \begin{minipage}[t]{0.99\textwidth}
        \adjustbox{max height=0.50\textwidth, max width=\textwidth}{%
          \parbox[t]{\linewidth}{%
            \RaggedRight\sloppy\small
            \setlength{\parindent}{0pt}\setlength{\parskip}{0.2em}
            \insertnote
          }%
        }%
      \end{minipage}
    \end{minipage}
  }%
}
\makeatother
% =============================================================================
% EIGENE BEFEHLE & GLOBALE EINSTELLUNGEN (befehle.tex)
% =============================================================================

% --- 1. GLOBALE DEFINITIONEN FÜR DIE TITELSEITE ---
% (Diese werden von 00_deckblatt.tex UND layout.tex genutzt)

% --- Titel & Autor (Variablen aus dem Repo) ---
\newcommand{\praktikumstitel}{Kolloquium}
\newcommand{\versuchstitel}{Fremdfeldbeeinflussung auf Messstromwandler in der Niederspannung} % Dein \subtitle
\newcommand{\autorenname}{Oliver-Luca Schmidt}


\newcommand{\betreuerEins}{Dr.-Ing. Sandro Günter}
\newcommand{\betreuerZwei}{Dipl.-Ing. Holger Kuhlemann}
\newcommand{\betreuerDrei}{Dipl.-Ing. Rainer Ludewig}
\newcommand{\betreuerVier}{Simon Westerbur, B. Eng.}

% --- Dein Kurztitel für die Fußzeile (aus Beamer) ---
\newcommand{\kurztitel}{Kolloquium} 

% --- Institut & Datum (aus Beamer) ---
\newcommand{\institut}{FB Technik | Abteilung Elektrotechnik und Informatik}
\newcommand{\company}{Rolf Janssen GmbH Elektrotechnische Werke}
\newcommand{\datum}{\today}

% (Platzhalter-Variablen aus dem Repo)
\newcommand{\semester}{7}
\newcommand{\versuchsnummer}{}
\newcommand{\gruppe}{}
\newcommand{\studiengang}{Elektrotechnik}


% =============================================================================
% --- 2. GLOBALE PFAD- UND BILD-EINSTELLUNGEN ---
% =============================================================================

% Definiert die Standard-Pfade für \includegraphics
% (Pfade relativ von main.tex, da \graphicspath global gilt)
\graphicspath{ 
  {03_Ressourcen/Bilder/},
  {03_Ressourcen/Logo/} 
}

% =============================================================================
% --- 3. BEREICH FÜR DEINE EIGENEN BEFEHLE ---
% =============================================================================
% \newcommand{\OPV}{Operationsverstärker}

% --- HIER: Angepasste Gliederung ---

% 1. Schriftart (normal/nicht-fett UND nicht-kursiv)
\setbeamerfont{section in toc}{series=\mdseries, shape=\upshape}

% 2. Textfarbe für Gliederung auf Schwarz setzen
\setbeamercolor{section in toc}{fg=black}

% 3. Symbol-Templates (mit mattem Pfeil)
\setbeamertemplate{section in toc}{%
 \vspace{-5cm}%
  {\textcolor{cHSblue}{$\blacktriangleright$}} % <--- HIER WAR DER FEHLER (das "\ " ist weg)
  \usebeamerfont{section in toc}% <--- Holt Schriftart (nicht-kursiv)
  \usebeamercolor[fg]{section in toc}%<--- Holt Textfarbe (jetzt schwarz)
  \inserttocsection%
   }

\setbeamertemplate{itemize subitem}{\textbullet}

% ###### Navigationssymbole entfernen ######
\setbeamertemplate{navigation symbols}{}

% JETZT erst konfigurieren, da das Paket nun geladen ist
\sisetup{
    locale = DE,
    range-units = single,
    list-units = single
}

% --- NOTIZEN KONFIGURATION ---
% Option: Notizen auf zweitem Bildschirm (Rechts)
\setbeameroption{show notes on second screen=right}

% 3 DOKUMENTEN INHALT
\begin{document}

% 3.1 VORSPANN
% --- 02_Inhalt/01_Vorspann/00_deckblatt.tex ---
\title[\kurztitel]{\praktikumstitel}
\subtitle{\versuchstitel}
\date{25. November 2025}

\institute{

  \Huge{\autorenname}

  \vspace{1em}
  \large{\company} \\
  \vspace{1em}
  \small{
    Betreuung: \\
    \betreuerEins \\
    \betreuerDrei \\
  }
}


\titleframe
\begin{frame}{Agenda}
    \tableofcontents

    \slidetime{01:00}{01:30}
    \keynote{
        \begin{itemize}
            \item Übersicht des Vortrags
            \item Roter Faden
            \item Von der Theorie zur Praxis
        \end{itemize}
    }
    \note{
        \textbf{Struktur}
        Ich habe meine Präsentation in folgende Bereiche gegliedert

        \textbf{Einleitung}
        Ich beginne mit der Motivation und der Problemstellung
        Daraus leite ich die Zielsetzung meiner Arbeit ab

        \textbf{Hauptteil}
        Anschließend gehe ich auf die theoretischen Grundlagen und den Versuchsaufbau ein
        Im Kernteil präsentiere ich die Messergebnisse und deren Analyse

        \textbf{Abschluss}
        Zum Schluss fasse ich die Erkenntnisse zusammen und gebe einen Ausblick
    }
\end{frame}
% Hier wird der Inhalt eingebunden, der weiter unten definiert ist
\section{Einleitung}

% --- Folie 1: Motivation ---
\begin{frame}{Motivation}
    \textbf{Kontext der Energiewende}
    \begin{itemize}
        \item Dezentralisierung erhöht Anforderungen an die Energieverteilung
        \item Niederspannungsschaltanlagen als zentrale Netzknoten
        \item Steigende Relevanz präziser Abrechnung und Netzstabilität
    \end{itemize}

    \vspace{0.5cm}

    \textbf{Konstruktiver Zielkonflikt}
    \begin{itemize}
        \item Wirtschaftliche Forderung nach kompakten Anlagen
        \item Hohe Packungsdichte der verbauten Komponenten
        \item Führung hoher Ströme auf engem Raum
        \item Räumliche Nähe von Sammelschienen und Messstromwandlern
    \end{itemize}

    \slidetime{01:30}{03:00}
    \keynote{
    \begin{itemize}
        \item Wandel der Anforderungen
        \item Platzmangel vs. Leistung
        \item Konflikt: Kompaktheit und Physik
    \end{itemize} }
    \note{
    \textbf{Stichpunkte zur Motivation:}
    \begin{itemize}
        \item Die Energiewende bringt neue Herausforderungen für die Verteilung
        \item Präzise Messwerte sind Geld wert (Abrechnung)
        \item Gleichzeitig müssen Anlagen immer kompakter und günstiger werden
        \item Das führt dazu, dass wir viel Strom auf wenig Raum haben
        \item Genau hier entsteht der Konflikt zwischen Baugröße und Messgenauigkeit
    \end{itemize} }
\end{frame}


% --- Folie 2: Problemstellung (Deine L2-Folie) ---
\begin{frame}{Problemstellung}


    \vspace{0.6em}
    \begin{center}
        \Large \alert{L2: \SI{130}{A} Messabweichung bei \SI{4000}{A}}\\
        \normalsize (\(\approx\) \SI{-3,28}{\%} – kritisch für Schutz und Abrechnung)
    \end{center}

    \vspace{0.3em}
    \begin{itemize}
        \item \textbf{Beobachtung:} \,Die Messung der Phase \textbf{L2} wird durch Fremdfelder benachbarter Leiter deutlich verfälscht
        \item \textbf{Ursache:} \,Kompakte Bauweise \(\rightarrow\) Messstromwandler und Sammelschienen liegen \textbf{sehr nahe beieinander}
    
        \item \textbf{Beispiel (L2, \SI{230}{V}):} \,\textbf{ca. 47\,000~€/a} mögliche Abrechnungsabweichung (bei \SI{8760}{h/a}, 0{,}20~€/kWh)
    \end{itemize}

% --- NOTIZEN GEHÖREN IN DEN FRAME ---
\slidetime{01:00}{02:00}
\keynote{
    \begin{itemize}
        \item Fokus auf \textbf{L2}: In kompakter Schienenanordnung koppeln Fremdfelder stark ein \(\rightarrow\) Messabweichung.
        \item Die Abweichung wirkt direkt auf \textbf{Energieverrechnung} und kann zudem Schutzorgane beeinflussen.
    \end{itemize}
}
\note{
    \begin{itemize}
        \item \textbf{Überschlag (nur Phase L2):} \(\Delta I=\SI{130}{A}\), \SI{230}{V}, \(\cos\varphi\approx 0{,}9\)
        \item \textbf{Leistungsdifferenz:} ca. \SI{27}{kW}
        \item \textbf{Energie/Jahr:} bei \SI{8760}{h/a} ca. \SI{235}{MWh/a}
        \item \textbf{Kosten/Jahr:} bei 0{,}20~€/kWh ca. 47\,000~€/a
        \item \textbf{Auswirkung:} \,Fehlerhafte Messwerte gefährden \textbf{Verrechnung}
        \item 47\,000~€/a ist pro Feld. 
    \end{itemize}
}
\end{frame}


% --- Folie 3: Zielsetzung ---
\begin{frame}{Zielsetzung der Arbeit}
    \textbf{Evaluation einer Lösung}
    \begin{itemize}
        \item Findung einer technisch zuverlässigen und wirtschaftlichen Konfiguration
        \item Sicherstellung der Messgenauigkeit unter Fremdfeldeinfluss
    \end{itemize}

    \vspace{0.5cm}

    \textbf{Untersuchungsschwerpunkte}
    \begin{itemize}
        \item Systematische Analyse der Fehler im Drehstromsystem (L1, L2, L3)
        \item Vergleich von Standardwandlern und kompensierten Spezialwandlern
        \item Prüfung konstruktiver Maßnahmen durch Anpassung der Leitergeometrie
        \item Ableitung von Handlungsempfehlungen für die Neukonstruktion
    \end{itemize}

    \slidetime{06:00}{07:00}
    \keynote{
    \begin{itemize}
        \item Standard vs. Spezial
        \item Geometrieoptimierung
        \item Wirtschaftlichkeit prüfen
    \end{itemize} }
    \note{
    \textbf{Ziele:}
    \begin{itemize}
        \item Es geht nicht nur um "Messung korrigieren", sondern um die beste Lösung
        \item Brauchen wir teure Spezialwandler? Oder reicht eine bessere Schienenführung?
        \item Ich vergleiche verschiedene Wandlertypen bei Strömen bis 5000 A
        \item Am Ende soll eine klare Empfehlung für die neue Anlagengeneration stehen
    \end{itemize} }
\end{frame}

% 3.2 HAUPTTEIL
\section{Grundlagen der Arbeit}

% --- Folie 1: Funktionsprinzip und Aufbau (Standard) ---
\begin{frame}{Funktionsprinzip und Aufbau}
    \begin{columns}[c]
        % Linke Spalte: Text
        \begin{column}{0.55\textwidth}
            \textbf{Aufgaben des Messstromwandlers}
            \begin{itemize}
                \item Transformation hoher Primärströme auf normierte Signale (\SI{1}{A} / \SI{5}{A})
                \item Galvanische Trennung zum Schutz der Messgeräte
                \item Bündelung des magnetischen Flusses durch Eisenkern
            \end{itemize}
        \end{column}

        % Rechte Spalte: Bild Standard
        \begin{column}{0.40\textwidth}
            \centering
            \includegraphics[width=1.2\textwidth]{03_Ressourcen/zeichnungen/aufbau_wandler.drawio.pdf}
            \par\vspace{0.2cm}
            {\tiny Prinzipieller Aufbau eines Aufsteckstromwandlers}
        \end{column}
    \end{columns}

    \slidetime{05:30}{06:15}
    \keynote{
        \begin{itemize}
            \item Transformator-Prinzip
            \item Schutzfunktion
            \item Normierte Signale
        \end{itemize}
    }
    \note{
        \textbf{Zum Aufbau:}
        Wir sehen hier rechts den schematischen Aufbau eines Standard-Wandlers. Die Kupferschiene ($P_1/P_2$) dient als Primärwicklung mit einer einzigen Windung.

        \textbf{Zur Funktion:}
        Der Eisenkern (grau) bündelt den magnetischen Fluss um den Leiter und induziert in der Sekundärwicklung ($S_1/S_2$) einen proportionalen, aber viel kleineren Strom.

        \textbf{Das Ziel:}
        Das ermöglicht uns, Ströme von mehreren tausend Ampere sicher vom Hochpotenzial zu trennen und für die Messgeräte auf 1 oder 5 Ampere zu normieren.
    }
\end{frame}

% --- Folie 2: Lösungsansatz 1 - Kompensation ---
\begin{frame}{Lösungsansatz: Kompensierte Wandler}
    \begin{columns}[c]
        % Linke Spalte: Text
        \begin{column}{0.55\textwidth}
            \textbf{Funktionsprinzip}
            \begin{itemize}
                \item Einsatz zusätzlicher Wicklungen auf dem Eisenkern
                \item Erzeugung eines magnetischen Gegenfeldes
                \item Aktive oder passive Korrektur des Übersetzungsfehlers
            \end{itemize}

            \vspace{0.4cm}

            \textbf{Vor- und Nachteile}
            \begin{itemize}
                \item Deutliche Reduktion der Sättigungseffekte
                \item Höherer konstruktiver Aufwand und Kostenfaktor
                \item Größerer Platzbedarf im Schaltschrank
            \end{itemize}
        \end{column}

        % Rechte Spalte: Bild Kompensation
        \begin{column}{0.40\textwidth}
            \centering
            \includegraphics[width=1\textwidth]{03_Ressourcen/zeichnungen/aufbau_wandler_kompensiert.drawio.pdf}
            \par\vspace{0.2cm}
            {\tiny Prinzip der Kompensationswicklung}
        \end{column}
    \end{columns}

    \slidetime{06:15}{07:00}
    \keynote{
        \begin{itemize}
            \item Gegenfeld erzeugen
            \item Fehler minimieren
            \item Aber: Teuer \& Groß
        \end{itemize}
    }
    \note{
        \textbf{Das Prinzip der Kompensation:}
        Eine technische Weiterentwicklung ist der kompensierte Wandler. Wie in der Grafik zu sehen, nutzen wir hier zusätzliche Wicklungen (W3/W4), um ein magnetisches Gegenfeld zu erzeugen.

        \textbf{Die Wirkung:}
        Dadurch wird der Arbeitspunkt des Kerns stabilisiert und Sättigungseffekte werden aktiv ausgeglichen. Der Wandler misst linearer.

        \textbf{Der Nachteil:}
        Allerdings ist dieser Aufbau deutlich komplexer. Das treibt die Kosten und – was fast noch kritischer ist – den Platzbedarf im Schaltschrank in die Höhe.
    }
\end{frame}

% --- Folie 3: Lösungsansatz 2 - FFP-Technologie ---
\begin{frame}{Lösungsansatz: Fremdfeld-Protektion (FFP)}

    \begin{columns}[c]
        % Linke Spalte: Text
        \begin{column}{0.55\textwidth}
            \textbf{Konstruktive Optimierung}
            \begin{itemize}
                \item Gezielte Schirmung des Messkerns
                \item Umleitung der magnetischen Störfeldlinien
                \item Prinzip des magnetischen Nebenschlusses
            \end{itemize}

            \vspace{0.5cm}

            \textbf{Zielsetzung}
            \begin{itemize}
                \item Einhaltung der Genauigkeitsklasse 1 trotz enger Phasenabstände
                \item Schutz vor partieller Sättigung durch Nachbarleiter
            \end{itemize}
        \end{column}

        % Rechte Spalte: Bild (Redur Patent)
        \begin{column}{0.40\textwidth}
            \centering
            \begin{beamercolorbox}[sep=0.5em,center,rounded=true,shadow=false]{white}
                \centering
                % Falls .png nicht existiert, Endung ggf. auf .pdf anpassen
                \includegraphics[width=0.95\textwidth]{03_Ressourcen/Bilder/wandler-ffp-redur-patent.png}
                \par\vspace{0.2cm}
                {\tiny \color{gray} Quelle: Patent DE102021106843A1 (Redur)}
            \end{beamercolorbox}
        \end{column}
    \end{columns}

    \slidetime{07:00}{07:45}
    \keynote{
        \begin{itemize}
            \item Schirmung als "Umleitung"
            \item Schutz des Kerns
            \item Klasse 1 sichern
        \end{itemize}
    }
    \note{
        \textbf{Die FFP-Technologie:}
        Eine Alternative ist die sogenannte Fremdfeld-Protektion (FFP). Das Bild rechts zeigt eine Patentschrift der Firma Redur dazu.

        \textbf{Das Prinzip (Schirmung):}
        Hier wird der eigentliche Messkern durch spezielle Leitbleche (im Bild die äußeren Schalen) abgeschirmt. Diese wirken wie ein magnetischer Nebenschluss und leiten die Feldlinien der Nachbarleiter am Kern vorbei.

        \textbf{Das Ziel:}
        Der innere Kern bleibt "sauber", gerät nicht in die Sättigung und soll so die Genauigkeitsklasse 1 auch bei engsten Abständen garantieren.
    }
\end{frame}

% --- Folie 4: Das Problem (Fremdfelder) ---
\begin{frame}{Physikalisches Problem: Fremdfeldeinfluss}

    \textbf{Ursache: Räumliche Nähe}
    \begin{itemize}
        \item Benachbarte Leiter erzeugen starke eigene Magnetfelder
        \item Diese Felder koppeln als \textbf{Störfluss} in den Wandlerkern ein
    \end{itemize}

    \vspace{0.4cm}

    \textbf{Wirkung: Partielle Sättigung}
    \begin{itemize}
        \item Der Störfluss addiert sich vektoriell zum Nutzfluss
        \item \textbf{Folge:} Der Kern gerät lokal in die magnetische Sättigung
        \item Die Permeabilität $\mu_r$ sinkt $\rightarrow$ Das Übertragungsverhältnis stimmt nicht mehr
    \end{itemize}

    \vspace{0.3cm}
    \begin{beamercolorbox}[sep=0.5em,center,rounded=true,shadow=false]{alerted text}
        \textbf{Resultat:} Der Sekundärstrom sinkt, die Messung zeigt zu wenig an.
    \end{beamercolorbox}

    \slidetime{07:45}{08:30}
    \keynote{
        \begin{itemize}
            \item Einkopplung von außen
            \item Kern ist „voll“ (Sättigung)
            \item Messwert sinkt ab
        \end{itemize}
    }
    \note{
        \textbf{Der physikalische Hintergrund:}
        Warum brauchen wir diese Technologien? Wenn wir Standard-Wandler in der engen Anlage verbauen, durchdringen die Magnetfelder der Nachbarleiter den Kern.

        \textbf{Der Mechanismus:}
        Der Eisenkern addiert den "Nutzfluss" und den "Störfluss". Ist die Summe zu groß, gerät das Eisen in die Sättigung.

        \textbf{Die Folge:}
        In der Sättigung verliert der Kern seine Leitfähigkeit für das Magnetfeld ($\mu_r$ sinkt). Er kann den Strom nicht mehr vollständig übertragen. Das Resultat ist genau der Einbruch, den wir bei Phase L2 gemessen haben.
    }
\end{frame}

% --- Folie 5: Normative Anforderungen (Klassen) ---
\begin{frame}{Normative Anforderungen (Klassen)}
    \textbf{Definition der Genauigkeit nach DIN EN 61869-2}
    \begin{itemize}
        \item Einteilung der Wandler in Genauigkeitsklassen
        \item Klasse definiert die maximal zulässige Messabweichung
    \end{itemize}

    \vspace{0.8cm}

    % Tabelle der Klassen (OHNE Anwendungsbereich)
    \centering
    \renewcommand{\arraystretch}{1.3}
    \begin{tabular}{l c}
        \toprule
        \textbf{Klasse} & \textbf{Fehler bei Nennstrom} \\
        \midrule
        \textbf{0,2 S}  & $\pm\, \SI{0,2}{\%}$          \\
        \rowcolor{cHSblue!10}
        \textbf{0,5}    & $\pm\, \SI{0,5}{\%}$          \\
        \rowcolor{cHSblue!10}
        \textbf{1}      & $\pm\, \SI{1,0}{\%}$          \\
        \textbf{3}      & $\pm\, \SI{3,0}{\%}$          \\
        \bottomrule
    \end{tabular}

    \vspace{0.8cm}
    \small \textit{Hinweis: Für diese Arbeit ist die Einhaltung der \textbf{Klasse 1} das Mindestziel.}

    \slidetime{08:30}{09:15}
    \keynote{
        \begin{itemize}
            \item Normierung nach IEC
            \item Ziel: Klasse 1
            \item Fehlergrenzen definiert
        \end{itemize}
    }
    \note{
        \textbf{Bewertungsgrundlage:}
        Um in den folgenden Ergebnissen zu entscheiden, ob eine Lösung tauglich ist, brauchen wir einen Maßstab. Die Norm definiert hierfür feste Genauigkeitsklassen.

        \textbf{Die Tabelle:}
        Wie wir sehen, darf ein Wandler der Klasse 1 bei Nennstrom maximal 1 Prozent abweichen.

        \textbf{Mein Ziel:}
        Das ist der Benchmark für meine Arbeit: Schaffen es die Wandler unter Fremdfeldeinfluss, diese \textbf{Klasse 1} zu halten? Alles, was schlechter ist, ist für eine Verrechnungsmessung inakzeptabel.
    }
\end{frame}
\begin{frame}{Fremdfeld und Wirkmechanismus}
    \begin{itemize}
        \item Überlagerung magnetischer Flüsse im Kern
        \item Lokale Sättigungserscheinungen im Eisenweg
        \item Asymmetrie der Hystereseschleife
        \item Abhängigkeit von der Phasenlage des Fremdstroms
        \item Geometrische Orientierung von Rückleiter und Wandler
    \end{itemize}
\end{frame}
\begin{frame}{Methodik und Versuchsaufbau}
    \textbf{Aufbau}
    \begin{itemize}
        \item Hochstromquelle bis Nennstrom
        \item Referenzwandler Klasse 0.1
        \item Variabler Störleiteraufbau
        \item Definierte Abstände und Winkel
    \end{itemize}

    \vspace{0.5cm}

    \textbf{Durchführung}
    \begin{itemize}
        \item Aufnahme der Fehlerkurven
        \item Variation der Primärstromstärke
        \item Änderung der Phasenverschiebung
    \end{itemize}
\end{frame}
\section{Ergebnisse}


% --- Folie: Quantifizierung Fremdfeld ---
\begin{frame}{Ergebnisse der Messung}

    \slidetime{--:--}{--:--}
    \keynote{
        \begin{itemize}
            \item empty
        \end{itemize}
    }
    \note{
        \begin{itemize}
            \item empty
        \end{itemize}
    }
\end{frame}

% 3.3 NACHSPANN
\begin{frame}{Fazit}
    \begin{itemize}
        \item Fremdfelder verursachen relevante Messabweichungen
        \item Einhaltung von Mindestabständen notwendig
        \item Kompaktanlagen erfordern besondere Schirmung
        \item Bestätigung der theoretischen Vorüberlegungen
        \item Sensibilisierung für Einbaugeometrie wichtig
    \end{itemize}
\end{frame}
\begin{frame}{Ausblick}
    \begin{itemize}
        \item Untersuchung weiterer Kernmaterialien
        \item Simulation komplexer Schienensysteme
        \item Entwicklung aktiver Kompensationsmethoden
        \item Erweiterung auf höhere Frequenzen
        \item Langzeitmessungen im Realbetrieb
    \end{itemize}
\end{frame}
\begin{frame}{Vielen Dank}
    \centering
    \textbf{Vielen Dank für die Aufmerksamkeit}
    
    \vspace{1cm}
    
    Offene Fragen
\end{frame}

% 3.4 BACKUP
\begin{frame}{Backup Folien}
    \centering
    \Huge Backup
\end{frame}

\begin{frame}{Detaillierte Spezifikationen}
    \begin{itemize}
        \item Wandlerdaten Typ XYZ
        \item Genauigkeitsklasse 0.5
        \item Bürde 15 VA
        \item Nennstrom 1000 A
    \end{itemize}
\end{frame}

\begin{frame}{Formelwerk}
    \begin{equation}
        B = \mu_0 \cdot \mu_r \cdot H
    \end{equation}
    \begin{itemize}
        \item Berechnung der Flussdichte
        \item Biot Savart Gesetz für Leiterfelder
    \end{itemize}
\end{frame}

\end{document}